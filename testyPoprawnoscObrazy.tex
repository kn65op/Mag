\subsubsection{Omówienie różnic w~obrazach wynikowych}
\label{subsec:prezentacjaObrazowRoznicowych}

Poniżej przedstawiono wyniki porównań pomiędzy obrazami otrzymanymi za pomocą różnych implementacji. Wykorzystano po dwa obrazy wejściowe dla każdej wykrywanej cechy oraz dla reprezentacji skali. Na każdym zestawie wyników znajdują się:
\begin{itemize}
\item obraz wejściowy,
\item wynik obliczeń dla implementacji OpenCL,
\item trzy obrazy różnicowe pozwalające porównać wszystkie implementacje.
\end{itemize}

Obrazy różnicowe są otoczone ramką w~celu zwiększenia czytelności. Obrazy zostały oznaczone liczbami rzymskimi określającymi kolejność pojawiania się ich w~pracy. Wyniki reprezentacji skali są przedstawione dla obrazów I~i~II, wynik detekcji plam dla obrazów III~i~IV, wynik detekcji krawędzi dla obrazów V~i~VI, wynik detekcji narożników dla obrazów VII~i~VIII, wynik detekcji grani dla obrazów IX~i~X.

\begin{itemize}

\newpage
\item{Reprezentacja skali}
\label{subsubsec:reprezentacjaSkaliRysunki}


Zestawy obrazów są przedstawione na rys. \ref{fig:valPure2} i~\ref{fig:valPure3}. Na podstawie obrazów różnicowych można stwierdzić, że główne rozbieżności pomiędzy implementacjami są niewielkie i~występują głównie na brzegach obrazu. Maksymalna wartość odchylenia nie przekracza $ 0,8\% $ maksymalnej wartości piksela. Rozbieżności pomiędzy implementacjami są spowodowane błędami powstającymi podczas realizacji obliczeń numerycznych (m. in. błąd zaokrąglenia). Istotne obiekty, które mają być analizowane znajdują się zazwyczaj w~środku obrazu, nie w~bezpośrednim sąsiedztwie krawędzi, dlatego różnice na brzegu obrazu nie są bardzo istotne dla dalszego przetwarzania. W centrum obrazu widać także niewielkie różnice, lecz są one niewielkie (podobnie, nie przekraczają $ 0,8\% $ maksymalnej wartości piksela, co się przekłada na zmianę bezwzględną wartości piksela o~2). Widać je zwłaszcza na obrazach \ref{fig:valPure3CLCVCPU} i~\ref{fig:valPure3CLCVGPU}. Można też zauważyć, że różnice występują pojedynczo, nie są zgrupowane.

Funkcje dostępne w~OpenCV, dla liczenia reprezentacji skali, dają identyczne wyniki, co można zauważyć na obrazach \ref{fig:valPure2CVCPUCVGPU} i~\ref{fig:valPure3CVCPUCVGPU}.

\begin{figure}[H]

\begin{center}
\begin{subfigure}[t]{0.3\textwidth}
\includegraphics[width=\textwidth]{TestyPoprawnosci/in_pure_02.png}
\caption{Obraz I, rodzielczość 3264x2176}
\label{fig:valPure02}
\end{subfigure}
~
\begin{subfigure}[t]{0.3\textwidth}
\includegraphics[width=\textwidth]{TestyPoprawnosci/in_pure_result_02_09.png}
\caption{Fragment reprezentacji skali (rozmiar skali 41)}
\label{fig:valPureResult02}
\end{subfigure}
\end{center}

\begin{subfigure}[t]{0.3\textwidth}
	\centering
	\setlength\fboxsep{0pt}
	\setlength\fboxrule{0.5pt}
	\fbox{\includegraphics[width=\textwidth]{TestyPoprawnosci/diff_pure_CL-CVCPU_02_09.png}}
	\caption{Porównanie implementacji CL i CVCPU dla reprezentacji skali}
	\label{fig:valPure2CLCVCPU}
\end{subfigure}
~
\begin{subfigure}[t]{0.3\textwidth}
	\centering
	\setlength\fboxsep{0pt}
	\setlength\fboxrule{0.5pt}
	\fbox{\includegraphics[width=\textwidth]{TestyPoprawnosci/diff_pure_CL-CVGPU_02_09.png}}
	\caption{Porównanie implementacji CL i CVGPU dla reprezentacji skali}
	\label{fig:valPure2CLCVGPU}
\end{subfigure}
~
\begin{subfigure}[t]{0.3\textwidth}
	\centering
	\setlength\fboxsep{0pt}
	\setlength\fboxrule{0.5pt}
	\fbox{\includegraphics[width=\textwidth]{TestyPoprawnosci/diff_pure_CVCPU-CVCPU_02_09.png}}
	\caption{Porównanie implementacji CVCPU i CVGPU dla reprezentacji skali}
	\label{fig:valPure2CVCPUCVGPU}                 
\end{subfigure}
\caption{Zestaw obrazów dla reprezentacji skali w~obrazie I, \tiny{źródło~własne}}

\label{fig:valPure2}
\end{figure}

\begin{figure}[H]
\begin{center}
\begin{subfigure}[t]{0.3\textwidth}
\includegraphics[width=\textwidth]{TestyPoprawnosci/in_pure_03.png}
\caption{Obraz II, rozdzielczość 1181x886}
\label{fig:valPure03}
\end{subfigure}
~
\begin{subfigure}[t]{0.3\textwidth}
\includegraphics[width=\textwidth]{TestyPoprawnosci/in_pure_result_03_09.png}
\caption{Fragment reprezentacji skali (rozmiar skali 41)}
\label{fig:valPure03}
\end{subfigure}
\end{center}

\begin{center}
\begin{subfigure}[t]{0.3\textwidth}
	\centering
	\setlength\fboxsep{0pt}
	\setlength\fboxrule{0.5pt}
	\fbox{\includegraphics[width=\textwidth]{TestyPoprawnosci/diff_pure_CL-CVCPU_03_09.png}}
	\caption{Porównanie implementacji CL i CVCPU dla reprezentacji skali}
	\label{fig:valPure3CLCVCPU}
\end{subfigure}
~
\begin{subfigure}[t]{0.3\textwidth}
	\centering
	\setlength\fboxsep{0pt}
	\setlength\fboxrule{0.5pt}
	\fbox{\includegraphics[width=\textwidth]{TestyPoprawnosci/diff_pure_CL-CVGPU_03_09.png}}
	\caption{Porównanie implementacji CL i CVGPU dla reprezentacji skali}
	\label{fig:valPure3CLCVGPU}
\end{subfigure}
~
\begin{subfigure}[t]{0.3\textwidth}
	\centering
	\setlength\fboxsep{0pt}
	\setlength\fboxrule{0.5pt}
	\fbox{\includegraphics[width=\textwidth]{TestyPoprawnosci/diff_pure_CVCPU-CVCPU_03_09.png}}
	\caption{Porównanie implementacji CVCPU i CVGPU dla reprezentacji skali}
	\label{fig:valPure3CVCPUCVGPU}                 
\end{subfigure}
\end{center}
\caption{Zestaw obrazów dla reprezentacji skali w~obrazie II, \tiny{źródło http://www.ptext.de/sites/default/files/1201/Optimierte\_Herstellungsprozesse\_bei\_Gaussspiegeln-127292.jpg}}

\label{fig:valPure3}
\end{figure}

\newpage

\item{Detekcja plam}
\label{subsubsec:plamyRysunki}

Zestawy obrazów są przedstawione na rys. \ref{fig:valBlob1} i~\ref{fig:valBlob2}. 

Z~przedstawionych obrazów wynika, że podczas detekcji plam, najwięcej różnic powstaje w~okolicach krawędzi. Jest to dobrze widoczne na obrazach \ref{fig:valBlob2CLCVCPU} i~\ref{fig:valBlob2CLCVGPU}. Porównując je do wyników testów dla reprezentacji skali można wywnioskować że te różnice są spowodowane rozbieżnościami powstałymi podczas filtracji Gaussa.

Na rys. \ref{fig:valBlob2} widać również różnice w~środku obrazu. Różnice te są spowodowane rozbieżnościami związanymi z~błędami powstałymi podczas obliczeń numerycznych (m. in. błąd zaokrąglenia). Liczba różnych pikseli nie jest duża, lecz większa niż w~przypadku innych wyznaczanych cech.

\begin{figure}[H]

\begin{center}
\begin{subfigure}[t]{0.3\textwidth}
\includegraphics[width=\textwidth]{TestyPoprawnosci/in_blob_01.png}
\caption{Obraz III, rozdzielczość 199x200}
\label{fig:valBlob01}
\end{subfigure}
~
\begin{subfigure}[t]{0.3\textwidth}
\includegraphics[width=\textwidth]{TestyPoprawnosci/in_blob_result_01_09.png}
\caption{Obraz wyjściowy dla detekcji plam (rozmiar skali 41)}
\label{fig:valBlobResult01}
\end{subfigure}
\end{center}


\begin{subfigure}[t]{0.3\textwidth}
	\centering
	\setlength\fboxsep{0pt}
	\setlength\fboxrule{0.5pt}
	\fbox{\includegraphics[width=\textwidth]{TestyPoprawnosci/diff_blob_CL-CVCPU_01_09.png}}
	\caption{Porównanie implementacji CL i CVCPU dla detekcji plam}
	\label{fig:valBlob2CLCVCPU}
\end{subfigure}
~
\begin{subfigure}[t]{0.3\textwidth}
	\centering
	\setlength\fboxsep{0pt}
	\setlength\fboxrule{0.5pt}
	\fbox{\includegraphics[width=\textwidth]{TestyPoprawnosci/diff_blob_CL-CVGPU_01_09.png}}
	\caption{Porównanie implementacji CL i CVGPU dla detekcji plam}
	\label{fig:valBlob2CLCVGPU}
\end{subfigure}
~
\begin{subfigure}[t]{0.3\textwidth}
	\centering
	\setlength\fboxsep{0pt}
	\setlength\fboxrule{0.5pt}
	\fbox{\includegraphics[width=\textwidth]{TestyPoprawnosci/diff_blob_CVCPU-CVCPU_01_09.png}}
	\caption{Porównanie implementacji CVCPU i CVGPU dla detekcji plam}
	\label{fig:valblob2CVCPUCVGPU}                 
\end{subfigure}
\caption{Zestaw obrazów dla detekcji plam w~obrazie III, \tiny{źródło~własne}}

\label{fig:valBlob1}
\end{figure}

\begin{figure}[H]
\begin{center}
\begin{subfigure}[t]{0.3\textwidth}
\includegraphics[width=\textwidth]{TestyPoprawnosci/in_blob_02.png}
\caption{Obraz IV, rozdzielczość 750x562}
\label{fig:valBlob02}
\end{subfigure}
~
\begin{subfigure}[t]{0.3\textwidth}
\includegraphics[width=\textwidth]{TestyPoprawnosci/in_blob_result_02_09.png}
\caption{Obraz wyjściowy dla detekcji plam (rozmiar skali 41)}
\label{fig:valBlobResult02}
\end{subfigure}
\end{center}


\begin{subfigure}[t]{0.3\textwidth}
	\centering
	\setlength\fboxsep{0pt}
	\setlength\fboxrule{0.5pt}
	\fbox{\includegraphics[width=\textwidth]{TestyPoprawnosci/diff_blob_CL-CVCPU_02_09.png}}
	\caption{Porównanie implementacji CL i CVCPU dla detekcji plam}
	\label{fig:valBlob3CLCVCPU}
\end{subfigure}
~
\begin{subfigure}[t]{0.3\textwidth}
	\centering
	\setlength\fboxsep{0pt}
	\setlength\fboxrule{0.5pt}
	\fbox{\includegraphics[width=\textwidth]{TestyPoprawnosci/diff_blob_CL-CVGPU_02_09.png}}
	\caption{Porównanie implementacji CL i CVGPU dla detekcji plam}
	\label{fig:valBlob3CLCVGPU}
\end{subfigure}
~
\begin{subfigure}[t]{0.3\textwidth}
	\centering
	\setlength\fboxsep{0pt}
	\setlength\fboxrule{0.5pt}
	\fbox{\includegraphics[width=\textwidth]{TestyPoprawnosci/diff_blob_CVCPU-CVCPU_02_09.png}}
	\caption{Porównanie implementacji CVCPU i CVGPU dla detekcji plam}
	\label{fig:valBlob3CVCPUCVGPU}                 
\end{subfigure}
\caption{Zestaw obrazów dla detekcji plam w~obrazie IV, \tiny{źródło~\texttt{http://www.free-picture.net/albums/flowers/sunflower/beautiful-flowers-ws.jpg}}}

\label{fig:valBlob2}
\end{figure}
\newpage

\item{Detekcja krawędzi}
\label{subsubsec:krawedzieRysunki}

Zestawy obrazów są przedstawione na rys. \ref{fig:valEdge0} i~\ref{fig:valEdge2}. Na obrazach różnicowych powstałych podczas detekcji krawędzi można zauważyć, że liczba różnych pikseli jest niewielka. Zdecydowana większość tych pikseli znajduje się na krawędziach. Jest to spowodowane różnicami powstałymi na etapie tworzenia reprezentacji skali. W~środku obrazu również można zauważyć różnice, lecz jest ich niewielka liczba.


\begin{figure}[H]

\begin{center}
\begin{subfigure}[t]{0.3\textwidth}
\includegraphics[width=\textwidth]{TestyPoprawnosci/in_edge_00.png}
\caption{Obraz V, rozdzielczość 1024x768}
\label{fig:valEdge00}
\end{subfigure}
~
\begin{subfigure}[t]{0.3\textwidth}
\includegraphics[width=\textwidth]{TestyPoprawnosci/in_edge_result_00_00.png}
\caption{Obraz wyjściowy dla detekcji krawędzi (rozmiar skali~5)}
\label{fig:valEdgeResult00}
\end{subfigure}
\end{center}

\begin{subfigure}[t]{0.3\textwidth}
	\centering
	\setlength\fboxsep{0pt}
	\setlength\fboxrule{0.5pt}
	\fbox{\includegraphics[width=\textwidth]{TestyPoprawnosci/diff_edge_CL-CVCPU_00_00.png}}
	\caption{Porównanie implementacji CL i CVCPU dla detekcji krawędzi}
	\label{fig:valEdge0CLCVCPU}
\end{subfigure}
~
\begin{subfigure}[t]{0.3\textwidth}
	\centering
	\setlength\fboxsep{0pt}
	\setlength\fboxrule{0.5pt}
	\fbox{\includegraphics[width=\textwidth]{TestyPoprawnosci/diff_edge_CL-CVGPU_00_00.png}}
	\caption{Porównanie implementacji CL i CVGPU dla detekcji krawędzi}
	\label{fig:valEdge0CLCVGPU}
\end{subfigure}
~
\begin{subfigure}[t]{0.3\textwidth}
	\centering
	\setlength\fboxsep{0pt}
	\setlength\fboxrule{0.5pt}
	\fbox{\includegraphics[width=\textwidth]{TestyPoprawnosci/diff_edge_CVCPU-CVCPU_00_00.png}}
	\caption{Porównanie implementacji CVCPU i CVGPU dla detekcji krawędzi}
	\label{fig:valEdge0CVCPUCVGPU}                 
\end{subfigure}
\caption{Zestaw obrazów dla detekcji krawędzi w~obrazie V, \tiny{źródło~obrazu \texttt{http://www.cs.princeton.edu/~cdecoro/edgedetection/csbldg.jpg}}}

\label{fig:valEdge0}
\end{figure}

\begin{figure}[H]

\begin{center}
\begin{subfigure}[t]{0.3\textwidth}
\includegraphics[width=\textwidth]{TestyPoprawnosci/in_edge_02.png}
\caption{Obraz VI, rozdzielczość 3200x2142}
\label{fig:valEdge02}
\end{subfigure}
~
\begin{subfigure}[t]{0.3\textwidth}
\includegraphics[width=\textwidth]{TestyPoprawnosci/in_edge_result_02_03.png}
\caption{Obraz wyjściowy dla detekcji krawędzi (rozmiar skali 17)}
\label{fig:valEdgeResult02}
\end{subfigure}
\end{center}

\begin{subfigure}[t]{0.3\textwidth}
	\centering
	\setlength\fboxsep{0pt}
	\setlength\fboxrule{0.5pt}
	\fbox{\includegraphics[width=\textwidth]{TestyPoprawnosci/diff_edge_CL-CVCPU_02_03.png}}
	\caption{Porównanie implementacji CL i CVCPU dla detekcji krawędzi}
	\label{fig:valEdge2CLCVCPU}
\end{subfigure}
~
\begin{subfigure}[t]{0.3\textwidth}
	\centering
	\setlength\fboxsep{0pt}
	\setlength\fboxrule{0.5pt}
	\fbox{\includegraphics[width=\textwidth]{TestyPoprawnosci/diff_edge_CL-CVGPU_02_03.png}}
	\caption{Porównanie implementacji CL i CVGPU dla detekcji krawędzi}
	\label{fig:valEdge2CLCVGPU}
\end{subfigure}
~
\begin{subfigure}[t]{0.3\textwidth}
	\centering
	\setlength\fboxsep{0pt}
	\setlength\fboxrule{0.5pt}
	\fbox{\includegraphics[width=\textwidth]{TestyPoprawnosci/diff_edge_CVCPU-CVCPU_02_03.png}}
	\caption{Porównanie implementacji CVCPU i CVGPU dla detekcji krawędzi}
	\label{fig:valEdge2CVCPUCVGPU}                 
\end{subfigure}
\caption{Zestaw obrazów dla detekcji krawędzi w~obrazie VI, \tiny{źródło~obrazu \texttt{http://stockarch.com/files/imagecache/Preview/12/04/basketball\_court\_from\_above.jpg}}}

\label{fig:valEdge2}
\end{figure}
\newpage

\item{Detekcja narożników}
\label{subsubsec:naroznikiRysunki}

Zestawy obrazów są przedstawione na rys. \ref{fig:valCorner1} i~\ref{fig:valCorner2}. Na obrazach różnicowych można zauważyć pojedyncze różnice. Jest to też efektem tego, że liczba wykrywanych narożników na przedstawionych obrazach w~wybranej skali nie jest duża. Mniejsza liczba wykrytych obiektów powoduje, że liczba błędów jest mała.

Na rys. \ref{fig:valCorner2} nie występują żadne różnice.
\begin{figure}[H]

\begin{center}
\begin{subfigure}[t]{0.3\textwidth}
\includegraphics[width=\textwidth]{TestyPoprawnosci/in_corner_01.png}
\caption{Obraz VII, rozdzielczość 640x480}
\label{fig:valCorner01}
\end{subfigure}
~
\begin{subfigure}[t]{0.3\textwidth}
\includegraphics[width=\textwidth]{TestyPoprawnosci/in_corner_result_01_07.png}
\caption{Obraz wyjściowy dla detekcji narożników (rozmiar skali 33}
\label{fig:valCornerResult01}
\end{subfigure}
\end{center}

\begin{subfigure}[t]{0.3\textwidth}
	\centering
	\setlength\fboxsep{0pt}
	\setlength\fboxrule{0.5pt}
	\fbox{\includegraphics[width=\textwidth]{TestyPoprawnosci/diff_corner_CL-CVCPU_01_07.png}}
	\caption{Porównanie implementacji CL i CVCPU dla detekcji narożników}
	\label{fig:valCorner1CLCVCPU}
\end{subfigure}
~
\begin{subfigure}[t]{0.3\textwidth}
	\centering
	\setlength\fboxsep{0pt}
	\setlength\fboxrule{0.5pt}
	\fbox{\includegraphics[width=\textwidth]{TestyPoprawnosci/diff_corner_CL-CVGPU_01_07.png}}
	\caption{Porównanie implementacji CL i CVGPU dla detekcji narożników}
	\label{fig:valCorner1CLCVGPU}
\end{subfigure}
~
\begin{subfigure}[t]{0.3\textwidth}
	\centering
	\setlength\fboxsep{0pt}
	\setlength\fboxrule{0.5pt}
	\fbox{\includegraphics[width=\textwidth]{TestyPoprawnosci/diff_corner_CVCPU-CVCPU_01_07.png}}
	\caption{Porównanie implementacji CVCPU i CVGPU dla detekcji narożników}
	\label{fig:valCorner1CVCPUCVGPU}                 
\end{subfigure}
\caption{Zestaw obrazów dla detekcji narożników w~obrazie VII, \tiny{źródło~\texttt{http://www.alanchard.com/cork/images/IMG\_3404.JPG}}}

\label{fig:valCorner1}
\end{figure}

\begin{figure}[H]

\begin{center}
\begin{subfigure}[t]{0.3\textwidth}
\includegraphics[width=\textwidth]{TestyPoprawnosci/in_corner_02.png}
\caption{Obraz VIII, rozdzielczość 256x256}
\label{fig:valCorner02}
\end{subfigure}
~
\begin{subfigure}[t]{0.3\textwidth}
\includegraphics[width=\textwidth]{TestyPoprawnosci/in_corner_result_02_06.png}
\caption{Obraz wyjściowy dla detekcji narożników (rozmiar skali 29)}
\label{fig:valCornerResult02}
\end{subfigure}
\end{center}

\begin{subfigure}[t]{0.3\textwidth}
	\centering
	\setlength\fboxsep{0pt}
	\setlength\fboxrule{0.5pt}
	\fbox{\includegraphics[width=\textwidth]{TestyPoprawnosci/diff_corner_CL-CVCPU_02_06.png}}
	\caption{Porównanie implementacji CL i CVCPU dla detekcji narożników}
	\label{fig:valCorner2CLCVCPU}
\end{subfigure}
~
\begin{subfigure}[t]{0.3\textwidth}
	\centering
	\setlength\fboxsep{0pt}
	\setlength\fboxrule{0.5pt}
	\fbox{\includegraphics[width=\textwidth]{TestyPoprawnosci/diff_corner_CL-CVGPU_02_06.png}}
	\caption{Porównanie implementacji CL i CVGPU dla detekcji narożników}
	\label{fig:valCorner2CLCVGPU}
\end{subfigure}
~
\begin{subfigure}[t]{0.3\textwidth}
	\centering
	\setlength\fboxsep{0pt}
	\setlength\fboxrule{0.5pt}
	\fbox{\includegraphics[width=\textwidth]{TestyPoprawnosci/diff_corner_CVCPU-CVCPU_02_06.png}}
	\caption{Porównanie implementacji CVCPU i CVGPU dla detekcji narożników}
	\label{fig:valCorner2CVCPUCVGPU}                 
\end{subfigure}
\caption{Zestaw obrazów dla detekcji narożników w~obrazie VIII, \tiny{źródło~\texttt{http://www.multimedia-computing.de/mediawiki//images/c/c4/J.jpg}}}

\label{fig:valCorner2}
\end{figure}
\newpage

\item{Detekcja grani}
\label{subsubsec:granieRysunki}

Zestawy obrazów są przedstawione na rys. \ref{fig:valRidge1} i~\ref{fig:valRidge4}. Na obrazach różnicowych można zauważyć, że na krawędziach powstało bardzo dużo różnic. Są one pochodną rozbieżności powstałych na etapie tworzenia reprezentacji skali. W~środku obrazu widać tylko pojedyncze różnice. 
\begin{figure}[H]

\begin{center}
\begin{subfigure}[t]{0.3\textwidth}
\includegraphics[width=\textwidth]{TestyPoprawnosci/in_ridge_01.png}
\caption{Obraz IX, rozdzielczość 750x707}
\label{fig:valRidge01}
\end{subfigure}
~
\begin{subfigure}[t]{0.3\textwidth}
\includegraphics[width=\textwidth]{TestyPoprawnosci/in_ridge_result_01_01.png}
\caption{Obraz wyjściowy dla detekcji grani (rozmiar skali 9)}
\label{fig:valRidgeResult01}
\end{subfigure}
\end{center}

\begin{subfigure}[t]{0.3\textwidth}
	\centering
	\setlength\fboxsep{0pt}
	\setlength\fboxrule{0.5pt}
	\fbox{\includegraphics[width=\textwidth]{TestyPoprawnosci/diff_ridge_CL-CVCPU_01_01.png}}
	\caption{Porównanie implementacji CL i CVCPU dla detekcji grani}
	\label{fig:valRidge1CLCVCPU}
\end{subfigure}
~
\begin{subfigure}[t]{0.3\textwidth}
	\centering
	\setlength\fboxsep{0pt}
	\setlength\fboxrule{0.5pt}
	\fbox{\includegraphics[width=\textwidth]{TestyPoprawnosci/diff_ridge_CL-CVGPU_01_01.png}}
	\caption{Porównanie implementacji CL i CVGPU dla detekcji grani}
	\label{fig:valRidge1CLCVGPU}
\end{subfigure}
~
\begin{subfigure}[t]{0.3\textwidth}
	\centering
	\setlength\fboxsep{0pt}
	\setlength\fboxrule{0.5pt}
	\fbox{\includegraphics[width=\textwidth]{TestyPoprawnosci/diff_ridge_CVCPU-CVCPU_01_01.png}}
	\caption{Porównanie implementacji CVCPU i CVGPU dla detekcji grani}
	\label{fig:valRidge1CVCPUCVGPU}                 
\end{subfigure}
\caption{Zestaw obrazów dla detekcji grani w~obrazie IX, \tiny{źródło~\texttt{http://wvs.topleftpixel.com/photos/newyork\_from\_above\_01b.jpg}}}

\label{fig:valRidge1}
\end{figure}

\begin{figure}[H]

\begin{center}
\begin{subfigure}[t]{0.3\textwidth}
\includegraphics[width=\textwidth]{TestyPoprawnosci/in_ridge_04.png}
\caption{Obraz X, rozdzielczość 560x450}
\label{fig:valRidge04}
\end{subfigure}
~
\begin{subfigure}[t]{0.3\textwidth}
\includegraphics[width=\textwidth]{TestyPoprawnosci/in_ridge_result_04_04.png}
\caption{Obraz wyjściowy dla detekcji grani (rozmiar skali 21)}
\label{fig:valRidgeResult04}
\end{subfigure}
\end{center}

\begin{subfigure}[t]{0.3\textwidth}
	\centering
	\setlength\fboxsep{0pt}
	\setlength\fboxrule{0.5pt}
	\fbox{\includegraphics[width=\textwidth]{TestyPoprawnosci/diff_ridge_CL-CVCPU_04_04.png}}
	\caption{Porównanie implementacji CL i CVCPU dla detekcji grani}
	\label{fig:valRidge4CLCVCPU}
\end{subfigure}
~
\begin{subfigure}[t]{0.3\textwidth}
	\centering
	\setlength\fboxsep{0pt}
	\setlength\fboxrule{0.5pt}
	\fbox{\includegraphics[width=\textwidth]{TestyPoprawnosci/diff_ridge_CL-CVGPU_04_04.png}}
	\caption{Porównanie implementacji CL i CVGPU dla detekcji grani}
	\label{fig:valRidge4CLCVGPU}
\end{subfigure}
~
\begin{subfigure}[t]{0.3\textwidth}
	\centering
	\setlength\fboxsep{0pt}
	\setlength\fboxrule{0.5pt}
	\fbox{\includegraphics[width=\textwidth]{TestyPoprawnosci/diff_ridge_CVCPU-CVCPU_04_04.png}}
	\caption{Porównanie implementacji CVCPU i CVGPU dla detekcji grani}
	\label{fig:valRidge4CVCPUCVGPU}                 
\end{subfigure}
\caption{Zestaw obrazów dla detekcji grani w~obrazie X, \tiny{źródło~\texttt{https://wesfiles.wesleyan.edu/courses/astr103/Lectures/triple.jpg}}}

\label{fig:valRidge4}
\end{figure}

\end{itemize}