\subsection{Prezentacja różnic w~obrazach wyjściowych}
\label{subsec:prezentacjaObrazowRoznicowych}

Poniżej przedstawiono wyniki porównań pomiędzy obrazami otrzymanymi za pomocą różnych implementacji. Przedstawione są tutaj jedynie dwa wybrane obrazy wejściowe wybrane spośród pięciu, dla każdej rozpoznawanej cechy oraz dla reprezentacji skali. Na każdym zestawie obrazów znajdują się:
\begin{itemize}
\item obraz wejciowy,
\item wynik obliczeń dla implementacji OpenCL,
\item trzy obrazy różnicowe porównujące wszystkie implementacje.
\end{itemize}

Obrazy różnicowe są otoczone ramką w~celu zwiększenia czytelności.

\subsubsection{Reprezentacja skali}
\label{subsubsec:reprezentacjaSkaliRysunki}

Obrazy wejściowe przedstawione są na rys. \ref{fig:valPure02} i \ref{fig:valPure03}. 

Obrazy różnicowe są przedstawione na rys. \ref{fig:valPure2} i~\ref{fig:valPure3}. 

\begin{figure}[h]

\begin{center}
\begin{subfigure}[t]{0.3\textwidth}
\includegraphics[width=\textwidth]{TestyPoprawnosci/in_pure_02.png}
\caption{Pierwszy obraz wejściowy dla reprezentacji skali}
\label{fig:valPure02}
\end{subfigure}
~
\begin{subfigure}[t]{0.3\textwidth}
\includegraphics[width=\textwidth]{TestyPoprawnosci/in_pure_result_02_09.png}
\caption{Pierwszy obraz wyjściowy dla reprezentacji skali}
\label{fig:valPureResult02}
\end{subfigure}
\end{center}

\begin{subfigure}[t]{0.3\textwidth}
	\centering
	\setlength\fboxsep{0pt}
	\setlength\fboxrule{0.5pt}
	\fbox{\includegraphics[width=\textwidth]{TestyPoprawnosci/diff_pure_CL-CVCPU_02_09.png}}
	\caption{Porównanie implementacji CL i CVCPU dla reprezentacji skali}
	\label{fig:valPure2CLCVCPU}
\end{subfigure}
~
\begin{subfigure}[t]{0.3\textwidth}
	\centering
	\setlength\fboxsep{0pt}
	\setlength\fboxrule{0.5pt}
	\fbox{\includegraphics[width=\textwidth]{TestyPoprawnosci/diff_pure_CL-CVGPU_02_09.png}}
	\caption{Porównanie implementacji CL i CVGPU dla reprezentacji skali}
	\label{fig:valPure2CLCVGPU}
\end{subfigure}
~
\begin{subfigure}[t]{0.3\textwidth}
	\centering
	\setlength\fboxsep{0pt}
	\setlength\fboxrule{0.5pt}
	\fbox{\includegraphics[width=\textwidth]{TestyPoprawnosci/diff_pure_CVCPU-CVCPU_02_09.png}}
	\caption{Porównanie implementacji CVCPU i CVGPU dla reprezentacji skali}
	\label{fig:valPure2CVCPUCVGPU}                 
\end{subfigure}
\caption{Obrazy różnicowe reprezentacji skali dla pierwszego pliku wejściowego}

\label{fig:valPure2}
\end{figure}

\begin{figure}[h]
\begin{center}
\begin{subfigure}[t]{0.3\textwidth}
\includegraphics[width=\textwidth]{TestyPoprawnosci/in_pure_03.png}
\caption{Pierwszy obraz wejściowy dla reprezentacji skali}
\label{fig:valPure02}
\end{subfigure}
~
\begin{subfigure}[t]{0.3\textwidth}
\begin{center}
\includegraphics[width=\textwidth]{TestyPoprawnosci/in_pure_result_03_09.png}
\end{center}
\caption{Pierwszy obraz wyjściowy dla reprezentacji skali}
\label{fig:valPure02}
\end{subfigure}
\end{center}

\begin{subfigure}[t]{0.3\textwidth}
	\centering
	\setlength\fboxsep{0pt}
	\setlength\fboxrule{0.5pt}
	\fbox{\includegraphics[width=\textwidth]{TestyPoprawnosci/diff_pure_CL-CVCPU_03_09.png}}
	\caption{Porównanie implementacji CL i CVCPU dla reprezentacji skali}
	\label{fig:valPure3CLCVCPU}
\end{subfigure}
~
\begin{subfigure}[t]{0.3\textwidth}
	\centering
	\setlength\fboxsep{0pt}
	\setlength\fboxrule{0.5pt}
	\fbox{\includegraphics[width=\textwidth]{TestyPoprawnosci/diff_pure_CL-CVGPU_03_09.png}}
	\caption{Porównanie implementacji CL i CVGPU dla reprezentacji skali}
	\label{fig:valPure3CLCVGPU}
\end{subfigure}
~
\begin{subfigure}[t]{0.3\textwidth}
	\centering
	\setlength\fboxsep{0pt}
	\setlength\fboxrule{0.5pt}
	\fbox{\includegraphics[width=\textwidth]{TestyPoprawnosci/diff_pure_CVCPU-CVCPU_03_09.png}}
	\caption{Porównanie implementacji CVCPU i CVGPU dla reprezentacji skali}
	\label{fig:valPure3CVCPUCVGPU}                 
\end{subfigure}
\caption{Obrazy różnicowe reprezentacji skali dla drugiego pliku wejściowego}

\label{fig:valPure3}
\end{figure}

Na podstawie obrazów różnicowych można stwierdzić, że główne rozbieżności pomiędzy implementacjami występują głównie na krawędziach. Poza tym różnice są niewielkie. Maksymalna wartość odchylenia nie przekracza $ 0,3\% $ maksymalnej wartości piksela. Rozbieżności pomiędzy implementacjami są spowodowane błędami związanymi z~ograniczaniemi systemów obliczeniowych (m. in. błąd zaokrąglenia). Istotne obiekty, które mają być rozpoznane znajdują się w~środku obrazu, nie w~bezpośrednim sąsiedztwie krawędzi, dlatego różnice na brzegu obrazu nie są istotne dla dalszego przetwarzania. W centrum obrazu widać także niewielkie różne, lecz są one niewielkie (podobnie, nie przekraczają $ 0,3\% $ maksymalnej wartości piksela). Widać je zwłaszcza na obrazach \ref{fig:valPure3CLCVCPU} i~\ref{fig:valPure3CLCVCPU}. Można też zauważyć, że różnice występują pojedynczo, nie są zgrupowane.

Implementacje wykonane w~OpenCV dla liczenia reprezentacji skali dają identyczne wyniki, co można zauważyć na obrazach \ref{fig:valPure2CVCPUCVGPU} i~\ref{fig:valPure3CVCPUCVGPU}.

\subsubsection{Plamy}
\label{subsubsec:plamyRysunki}

Obrazy wejściowe przedstawione są na rys. \ref{fig:valBlob01} i \ref{fig:valBlob02}.

Obrazy różnicowe są przedstawione na rys. \ref{fig:valBlob1} i~\ref{fig:valBlob2}. 

\begin{figure}[h]

\begin{center}
\begin{subfigure}[t]{0.3\textwidth}
\includegraphics[width=\textwidth]{TestyPoprawnosci/in_blob_01.png}
\caption{Pierwszy obraz wejściowy dla reprezentacji skali}
\label{fig:valBlob02}
\end{subfigure}
~
\begin{subfigure}[t]{0.3\textwidth}
\includegraphics[width=\textwidth]{TestyPoprawnosci/in_blob_result_01_09.png}
\caption{Pierwszy obraz wyjściowy dla reprezentacji skali}
\label{fig:valBlobResult02}
\end{subfigure}
\end{center}


\begin{subfigure}[t]{0.3\textwidth}
	\centering
	\setlength\fboxsep{0pt}
	\setlength\fboxrule{0.5pt}
	\fbox{\includegraphics[width=\textwidth]{TestyPoprawnosci/diff_blob_CL-CVCPU_01_09.png}}
	\caption{Porównanie implementacji CL i CVCPU dla rozpoznawania plam}
	\label{fig:valBlob2CLCVCPU}
\end{subfigure}
~
\begin{subfigure}[t]{0.3\textwidth}
	\centering
	\setlength\fboxsep{0pt}
	\setlength\fboxrule{0.5pt}
	\fbox{\includegraphics[width=\textwidth]{TestyPoprawnosci/diff_blob_CL-CVGPU_01_09.png}}
	\caption{Porównanie implementacji CL i CVGPU dla rozpoznawania plam}
	\label{fig:valBlob2CLCVGPU}
\end{subfigure}
~
\begin{subfigure}[t]{0.3\textwidth}
	\centering
	\setlength\fboxsep{0pt}
	\setlength\fboxrule{0.5pt}
	\fbox{\includegraphics[width=\textwidth]{TestyPoprawnosci/diff_blob_CVCPU-CVCPU_01_09.png}}
	\caption{Porównanie implementacji CVCPU i CVGPU dla rozpoznawania plam}
	\label{fig:valblob2CVCPUCVGPU}                 
\end{subfigure}
\caption{Obrazy różnicowe rozpoznawania plam dla pierwszego pliku wejściowego}

\label{fig:valBlob1}
\end{figure}

\begin{figure}[h]
\begin{center}
\begin{subfigure}[t]{0.3\textwidth}
\includegraphics[width=\textwidth]{TestyPoprawnosci/in_blob_02.png}
\caption{Drugi obraz wejściowy dla rozpoznawania plam}
\label{fig:valBlob02}
\end{subfigure}
~
\begin{subfigure}[t]{0.3\textwidth}
\includegraphics[width=\textwidth]{TestyPoprawnosci/in_blob_result_02_09.png}
\caption{Drugi obraz wyjściowy dla reprezentacji skali}
\label{fig:valBlobResult02}
\end{subfigure}
\end{center}


\begin{subfigure}[t]{0.3\textwidth}
	\centering
	\setlength\fboxsep{0pt}
	\setlength\fboxrule{0.5pt}
	\fbox{\includegraphics[width=\textwidth]{TestyPoprawnosci/diff_blob_CL-CVCPU_02_09.png}}
	\caption{Porównanie implementacji CL i CVCPU dla rozpoznawania plam}
	\label{fig:valBlob3CLCVCPU}
\end{subfigure}
~
\begin{subfigure}[t]{0.3\textwidth}
	\centering
	\setlength\fboxsep{0pt}
	\setlength\fboxrule{0.5pt}
	\fbox{\includegraphics[width=\textwidth]{TestyPoprawnosci/diff_blob_CL-CVGPU_02_09.png}}
	\caption{Porównanie implementacji CL i CVGPU dla rozpoznawania plam}
	\label{fig:valBlob3CLCVGPU}
\end{subfigure}
~
\begin{subfigure}[t]{0.3\textwidth}
	\centering
	\setlength\fboxsep{0pt}
	\setlength\fboxrule{0.5pt}
	\fbox{\includegraphics[width=\textwidth]{TestyPoprawnosci/diff_blob_CVCPU-CVCPU_02_09.png}}
	\caption{Porównanie implementacji CVCPU i CVGPU dla rozpoznawania plam}
	\label{fig:valBlob3CVCPUCVGPU}                 
\end{subfigure}
\caption{Obrazy różnicowe rozpoznawania plam dla drugiego pliku wejściowego}

\label{fig:valBlob2}
\end{figure}

Z~przedstawionych obrazów wynika, że podczas rozpoznawania plam, najwięcej różnic powstaje w~okolicach krawędzi. Jest to dobrze widoczne na obrazach \ref{fig:valBlob2CLCVCPU} i~\ref{fig:valBlob2CLCVGPU}. Porównując do wyników testów dla reprezentacji skali można wywnioskować że te różnice są spowodowane rozbieżnościami powstałymi podczas filtracji Gaussa.

Na rys. \ref{fig:valBlob2} widać również różnice w~środku obrazu. Różnice te są spowodowane błędami związanymi z~ograniczaniemi systemów obliczeniowych (m. in. błąd zaokrąglenia). Liczba różnych pikseli nie jest duża, lecz większa niż w~przypadku innych rozpoznawanych cech.


\subsubsection{Krawędzie}
\label{subsubsec:krawedzieRysunki}

Obrazy wejściowe przedstawione są na rys. \ref{fig:valEdge00} i \ref{fig:valEdge02}.

Obrazy różnicowe są przedstawione na rys. \ref{fig:valEdge0} i~\ref{fig:valEdge2}. 

\begin{figure}[h]

\begin{center}
\begin{subfigure}[t]{0.3\textwidth}
\includegraphics[width=\textwidth]{TestyPoprawnosci/in_edge_00.png}
\caption{Pierwszy obraz wejściowy dla rozpoznawania krawędzi}
\label{fig:valEdge00}
\end{subfigure}
~
\begin{subfigure}[t]{0.3\textwidth}
\includegraphics[width=\textwidth]{TestyPoprawnosci/in_edge_result_00_00.png}
\caption{Pierwszy obraz wyjściowy dla rozpoznawania krawędzi}
\label{fig:valEdgeResult00}
\end{subfigure}
\end{center}

\begin{subfigure}[t]{0.3\textwidth}
	\centering
	\setlength\fboxsep{0pt}
	\setlength\fboxrule{0.5pt}
	\fbox{\includegraphics[width=\textwidth]{TestyPoprawnosci/diff_edge_CL-CVCPU_00_00.png}}
	\caption{Porównanie implementacji CL i CVCPU dla rozpoznawania krawędzi}
	\label{fig:valEdge0CLCVCPU}
\end{subfigure}
~
\begin{subfigure}[t]{0.3\textwidth}
	\centering
	\setlength\fboxsep{0pt}
	\setlength\fboxrule{0.5pt}
	\fbox{\includegraphics[width=\textwidth]{TestyPoprawnosci/diff_edge_CL-CVGPU_00_00.png}}
	\caption{Porównanie implementacji CL i CVGPU dla rozpoznawania krawędzi}
	\label{fig:valEdge0CLCVGPU}
\end{subfigure}
~
\begin{subfigure}[t]{0.3\textwidth}
	\centering
	\setlength\fboxsep{0pt}
	\setlength\fboxrule{0.5pt}
	\fbox{\includegraphics[width=\textwidth]{TestyPoprawnosci/diff_edge_CVCPU-CVCPU_00_00.png}}
	\caption{Porównanie implementacji CVCPU i CVGPU dla rozpoznawania krawędzi}
	\label{fig:valEdge0CVCPUCVGPU}                 
\end{subfigure}
\caption{Obrazy różnicowe rozpoznawania krawędzi dla pierwszego pliku wejściowego}

\label{fig:valEdge0}
\end{figure}

\begin{figure}[h]

\begin{center}
\begin{subfigure}[t]{0.3\textwidth}
\includegraphics[width=\textwidth]{TestyPoprawnosci/in_edge_02.png}
\caption{Drugi obraz wejściowy dla rozpoznawania krawędzi}
\label{fig:valEdge02}
\end{subfigure}
~
\begin{subfigure}[t]{0.3\textwidth}
\includegraphics[width=\textwidth]{TestyPoprawnosci/in_edge_result_02_03.png}
\caption{Drugi obraz wyjściowy dla rozpoznawania krawędzi}
\label{fig:valEdgeResult02}
\end{subfigure}
\end{center}

\begin{subfigure}[t]{0.3\textwidth}
	\centering
	\setlength\fboxsep{0pt}
	\setlength\fboxrule{0.5pt}
	\fbox{\includegraphics[width=\textwidth]{TestyPoprawnosci/diff_edge_CL-CVCPU_02_03.png}}
	\caption{Porównanie implementacji CL i CVCPU dla rozpoznawania krawędzi}
	\label{fig:valEdge2CLCVCPU}
\end{subfigure}
~
\begin{subfigure}[t]{0.3\textwidth}
	\centering
	\setlength\fboxsep{0pt}
	\setlength\fboxrule{0.5pt}
	\fbox{\includegraphics[width=\textwidth]{TestyPoprawnosci/diff_edge_CL-CVGPU_02_03.png}}
	\caption{Porównanie implementacji CL i CVGPU dla rozpoznawania krawędzi}
	\label{fig:valEdge2CLCVGPU}
\end{subfigure}
~
\begin{subfigure}[t]{0.3\textwidth}
	\centering
	\setlength\fboxsep{0pt}
	\setlength\fboxrule{0.5pt}
	\fbox{\includegraphics[width=\textwidth]{TestyPoprawnosci/diff_edge_CVCPU-CVCPU_02_03.png}}
	\caption{Porównanie implementacji CVCPU i CVGPU dla rozpoznawania krawędzi}
	\label{fig:valEdge2CVCPUCVGPU}                 
\end{subfigure}
\caption{Obrazy różnicowe rozpoznawania krawędzi dla drugiego pliku wejściowego}

\label{fig:valEdge2}
\end{figure}

Na obrazach różnicowych powstałych poczas rozpoznawania krawędzi można zauważyć, że licza różnych pikseli jest niewielka. Zdecydowana większość tych pikseli znajduje się na krawędziach. Jest to spowodowane różnicami powstałymi na etapie tworzenia reprezentacji skali. W~środku obrazu również można zauważyć różnice, lecz jest ich niewielka liczba.

\subsubsection{Narożniki}
\label{subsubsec:naroznikiRysunki}

Obrazy wejściowe przedstawione są na rys. \ref{fig:valCorner01} i \ref{fig:valCorner02}.

Obrazy różnicowe są przedstawione na rys. \ref{fig:valCorner1} i~\ref{fig:valCorner2}. 

\begin{figure}[h]

\begin{center}
\begin{subfigure}[t]{0.3\textwidth}
\includegraphics[width=\textwidth]{TestyPoprawnosci/in_corner_01.png}
\caption{Pierwszy obraz wejściowy dla rozpoznawania narożników}
\label{fig:valCorner01}
\end{subfigure}
~
\begin{subfigure}[t]{0.3\textwidth}
\includegraphics[width=\textwidth]{TestyPoprawnosci/in_corner_result_01_07.png}
\caption{Pierwszy obraz wyjściowy dla rozpoznawania narożników}
\label{fig:valCornerResult01}
\end{subfigure}
\end{center}

\begin{subfigure}[t]{0.3\textwidth}
	\centering
	\setlength\fboxsep{0pt}
	\setlength\fboxrule{0.5pt}
	\fbox{\includegraphics[width=\textwidth]{TestyPoprawnosci/diff_corner_CL-CVCPU_01_07.png}}
	\caption{Porównanie implementacji CL i CVCPU dla rozpoznawania narożników}
	\label{fig:valCorner1CLCVCPU}
\end{subfigure}
~
\begin{subfigure}[t]{0.3\textwidth}
	\centering
	\setlength\fboxsep{0pt}
	\setlength\fboxrule{0.5pt}
	\fbox{\includegraphics[width=\textwidth]{TestyPoprawnosci/diff_corner_CL-CVGPU_01_07.png}}
	\caption{Porównanie implementacji CL i CVGPU dla rozpoznawania narożników}
	\label{fig:valCorner1CLCVGPU}
\end{subfigure}
~
\begin{subfigure}[t]{0.3\textwidth}
	\centering
	\setlength\fboxsep{0pt}
	\setlength\fboxrule{0.5pt}
	\fbox{\includegraphics[width=\textwidth]{TestyPoprawnosci/diff_corner_CVCPU-CVCPU_01_07.png}}
	\caption{Porównanie implementacji CVCPU i CVGPU dla rozpoznawania narożników}
	\label{fig:valCorner1CVCPUCVGPU}                 
\end{subfigure}
\caption{Obrazy różnicowe rozpoznawania narożników dla pierwszego pliku wejściowego}

\label{fig:valCorner1}
\end{figure}

\begin{figure}[h]

\begin{center}
\begin{subfigure}[t]{0.3\textwidth}
\includegraphics[width=\textwidth]{TestyPoprawnosci/in_corner_02.png}
\caption{Drugi obraz wejściowy dla rozpoznawania narożników}
\label{fig:valCorner02}
\end{subfigure}
~
\begin{subfigure}[t]{0.3\textwidth}
\includegraphics[width=\textwidth]{TestyPoprawnosci/in_corner_result_02_06.png}
\caption{Drugi obraz wyjściowy dla rozpoznawania narożników}
\label{fig:valCornerResult02}
\end{subfigure}
\end{center}

\begin{center}
\begin{subfigure}[t]{0.3\textwidth}
	\centering
	\setlength\fboxsep{0pt}
	\setlength\fboxrule{0.5pt}
	\fbox{\includegraphics[width=\textwidth]{TestyPoprawnosci/diff_corner_CL-CVGPU_02_06.png}}
	\caption{Porównanie implementacji CL i CVGPU dla rozpoznawania narożników}
	\label{fig:valCorner2CLCVGPU}
\end{subfigure}
~
\begin{subfigure}[t]{0.3\textwidth}
	\centering
	\setlength\fboxsep{0pt}
	\setlength\fboxrule{0.5pt}
	\fbox{\includegraphics[width=\textwidth]{TestyPoprawnosci/diff_corner_CL-CVGPU_02_06.png}}
	\caption{Porównanie implementacji CL i CVGPU dla rozpoznawania narożników}
	\label{fig:valCorner2CLCVGPU}
\end{subfigure}
\end{center}

\begin{subfigure}[t]{0.3\textwidth}
	\centering
	\setlength\fboxsep{0pt}
	\setlength\fboxrule{0.5pt}
	\fbox{\includegraphics[width=\textwidth]{TestyPoprawnosci/diff_corner_CL-CVCPU_02_06.png}}
	\caption{Porównanie implementacji CL i CVCPU dla rozpoznawania narożników}
	\label{fig:valCorner2CLCVCPU}
\end{subfigure}
~
\begin{subfigure}[t]{0.3\textwidth}
	\centering
	\setlength\fboxsep{0pt}
	\setlength\fboxrule{0.5pt}
	\fbox{\includegraphics[width=\textwidth]{TestyPoprawnosci/diff_corner_CL-CVGPU_02_06.png}}
	\caption{Porównanie implementacji CL i CVGPU dla rozpoznawania narożników}
	\label{fig:valCorner2CLCVGPU}
\end{subfigure}
~
\begin{subfigure}[t]{0.3\textwidth}
	\centering
	\setlength\fboxsep{0pt}
	\setlength\fboxrule{0.5pt}
	\fbox{\includegraphics[width=\textwidth]{TestyPoprawnosci/diff_corner_CVCPU-CVCPU_02_06.png}}
	\caption{Porównanie implementacji CVCPU i CVGPU dla rozpoznawania narożników}
	\label{fig:valCorner2CVCPUCVGPU}                 
\end{subfigure}
\caption{Obrazy różnicowe rozpoznawania narożników dla drugiego pliku wejściowego}

\label{fig:valCorner2}
\end{figure}

Na obrazach różnicowych można zauważyć pojedyncze różnice. Jest to też efektem tego, że liczba wykrywanych narożników na przedstawionych obrazach w~wybranej skali nie jest duża. Mniejsza liczba wykrytych obiektów powoduje, że liczba błędów jest mniejsza.

Na rys. \ref{fig:valCorner2} nie widać żadnych różnic.

\subsubsection{Granie}
\label{subsubsec:granieRysunki}

Obrazy wejściowe przedstawione są na rys. \ref{fig:valRidge01} i~\ref{fig:valRidge04}.

Obrazy różnicowe są przedstawione na rys. \ref{fig:valRidge1} i~\ref{fig:valRidge4}. 

\begin{figure}[h]

\begin{center}
\begin{subfigure}[t]{0.3\textwidth}
\includegraphics[width=\textwidth]{TestyPoprawnosci/in_ridge_01.png}
\caption{Pierwszy obraz wejściowy dla rozpoznawania grani}
\label{fig:valRidge01}
\end{subfigure}
~
\begin{subfigure}[t]{0.3\textwidth}
\includegraphics[width=\textwidth]{TestyPoprawnosci/in_ridge_result_01_01.png}
\caption{Pierwszy obraz wyjściowy dla rozpoznawania grani}
\label{fig:valRidgeResult01}
\end{subfigure}
\end{center}

\begin{subfigure}[t]{0.3\textwidth}
	\centering
	\setlength\fboxsep{0pt}
	\setlength\fboxrule{0.5pt}
	\fbox{\includegraphics[width=\textwidth]{TestyPoprawnosci/diff_ridge_CL-CVCPU_01_01.png}}
	\caption{Porównanie implementacji CL i CVCPU dla rozpoznawania grani}
	\label{fig:valRidge1CLCVCPU}
\end{subfigure}
~
\begin{subfigure}[t]{0.3\textwidth}
	\centering
	\setlength\fboxsep{0pt}
	\setlength\fboxrule{0.5pt}
	\fbox{\includegraphics[width=\textwidth]{TestyPoprawnosci/diff_ridge_CL-CVGPU_01_01.png}}
	\caption{Porównanie implementacji CL i CVGPU dla rozpoznawania grani}
	\label{fig:valRidge1CLCVGPU}
\end{subfigure}
~
\begin{subfigure}[t]{0.3\textwidth}
	\centering
	\setlength\fboxsep{0pt}
	\setlength\fboxrule{0.5pt}
	\fbox{\includegraphics[width=\textwidth]{TestyPoprawnosci/diff_ridge_CVCPU-CVCPU_01_01.png}}
	\caption{Porównanie implementacji CVCPU i CVGPU dla rozpoznawania grani}
	\label{fig:valRidge1CVCPUCVGPU}                 
\end{subfigure}
\caption{Obrazy różnicowe rozpoznawania grani dla pierwszego pliku wejściowego}

\label{fig:valRidge1}
\end{figure}

\begin{figure}[h]

\begin{center}
\begin{subfigure}[t]{0.3\textwidth}
\includegraphics[width=\textwidth]{TestyPoprawnosci/in_ridge_04.png}
\caption{Drugi obraz wejściowy dla rozpoznawania grani}
\label{fig:valRidge04}
\end{subfigure}
~
\begin{subfigure}[t]{0.3\textwidth}
\includegraphics[width=\textwidth]{TestyPoprawnosci/in_ridge_result_04_04.png}
\caption{Drugi obraz wyjściowy dla rozpoznawania grani}
\label{fig:valRidgeResult04}
\end{subfigure}
\end{center}

\begin{subfigure}[t]{0.3\textwidth}
	\centering
	\setlength\fboxsep{0pt}
	\setlength\fboxrule{0.5pt}
	\fbox{\includegraphics[width=\textwidth]{TestyPoprawnosci/diff_ridge_CL-CVCPU_04_04.png}}
	\caption{Porównanie implementacji CL i CVCPU dla rozpoznawania grani}
	\label{fig:valRidge4CLCVCPU}
\end{subfigure}
~
\begin{subfigure}[t]{0.3\textwidth}
	\centering
	\setlength\fboxsep{0pt}
	\setlength\fboxrule{0.5pt}
	\fbox{\includegraphics[width=\textwidth]{TestyPoprawnosci/diff_ridge_CL-CVGPU_04_04.png}}
	\caption{Porównanie implementacji CL i CVGPU dla rozpoznawania grani}
	\label{fig:valRidge4CLCVGPU}
\end{subfigure}
~
\begin{subfigure}[t]{0.3\textwidth}
	\centering
	\setlength\fboxsep{0pt}
	\setlength\fboxrule{0.5pt}
	\fbox{\includegraphics[width=\textwidth]{TestyPoprawnosci/diff_ridge_CVCPU-CVCPU_04_04.png}}
	\caption{Porównanie implementacji CVCPU i CVGPU dla rozpoznawania grani}
	\label{fig:valRidge4CVCPUCVGPU}                 
\end{subfigure}
\caption{Obrazy różnicowe rozpoznawania grani dla drugiego pliku wejściowego}

\label{fig:valRidge4}
\end{figure}

Na obrazach różnicowych można zauważyć, że na krawędziach powstało bardzo dużo różnic. Są one pochodną rozbieżności powstałych na etapie tworzenia reprezentacji skali. W~środku obrazu widać tylko pojedyncze różnice. 
