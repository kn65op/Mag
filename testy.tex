\chapter{Testy}
\label{cha:testy}

W niniejszym rozdziale przedstawiono proces testowania oraz ich wyniki. Obszarem zainteresowań jest poprawność zaimplementowanego algorytmu oraz porównanie szybkości działania z~innymi implementacjami.

Do porównań opracowano dwie dodatkowe implementacje algorytmu zrealizowane z użyciem biblioteki OpenCV: jedna przeprowadza obliczenia na procesorze (CPU), druga przeprowadza obliczenia na karcie graficznej (GPU).

\section{Testowanie poprawności}
\label{sec:testPoprawnosc}

Celem tej części testowania było sprawdzenie czy zrealizowane implementacje działają w~ten sam sposób. Proces testowania był następujący:
\begin{enumarate}
\item Przetworzenie tego samego obrazu za pomocą wszystkich implementacji. Wybrany obraz był reprezentowany w~kilku skalach. Przeprowadzono obliczenia dla każdej rozpoznawanej cechy.
\item Zestawienie obrazów wynikowych z~podziałem na cechy oraz skalę. Dla każdej cechy i~skali otrzymano trzy obrazy, po jednym dla każdej implementacji.
\item Porównanie wyników. Idealnym wynikiem byłoby otrzymanie trzech identycznych obrazów wynikowych. W praktyce jest to mało prawdopodobne, z~uwagi na różnice podczas zaokrąglenia obliczeń w~implementacjach.
\end{enumerate}



% porównanie poprawności, obrazy różnicowe itp.

\section{Testowanie szybkości działania}
\label{sec:testSzybkosc1}

Testowanie szybkości działania oparto na porównaniu szybkości działania wszystkich implementacji algorytmu Scale Space. 

%opis maszyn, na których były wykonane testy

\section{Podsumowanie testów}
\label{sec:testPodsumowanie}

%podsumowanie testów
