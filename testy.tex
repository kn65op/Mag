\chapter{Testy}
\label{cha:testy}

W niniejszym rozdziale przedstawiono proces testowania zaimplementowanych w~sposób równoległy algorytmów oraz uzyskane wyniki. Obszarem zainteresowań jest poprawność zaimplementowanego algorytmu oraz porównanie szybkości działania z~innymi implementacjami: równoległymi i~sekwencyjnymi.

Do porównań opracowano dwie dodatkowe implementacje algorytmu zrealizowane z użyciem biblioteki OpenCV: jedna przeprowadza obliczenia na procesorze (CPU), druga przeprowadza obliczenia na karcie graficznej (GPU) z~wykorzystaniem pakietu CUDA.

\section{Środowisko testowe}
\label{sec:srodowiskoTesty}

Do testów użyto zestawów urządzeń.

\begin{enumerate}
\item Zestaw pierwszy:
\begin{itemize}
\item karta graficzna: GeForce GT 555M \cite{GT555M} o następujących parametrach:
\begin{itemize}
\item liczba rdzeni: 144,
\item zegar układu graficznego: 675 MHz,
%\item zegar procesora: 1350 MHz,
\item częstotliwość szyny danych pamięci: 1800 MHz,
\item interfejs pamięci: 128-bitowy,
\item szerokość pasma pamięci: 28,80GB/s,
\item dostępna pamięć: 4095 MB.
\end{itemize}
\item procesor: Intel Core i7-2670QM pracujący z~częstotliwością 2200 MHz.
\end{itemize}

\item GeForce GTX 670 \cite{GTX670} o następujących parametrach:
\begin{itemize}
\item liczba rdzeni: 1344,
\item zegar układu graficznego: 980 MHz,
%\item zegar procesora: 1350 MHz,
\item częstotliwość szyny danych pamięci: 6008 MHz,
\item interfejs pamięci: 256-bitowy,
\item szerokość pasma pamięci: 192.26 GB/s,
\item dostępna pamięć: 4096 MB.
\end{itemize}
\item procesor: Intel Core i5-3570 pracujący z~częstotliwością 3400 MHz.
\end{enumerate}

W~niniejszym rodziale przyjęto następujące oznaczenia:
\begin{itemize}
\item CL, OpenCL - implementacja wykonana z~użyciem OpenCL,
\item CVCPU, OpenCVCPU - implementacja wykonana z~użyciem OpenCV wykonywania na procesorze CPU w~sposób sekwencyjny,
\item CVGPU, OpenCVGPU - implementacja wykonana z~użyciem OpenCV wykonywania na procesorze graficznym GPU w~sposób równoległy z~wykorzystaniem ,
\item CL-CVCPU - oznacza porównanie pomiędzy wynikiem implementacji wykonanej w~OpenCL a implementacją wykonaną za pomocą OpenCV na procesorze,
\item CL-CVGPU - oznacza porównanie pomiędzy wynikiem implementacji wykonanej w~OpenCL a implementacją wykonaną za pomocą OpenCV na karcie graficznej,
\item CVCPU-CVGPU - oznacza porównanie pomiędzy wynikiem implementacji wykonanej za pomocą OpenCV na procesorze a implementacją wykonaną za pomocą OpenCV na karcie graficznej.
\end{itemize}
