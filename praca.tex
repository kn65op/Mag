\documentclass[pdflatex,11pt]{aghdpl}
% \documentclass{aghdpl}               % przy kompilacji programem latex
% \documentclass[pdflatex,en]{aghdpl}  % praca w języku angielskim
\usepackage[polish]{babel}
\usepackage[utf8]{inputenc}

% dodatkowe pakiety
\usepackage{enumerate}
\usepackage{listings}
\usepackage{hyperref}
\usepackage{amsmath}
%\usepackage{graphicx}
\lstloadlanguages{TeX}

%---------------------------------------------------------------------------

\author{Tomasz Drzewiecki}
\shortauthor{T. Drzewiecki}

\titlePL{Równoległa realizacja algorytmu Scale Space przy pomocy procesorów graficznych}
\titleEN{title en}

\shorttitlePL{Równoległa realizacja algorytmu Scale Space} % skrócona wersja tytułu jeśli jest bardzo długi
\shorttitleEN{short title en}

\thesistypePL{Praca dyplomowa magisterska}
\thesistypeEN{Master's Thesis}

\supervisorPL{dr inż. Mirosław Jabłoński}
\supervisorEN{Mirosław Jabłoński, Ph.D}

\date{2013}

\departmentPL{Katedra Automatyki i~Inżynierii Biomedycznej}
\departmentEN{Department of Automatics and Biomedical Engineering}

\facultyPL{Wydział Elektrotechniki, Automatyki, Informatyki i~Inżynierii Biomedycznej}
\facultyEN{Faculty of Electrical Engineering, Automatics, Computer Science and Biomedical Engineering}

\acknowledgements{Serdecznie dziękuję }



\setlength{\cftsecnumwidth}{10mm}

%---------------------------------------------------------------------------

\begin{document}

\titlepages

\tableofcontents
\clearpage

\chapter{Wprowadzenie}
\label{cha:wprowadzenie}

W~tym rozdziale przedstawiono cel pracy oraz podstawę teoretyczną realizacji. Opisano szczegółowo implementowany algorytm Scale Space: podstawy matematyczne i~sposób działania. Przybliżono również standard OpenCL, który został wykorzystany do realizacji algorytmu.


\section{Cel pracy}
\label{sec:cel}
Celem niniejszej pracy była realizacja algortmu Scale Space, który służy do reprezentacji syngłów (w~szczególności obrazów) w~wielu skalach w celu późniejszej ich analizy.
Wiele algorytmów operujących na obrazach jest złożona obliczeniowo i~ich realizacja zajmuje dużo czasu, zwłaszcza dla dużych porcji danych. Dlatego zdecydowano się na realizację tego algorytmu z użyciem akceleracji w~postaci zrównoleglenia obliczeń z~użyciem karty lub kart graficznych (GPU).

Implementacja algorytmu została zrealizowana w~OpenCL \cite{OpenCL}, wieloplatformowym standardzie do zrównoleglania obliczeń. Wybrano ten standard, ponieważ jest dość rozpowszechnony oraz wspierany przez wszystkich największych producentów sprzętu komputerowego (m. in. Intel\textsuperscript{\textregistered}, NVIDIA\textsuperscript{\texttrademark}, AMD).

\section{Algorytm Scale Space}
\label{sec:algorytm}
Scale Space jest algorytmem służącym do przkształcenia sygnałów do reprezentacji skali. Reprezentacja skali jest to rodzina sygnałów reprezentująca oryginalny syngał w~różnych stopniach skali.
Ta reprezentacja pozwala na analizę oryginalnego sygnału w~różnych stopniach szczegółowości.
Pomysł na stworzenie takiego algorytmu wynika z~wieloskalowej struktury świata. Taka struktura powoduje, że obiekty mogą być różnie postrzegane w~zależności od skali obserwacji \cite{Enc09}.
W~tej pracy przetwarzanymi sygnałami będą obrazy.

Tworząc system automatycznego rozpoznawania obrazów nieznanych scen nie ma możliwości określenie z~góry w~jakiej skali będą przedstawiane obiekty, które będą interesujące dla użytkownika systemu. Dlatego w~takim systemie można użyć algorytmu Scale Space wraz z~automatycznym rozpoznawaniem najbajdziej interesujące skali. Po stworzeniu reprezentacji skali można zrealizować rozpoznawanie brzegów, obiektów. Te opracje mogą być przeprowadzane tylko w~jednej skali lub dla większej ilości skal, w~szcególności we wszystkich skalach.


\subsection{Filtracja Gaussa}
\label{subsec:filtracjaGaussa}
Filtry Gaussa są jednymi z podstawowych operacji wykorzystywanych w przetwarzaniu obrazów cyfrowych. Są to filtry dolnoprzepustowe, rozmywające obraz, po zastosowaniu których ze sceny można odczytać ogólne kształty przedstawionych obiektów. Po tej operacji szczegóły zostają usunięte, bądź zostaje znacznie zmniejszony ich wpływ na całość.

Kolejne filtry Gaussa w przestrzeni ciągłej dwuwymiarowej są określone wzorem \ref{eq:gaussian}:
\begin{equation}
\label{eq:gaussian}
g(x,y,\sigma)=\frac{1}{2 \cdot \pi \cdot ^ {2} }\cdot e^{(-\frac{x^{2} + y^{2}}{2 \cdot \sigma ^{2}})}
\end{equation}
gdzie:\\
$ x,y $ - położenie piksela na obrazie, \\
$ \sigma $ - wariancja.
\newline
Wariancja w~powyższym wzorze określa skalę, w~jakiej obraz wyjściowy jest przedstawiony. 

Ponieważ podczas obilczeń z~użyciem komputera nie jest możliwe używanie przestrzeni ciągłej, dlatego konieczene jest wprowadzenie reprezentacji dyskretnej. 
Dla dwuwymiarowych sygnałów dyskretnych, zostały użyte filtry Gaussa określone wzorem \ref{eq:gaussian_discrete}:
\begin{equation}
\label{eq:gaussian_discrete}
g(x,y,N,\sigma) = \alpha \cdot e^{-((x+y)-N/2)^2/(2 \cdot \sigma)^2}
\end{equation}
gdzie: \newline 
$ x, y$ - współrzędne obrazu, \newline
$ \alpha $ - współczynnik skalujący w celu normalizacji ($ \sum_x \sum_y g(x,y,N,\sigma) = 1 $), \newline
$ N $ - rozmiar filtru, \newline
$ \sigma $ - wariancja obliczona zgodnie ze wzorem $ \sigma = 0.3 \cdot (N \cdot 0,5 - 1) + 0,8$. \newline
W~tym przypadku rozmiar filtru określa skalę.

\subsection{Reprezentacja skali}
\label{subsec:reprezentacjaskali}
Reprezentacja skali dla sygnałów ciągłych dwuwymiarowych (np. obrazów) powstaje w sposób przedstawiony we wzorze \ref{eq:scalespace}:

\begin{equation}
\label{eq:scalespace}
\begin{split}
& L(\cdot,\cdot,0) = f(\cdot,\cdot) \\
& L(\cdot,\cdot,\sigma) = g(\cdot,\cdot,\sigma)\cdot f(\cdot,\cdot)
\end{split}
\end{equation}
gdzie:\\
$ f $ - sygnał oryginalny, \\
$ g $ - filtr Gaussa, \\
$ \sigma $ - wariancja (parametr skali).

Oznacza to, że w celu uzyskania reprezentacji skali obraz poddawany jest filtracji Gaussa, z~różnymi rosnącymi wartościami dyskretnymi $ \sigma $. Obraz bez zastosowania skali to obraz oryginalny.

W~ten sposób można otrzymać wiele wynikowych obrazów, w~których każdy przedstawia początkową w scenę w różnej skali. Dzięki temu można analizować obraz w~różnym stopniu szczegółowości, co jest jedną z głównych zalet algorymtu Scale Space.

Ponieważ filtry Gaussa spełniają aksjomaty Scale Space, to ich użycie gwarantuje nam, że libcza ekstremów lokalnych nie zwiększy się. Również wartości ekstremów nie zostaną zwiększone. Oznacza to, że wartość pikseli w~maksimach lokalnych nie będzie rosła, a~wartość pikseli w minimach lokalnych nie będzie malała. Także w~przestrzeni dyskretnej aksjomany Scale Space są spełnione \cite{SSFDS}.

%\subsection{Rozpoznawnie}
%\label{subsec:rozpoznawanie}

\subsection{Złożność obliczeniowa}
\label{subsec:zlozonosc_obliczeniowa}

\section{OpenCL}
\label{sec:OpenCL}

Ponieważ realizacja algorytmu Scale Space wymaga przeprowadzenia bardzo dużej liczby obliczeń, zdecydowano, że algorytm będzie zrealizowany z użyciem karty graficznej. Do wykorzystania kart graficznych w~dowolnych obliczeniach można zastosować otwarty standard OpenCL lub technologie stworzone przez producentów procesorów graficznych dedykowanych dla urządzeń danego producenta.

OpenCL jest to otwarty, wieloplatformowy standard pozwalający na realizację algorytmów w~sposób równoległy. Umożliwa realizację jednego algorytmu na wielu różnego typu urządzeniach: procesorach wielordzeniowych, kartach graficznych oraz innych, wszystkich które go wspierają[trzeba podać jakie inne obecnie są wspierane]. Standard jest wspierany przez wszystich największych producentów sprzętu elektronicznego, więc jego zastosowanie pozwala stworzyć oprogramowanie, które może być wykorzystane w~praktycznie dowolnej konfiguracji urządzeń przetwarzających dane.

Powyższe zalety spowodowały, że do implentacji został wybrany standard OpenCL, zamiast użycia technologii dedykowanych do urządzeń produkowanych przez konkretne firmy. Przykładem takich rozwiązań są: CUDA lub ATI Stream. Rozwiązania dostarczane przez producentów urządzeń są szybsze niż implementacja standardu OpenCL [!!!!!!Czy zawsze? Trzeba podać jakieś referencje lub wyniki własnych testów.!!!!!!!]. Różnica nie jest jednak duża, więc jest to do zaakceptowania, zwłaszcza w~porówaniu do korzyści osianiętych z wykorzystania OpenCL.



%\include{realizacja}
%\include{czesc_trzecia}
%\include{technologie}
%\include{wnioski}
%\chapter{Wnioski i podsumowanie}
\label{cha:podsumowanie}

Biorąc pod uwagę uzyskane wyniki testów udało się zrealizować cel pracy, jakim było stworzenie implementacji algorytmu Scale Space z~zastosowaniem biblioteki OpenCL. Testy poprawności pokazały, że zrealizowana implementacja jest zgodna (z~dokładnością do niewielkich różnic spowodowanych błędami w~obliczeniach numerycznych). Testy wydajnościowe wykazały, że implementacja wykorzystująca OpenCL jest szybsza niż realizacja algorytmu na procesorze ogólnego przeznaczenia CPU. Porównano również szybkość wykonywania obliczeń zrealizowanej biblioteki do szybkości działania rozwiązań zrealizowanych przy użyciu biblioteki OpenCV. Stworzona implementacja wykonuje obliczenia szybciej niż rozwiązanie opearte o~bibliotekę OpenCV wykorzystujące kartę graficzną. Uzyskane przyśpieszenie jest zależne od wielkości skali i~wynosi, pomijając najmniejszą skalę, od 2 do 12,5 przy błędzie mniejszym niż $ 0,8\% $. Dla obrazów o~rozdzielczości 512x512 pikseli oraz dla 4 skal osiągnięto działanie programu w~czasie rzeczywistym.

Można zauważyć, że poszczególne detektory różnią się od siebie. Wydaje się, że najlepiej działającym jest detektor narożników. W~przypadku kontynuacji prac warto byłoby przeanalizować i~udoskonalić działanie detektora grani.

Dalszymi krokami usprawniającymi przedstawione rozwiązanie może być umożliwienie wykorzystanie większej liczby kart graficznych, jeśli są dostępne w~danym systemie. Użycie większej ilości kart graficznym mogłoby umożliwić przetwarzanie obrazu wejściowego w~czasie rzeczywistym dla obrazów o~większej rozdzielczości.

Drugą możliwym kierunkiem rozwoju zrealizowanego systemu może być wykorzystanie otrzymanych informacji. Obecnie system realizuje detekcję krawędzi, narożników, plam oraz grani. Otrzymane informacje mogą być wykorzystane w~kolejnych krokach bardziej złożonego algorytmu, pozwalającego na analizę i~rozpoznawanie obiektów. Dzięki wstępnemu przetworzeniu, które jest realizowane w~tej pracy można do analizy obrazu wykorzystać mniejszą ilość danych. Tymi danymi będę miejsca (piksele), w~których znaleziono poszczególne strukury.


%\include{rozdzial2}


\bibliographystyle{plain}
\bibliography{praca}{}

%\appendix
%\include{zawartoscCD}
%\include{opiskodow}

\end{document}
