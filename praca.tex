\documentclass[pdflatex,11pt]{aghdpl}
% \documentclass{aghdpl}               % przy kompilacji programem latex
% \documentclass[pdflatex,en]{aghdpl}  % praca w języku angielskim
\usepackage[polish]{babel}
\usepackage[utf8]{inputenc}
\usepackage{morefloats} %może do usunięcia

% dodatkowe pakiety
\usepackage{enumerate}
\usepackage{listings}
\usepackage{hyperref}
\usepackage{amsmath}
\usepackage{listings}
\usepackage{graphicx}
\usepackage{subcaption}
\lstloadlanguages{TeX}

%---------------------------------------------------------------------------

\author{Tomasz Drzewiecki}
\shortauthor{T. Drzewiecki}

\titlePL{Równoległa realizacja algorytmu Scale Space przy pomocy procesorów graficznych}
\titleEN{Parallel implementation of ScaleSpace Algorithm on GPU processors}

\shorttitlePL{Równoległa realizacja algorytmu Scale Space} % skrócona wersja tytułu jeśli jest bardzo długi
\shorttitleEN{Parallel implementation of ScaleSpace Algorithm}

\thesistypePL{Praca dyplomowa magisterska}
\thesistypeEN{Master's Thesis}

\supervisorPL{dr inż. Mirosław Jabłoński}
\supervisorEN{Mirosław Jabłoński, Ph.D}

\date{2013}

\departmentPL{Katedra Automatyki i~Inżynierii Biomedycznej}
\departmentEN{Department of Automatics and Biomedical Engineering}

\facultyPL{Wydział Elektrotechniki, Automatyki, Informatyki i~Inżynierii Biomedycznej}
\facultyEN{Faculty of Electrical Engineering, Automatics, Computer Science and Biomedical Engineering}

\acknowledgements{Serdecznie dziękuję }



\setlength{\cftsecnumwidth}{10mm}

%---------------------------------------------------------------------------

\begin{document}

\titlepages

\tableofcontents
\clearpage

\chapter{Wprowadzenie}
\label{cha:wprowadzenie}

W~tym rozdziale przedstawiono cel pracy oraz podstawę teoretyczną realizacji. Opisano szczegółowo implementowany algorytm Scale Space: podstawy matematyczne i~sposób działania. Przybliżono również standard OpenCL, który został wykorzystany do realizacji algorytmu.


\section{Cel pracy}
\label{sec:cel}
Celem niniejszej pracy była realizacja algortmu Scale Space, który służy do reprezentacji syngłów (w~szczególności obrazów) w~wielu skalach w celu późniejszej ich analizy.
Wiele algorytmów operujących na obrazach jest złożona obliczeniowo i~ich realizacja zajmuje dużo czasu, zwłaszcza dla dużych porcji danych. Dlatego zdecydowano się na realizację tego algorytmu z użyciem akceleracji w~postaci zrównoleglenia obliczeń z~użyciem karty lub kart graficznych (GPU).

Implementacja algorytmu została zrealizowana w~OpenCL \cite{OpenCL}, wieloplatformowym standardzie do zrównoleglania obliczeń. Wybrano ten standard, ponieważ jest dość rozpowszechnony oraz wspierany przez wszystkich największych producentów sprzętu komputerowego (m. in. Intel\textsuperscript{\textregistered}, NVIDIA\textsuperscript{\texttrademark}, AMD).

\section{Algorytm Scale Space}
\label{sec:algorytm}
Scale Space jest algorytmem służącym do przkształcenia sygnałów do reprezentacji skali. Reprezentacja skali jest to rodzina sygnałów reprezentująca oryginalny syngał w~różnych stopniach skali.
Ta reprezentacja pozwala na analizę oryginalnego sygnału w~różnych stopniach szczegółowości.
Pomysł na stworzenie takiego algorytmu wynika z~wieloskalowej struktury świata. Taka struktura powoduje, że obiekty mogą być różnie postrzegane w~zależności od skali obserwacji \cite{Enc09}.
W~tej pracy przetwarzanymi sygnałami będą obrazy.

Tworząc system automatycznego rozpoznawania obrazów nieznanych scen nie ma możliwości określenie z~góry w~jakiej skali będą przedstawiane obiekty, które będą interesujące dla użytkownika systemu. Dlatego w~takim systemie można użyć algorytmu Scale Space wraz z~automatycznym rozpoznawaniem najbajdziej interesujące skali. Po stworzeniu reprezentacji skali można zrealizować rozpoznawanie brzegów, obiektów. Te opracje mogą być przeprowadzane tylko w~jednej skali lub dla większej ilości skal, w~szcególności we wszystkich skalach.


\subsection{Filtracja Gaussa}
\label{subsec:filtracjaGaussa}
Filtry Gaussa są jednymi z podstawowych operacji wykorzystywanych w przetwarzaniu obrazów cyfrowych. Są to filtry dolnoprzepustowe, rozmywające obraz, po zastosowaniu których ze sceny można odczytać ogólne kształty przedstawionych obiektów. Po tej operacji szczegóły zostają usunięte, bądź zostaje znacznie zmniejszony ich wpływ na całość.

Kolejne filtry Gaussa w przestrzeni ciągłej dwuwymiarowej są określone wzorem \ref{eq:gaussian}:
\begin{equation}
\label{eq:gaussian}
g(x,y,\sigma)=\frac{1}{2 \cdot \pi \cdot ^ {2} }\cdot e^{(-\frac{x^{2} + y^{2}}{2 \cdot \sigma ^{2}})}
\end{equation}
gdzie:\\
$ x,y $ - położenie piksela na obrazie, \\
$ \sigma $ - wariancja.
\newline
Wariancja w~powyższym wzorze określa skalę, w~jakiej obraz wyjściowy jest przedstawiony. 

Ponieważ podczas obilczeń z~użyciem komputera nie jest możliwe używanie przestrzeni ciągłej, dlatego konieczene jest wprowadzenie reprezentacji dyskretnej. 
Dla dwuwymiarowych sygnałów dyskretnych, zostały użyte filtry Gaussa określone wzorem \ref{eq:gaussian_discrete}:
\begin{equation}
\label{eq:gaussian_discrete}
g(x,y,N,\sigma) = \alpha \cdot e^{-((x+y)-N/2)^2/(2 \cdot \sigma)^2}
\end{equation}
gdzie: \newline 
$ x, y$ - współrzędne obrazu, \newline
$ \alpha $ - współczynnik skalujący w celu normalizacji ($ \sum_x \sum_y g(x,y,N,\sigma) = 1 $), \newline
$ N $ - rozmiar filtru, \newline
$ \sigma $ - wariancja obliczona zgodnie ze wzorem $ \sigma = 0.3 \cdot (N \cdot 0,5 - 1) + 0,8$. \newline
W~tym przypadku rozmiar filtru określa skalę. Za pomocą podanego wzoru obliczane są współczynniki maski, z~użyciem której będzie przeprowadzona konwolucja z~obrazem. W ten sposób otrzymany będzie obraz w danej skali.

\subsection{Reprezentacja skali}
\label{subsec:reprezentacjaskali}
Reprezentacja skali dla sygnałów ciągłych dwuwymiarowych (np. obrazów) powstaje w sposób przedstawiony we wzorze \ref{eq:scalespace}:

\begin{equation}
\label{eq:scalespace}
\begin{split}
& L(\cdot,\cdot,0) = f(\cdot,\cdot) \\
& L(\cdot,\cdot,\sigma) = g(\cdot,\cdot,\sigma)\cdot f(\cdot,\cdot)
\end{split}
\end{equation}
gdzie:\\
$ f $ - sygnał oryginalny, \\
$ g $ - filtr Gaussa, \\
$ \sigma $ - wariancja (parametr skali).

Oznacza to, że w celu uzyskania reprezentacji skali obraz poddawany jest filtracji Gaussa, dla różnych, rosnących wartości $ \sigma $. Obraz bez zastosowania skali to obraz oryginalny.

W~ten sposób można otrzymać wiele wynikowych obrazów, w~których każdy przedstawia początkową w scenę w różnej skali. Dzięki temu można analizować obraz w~różnym stopniu szczegółowości, co jest jedną z głównych zalet algorymtu Scale Space.

Ponieważ filtry Gaussa spełniają aksjomaty Scale Space, to ich użycie gwarantuje nam, że libcza ekstremów lokalnych nie zwiększy się. Również wartości ekstremów nie zostaną zwiększone. Oznacza to, że wartość pikseli w~maksimach lokalnych nie będzie rosła, a~wartość pikseli w minimach lokalnych nie będzie malała. Także w~przestrzeni dyskretnej aksjomany Scale Space są spełnione \cite{SSFDS}.

%\subsection{Rozpoznawnie}
%\label{subsec:rozpoznawanie}

\subsection{Złożność obliczeniowa}
\label{subsec:zlozonosc_obliczeniowa}

Do wykonania algorytmu Scale Space jest konieczne przeprowadzenie dużej ilości obliczeń. Poniżej przedstawione jest oszacownie ich liczby. Przyjęto następujące oznaczenia:
\begin{itemize}
\item $ N $ - liczba pikseli obrazu,
\item $ I $ - rozmiar reprezentacji skal (liczba skal, w~których zostanie przedstawiony obraz wejściowy),
\item $ M_i $ - rozmiar maski, za pomocą której zostanie wyzaczony obraz w $ i-tej $ skali,
\item $ A $ - liczba działań konieczna do wyznaczenia jednego współczynnika maski.
\end{itemize}

Do wyznaczenia obrazu będącego reprezntacją w~danej skali stosowana będzie konwolucja obrazu wejściowego z~maską określoną za pomocą wzoru \ref{eq:gaussian_discrete}. Im większa skala, tym większy będzie rozmiar maski. Do wykonania tej operacji konieczne będzie przeprowadzenia $ M_i $ działań mnożenia oraz $ M_i - 1 $ działań dodawania dla każdego z~$N$ pikseli obrazu. Łącznie koniecznych jest wykonanie $ N \cdot (2M_i - 1) $ operacji.

Do wyznaczenia całej reprezentacji konieczne będzie wykonanie $ I \cdot N \cdot (2M_i - 1) $ operacji tylko w celu obliczenia reprezentacji. Przedtem będzie konieczne wyznacznie współczynników maski. Jest to operacja jednorazowa dla każdej skali. Liczba działań do wykonania jest zależna od rozmiaru maski i~wynosi $ A \cdot M_i $.

Podsumowując do wyznaczenia całej reprezentacji skali konieczne wykonanie $ I \cdot (A \cdot M_i + N \cdot (2M_i - 1)) $ działań. Zakładając, że $ A << N $ to można stwierdzić, że wyznaczanie reprezentacji skali ma złożoność obliczeniową $ O(I \cdot N \cdot M_I) $.

\section{OpenCL}
\label{sec:OpenCL}

Ponieważ realizacja algorytmu Scale Space wymaga przeprowadzenia bardzo dużej liczby obliczeń, zdecydowano, że algorytm będzie zrealizowany z użyciem karty graficznej.

Do wykorzystania kart graficznych w~obliczeniach ogólnego przeznaczenia można zastosować otwarty standard OpenCL lub technologie stworzone przez producentów procesorów graficznych dedykowanych dla urządzeń danego producenta.

OpenCL jest to otwarty, wieloplatformowy standard pozwalający na realizację algorytmów w~sposób równoległy. Umożliwa realizację jednego algorytmu (z~dokładnością do wspieranej wersji standaru) na wielu różnego typu ~urządzeniach. Wśród nich znajdują się procesory oraz karty graficzne dla komputerów osobistych działających pod kontrolą jednego z~najpopularnieszych systemów operacyjnych (Windows, Linux, OS X), telefony komórkowe, procesory ARM (również wielordzeniowe). Prodenci wspierający standard to Intel,
QUALCOMM,
ARM,
AMD,
Apple,
Vivante Corporation,
STMicroelectronics International NV,
IBM Corporation,
Imagination Technologies,
Creative Labs.
Dzięki temu zastosowanie standardu pozwala stworzyć oprogramowanie, które może być wykorzystane w~praktycznie dowolnej konfiguracji urządzeń przetwarzających dane. Prace nad standardem rozpoczęły się w~2008 roku. Jest to technologia wciąż rozwijająca się oraz zapewaniająca wsteczną kompatybilność. 

Przykładem technologii stworzonych przez producentów procesorów graficznych dedykowanych dla urządzeń danego producenta są technologie CUDA firmy NVIDIA oraz ATI Stream. Technologie te są ograniczone tylko do produktów jednego producenta, co znacznie zmniejsza przenośność stworzonej aplikacji. Technologie te powstały w~2006 roku. Znacznie bardziej znaną obecnie jest technologia CUDA. Te technologie również są w~fazie intensywnego rozwoju oraz oferują wsteczną kompatybilnosć.

Porównując wydajność OpenCL z rozwiązaniami opracowanymi przez producentów należy zauważyć, że implementacje standardu są w~większości opracowywane przez producentów sprzętu wraz ze~sterownikami. W~związku z tym różnica w~efektywności działania nie istieje lub jest znikoma (w~zależności od użytych instrukcji). Dlatego ten aspekt nie był brany pod uwagę podczas wyboru technologii.

Biorąc pod uwagę powyższe cechy standardu OpenCL oraz technologii dedykowanych przez producentów sprzętu zdecydowano się na realizację algorytmu za pomocą OpenCL. Dzięki temu implementacja będzie mogła być wykorzystana w~różnych konfiguracjach sprzętowych.



\chapter{OpenCL}
\label{cha:opencl}

W~tym rodziale przedstawiono sposób implementacji algorytmów z~użyciem standardu OpenCL oraz stworzoną bibliotekę służącą jej ułatwieniu. Zdecydowano się na jej stworzenie z~uwagi na to, że w cześci programu, która kontroluje wykonanie kodu na karcie graficznej (kerneli) wiele czynności się powtarza przy każdym kolejnym algorytmie. Zatem zadaniem biblioteki jest ułatwienie implementacji kontrolera nie ograniczając możliwości oferowanych przez standard OpenCL.

\section{Szczegóły implementacji z użyciem OpenCL}  
\label{sec:szczegolyOpenCL}


\chapter{Obsługa kamery cyfrowej}
\label{cha:obslugakamery}

W niniejszym rozdziale przedstawiono szczegóły techniczne wykorzystanej kamery oraz implementację obsługi kamery.

\section{Specyfikacja techniczna kamery kolorowej}
\label{sec:specyfikacjaKamery}

Do testów została wykorzystano przemysłową kamerę kolorową wysokiej rozdzielczości: JAI BB-500GE. Jej maksymalna rozdzielczość to 2456 na 2058 pikseli, przy której można pobrać 15 klatek na sekundę. Możliwe jest pobranie pikseli o rozdzielczości bitowej 8, 10, 12 lub 16 bitów. Dane są pobierane za pomocą interfejsu Gigabit Ethernet. 

Obraz pobierany z kolorowej kamery jest w formacie Bayera \cite{Bayer}. Jest to sposób reprezentacji obrazu, który jest używany w~kamerach lub aparatach cyfrowych. Z~tego powodu konieczne było zastosowanie interpolacji obrazu barwnego. Do realizacji tego celu użyto implementacji stworzonej i~opisanej przez Autorów w~pracy \cite{BFIOCL}.

\section{Szczegóły obsługi kamery}
\label{sec:szczegolyObslugiKamery}

Obsługa kamery została zrealizowana z użyciem dostarczonej przez producenta biblioteki JAI SDK \cite{JAISDK}. Ponieważ jest ona napisana w~języku C, stworzono klasę odpowiedzialną za obsługę kamery. Klasa automatycznie zarządza zasobami, ustawieniami oraz dostępem do kamery, dzięki temu korzystanie z~niej jest łatwe.

Biblioteka JAI SDK ma duże możliwości. Poza połączeniem się z~kamerą i~pobraniem zarejestrowanego obrazu udostępnia wiele algorytmów. Między innymi zaimplementowana jest interpolacja obrazu barwnego i~redukcja zniekształceń pochodzących z~obiektywu.

Do celów tej pracy wykorzystano tylko podstawowa funkcjonalność kamery i~dostarczonej do niej biblioteki, do których należą: połączenie się z~kamerą oraz pobranie obrazu. Dlatego stworzona klasa korzysta tylko z~podstawowych możliwości kamery oraz dostarczonej do niej biblioteki.

\section{Implementacja obsługi kamery}
\label{sec:implementacjaKamery}

Zaimplementowana klasa realizuje podstawowe operacje z~poniższego zbioru. 
\begin{itemize}
\item Wyszukiwanie kamer - pierwszą zaimplementowaną funkcjonalnością jest wyszukiwanie kamer w~systemie, do których możliwy jest dostęp z~użyciem biblioteki JAI SDK. Jest to pierwsza czynność, którą należy wykonać, aby móc pobierać obrazy z~kamery. Po procesie wyszukiwania zwracana jest lista dostępnych kamer.
\item Otwarcie kamery - po wybraniu danej kamery należy ją otworzyć. Dopiero po otwarciu kamery można wykonywać na niej operacje. Na tym etapie można ustawić parametry kamery. Dostępnym parametrem do ustawienia jest ustalenie rozdzielczości bitowej piksela (8, 10, 12 lub 16 bitów).
\item Rozpoczęcie akwizycji obrazów - w~tym momencie następuje akwizycja obrazów. Obrazy są zapisywane w~kolejce o~określonym rozmiarze. W~przypadku przepełnienia się bufora kolejne rejestrowane obrazy będą odrzucane do momentu, aż jakiś obraz zostanie pobrany z kolejki.
\item Pobranie kolejnego obrazu - po rozpoczęciu akwizycji można w~dowolnym momencie pobrać ostatni obraz z~kolejki. Jeśli jest ona pusta, zostanie to zgłoszone użytkownikowi.
\item Zakończenie akwizycji obrazów - zatrzymuje zapisywanie obrazów z~kamery w~kolejce. Nie jest ona kasowana, więc już zarejestrowane obrazy są dostępne dla użytkownika. W~dowolnym momencie można rozpocząć rejestrację ponownie.
\item Zamknięcie kamery - zwalnia zasoby kamery i~umożliwia innym programom dostęp do niej. W~celu ponownej rejestracji obrazów należy otworzyć kamerę ponownie.
\end{itemize}
Operacje zakończenia akwizycji obrazów i~zamknięcie kamery są realizowane w~destruktorze, co pozwala na automatyczne zwalnianie zasobów kiedy nie są już potrzebne.

Pobieranie obrazu jest realizowane asynchronicznie. Uruchamiany jest wątek, który jest wybudzany w~momencie gdy kamera przesyła obraz do komputera. W~tym wątku obraz z~kamery jest zapisywany do kolejki obrazów. Użytkownik może pobierać obrazy z~kolejki niezależnie w~oddzielnym wątku. Jest to niemożliwe jedynie podczas dodawanie nowego obrazu do kolejki. Aby zapobiec błędom związanym z~przetwarzaniem wielowątkowym wykorzystano mechanizm wzajemnego wykluczenia dostępny w~bibliotece standardowej C++ w~postaci mutex'ów \cite{mutexCpp}.
\chapter{Przedstawienie działania programu i~algorytmu}
\label{cha:dzialanie}

\begin{figure}[h]
\begin{center}

\begin{subfigure}[t]{0.3\textwidth}
\includegraphics[width=\textwidth]{Operation/input.png}
\caption{Wejściowy obraz pozyskany z~kamery}
\label{fig:input}
\end{subfigure}
~
\begin{subfigure}[t]{0.3\textwidth}
\includegraphics[width=\textwidth]{Operation/reprezentacja.png}
\caption{Reprezentacji skali obrazu wejściowego}
\label{fig:dzialanieRep}
\end{subfigure}

\end{center}
\label{fig:inputIDzialanie}
\caption{Obraz wejściowy oraz fragment reprezentacji skali}
\end{figure}

W~niniejszym rodziale przedstawiono przykładowe wyniki działania algorytmu Scale Space zaimplementowanego z~użyciem OpenCL. Obrazem wejściowym jest obraz pobrany przy pomocy kamery przedstawionej w~rozdziale \ref{cha:obslugakamery} o~rozdielczości 1024x1024, który jest przedstawiony na rys \ref{fig:input}. Wszystkie obrazy są przedstawione w~skali 3.

\section{Reprezentacja skali}
\label{sec:dzialanieRep}

Reprezentacja skali jest przedstawiona na rys \ref{fig:dzialanieRep}. Jest to obraz otrzymany po zastosowaniu dolnoprzepustowego filtru Gaussa. Filtracji podlega obraz w~skali szarości, które jest otrzymany z~wyniku barwnej interpolacji obrazu z kamery. Aproksymacja jest zrealizowana również na karcie graficznej z~użyciem OpenCL. Jest to również obraz wejściowy dla dalszych kroków algorytmu.

Na przedstawionym obrazie widać, że szczegóły są mało widoczne i~zlewają się w jeden obiekt. W~sposób można otrzymać większą skalę i~rozpoznawać obiekty, które są większe, pomijając szczegóły. Reprezentacja skali składa się z~kilku takich obrazów w~różnych skalach.


\section{Detekcja plam}
\label{sec:dzialanieBlob}

\begin{figure}[h]
\begin{center}

\begin{subfigure}[t]{0.3\textwidth}
\includegraphics[width=\textwidth]{Operation/blobIntermediate.png}
\caption{Obraz pośredni używany przy detekcji plam}
\label{fig:blobIntermediate}
\end{subfigure}
~
\begin{subfigure}[t]{0.3\textwidth}
\includegraphics[width=\textwidth]{Operation/blobResult.png}
\caption{Wynik detekcji plam}
\label{fig:blobResult}
\end{subfigure}

\end{center}
\label{fig:showBlob}
\caption{Obraz pośredni oraz wynik detekcji plam}
\end{figure}

Do detekcji plam używany jest operator przedstawiony na rys. \ref{fig:laplacian_kernel}. Rys. \ref{fig:blobIntermediate} przedstawia wynik działania operatora Laplace'a na jednym z~obrazów reprezentacji skali. Jasne punkty oznaczają wartości bliskie zera. Im ciemniejszy piksel tym większa wartość. Za punkty środkowe plam (aproksymowane jako koła) są uznawane maksima lokalne, które są przedstawione na rys. \ref{fig:blobResult}. Jest to wynik działania algorytmu detekcji plam. Ciemne piksele oznaczają środek wykrytych plam.

\section{Detekcja krawędzi}
\label{sec:dzialanieEdge}

\begin{figure}[h]
\begin{center}

\begin{subfigure}[t]{0.3\textwidth}
\includegraphics[width=\textwidth]{Operation/edge01Intermediate.png}
\caption{Obraz pośredni otrzymany przy detekcji krawędzi, wynik warunku pierwszego \eqref{eq:edgeDetection}}
\label{fig:edgeIntermediate1}
\end{subfigure}
~
\begin{subfigure}[t]{0.3\textwidth}
\includegraphics[width=\textwidth]{Operation/edge02Intermediate.png}
\caption{Obraz pośredni używany przy detekcji krawędzi, wynik warunku drugiego \eqref{eq:edgeDetection}}
\label{fig:edgeIntermediate2}
\end{subfigure}

\begin{subfigure}[t]{0.3\textwidth}
\includegraphics[width=\textwidth]{Operation/edgeResult.png}
\caption{Wynik detekcji krawędzi}
\label{fig:edgeResult}
\end{subfigure}

\end{center}
\label{fig:showEdge}
\caption{Obrazy pośrednie oraz wynik detekcji krawędz}
\end{figure}

Podczas detekcji krawędzi pierwszym krokiem jest wyznaczenie wartości wyrażeń przedstawionych na równaniu \eqref{eq:edgeDetection}. Rys. \ref{fig:edgeIntermediate1}~i~\ref{fig:edgeIntermediate2} przedstawiają wartości wyżej wymienionych wyrażeń. Kolor szary na tych obrazach oznacza wartości zerowe lub bliskie zeru. Punkty ciemniejsze to wartości ujemne a punkty jaśniejsze to wartości dodatnie. Można zauważyć, że w~miejscach, gdzie występują krawędzie jest zmiana znaku wartości pierwszego warunku oraz drugi warunek przyjmuje wartości ujemne. Wynik działania algorytmu detekcji krawędzi jest przedstawiony na rys. \ref{fig:edgeResult}. Ciemne piksele występują w~miejscach, gdzie wykryto krawędź.

Z~przedstawionych obrazów wynika, że konieczna jest modyfikacja pierwszego warunku, aby wyszukiwać miejsca, gdzie pierwszy znak zmienia znak zamiast wartości zerowych.

\section{Detekcja narożników}
\label{sec:dzialanieCorner}

\begin{figure}[h]
\begin{center}

\begin{subfigure}[t]{0.3\textwidth}
\includegraphics[width=\textwidth]{Operation/cornerIntermediate.png}
\caption{Obraz pośredni używany przy detekcji narożników}
\label{fig:cornerIntermediate}
\end{subfigure}
~
\begin{subfigure}[t]{0.3\textwidth}
\includegraphics[width=\textwidth]{Operation/cornerResult.png}
\caption{Wynik detekcji narożników}
\label{fig:cornerResult}
\end{subfigure}

\end{center}
\label{fig:showEdge}
\caption{Obraz pośredni oraz wynik detekcji narożników}
\end{figure}

Podczas detekcji krawędzi pierwszym krokiem jest wyznaczenie wartości współczynnika przedstawionego w~równaniu \eqref{eq:cornerDetection}. Wyniki dla podanego obrazu wejściowego są przedstawione na rys. \ref{fig:cornerIntermediate}. Jasne punkty oznaczają wartości bliskie zera. Im ciemniejszy piksel tym większą ma wartość. Zidentyfikowane lokalne maksima są przedstawione na rys \ref{fig:cornerResult}. Jest to wynik algorytmu detekcji krawędzi. Ciemne piksele oznaczają miejsca, gdzie wykryto narożniki.

\section{Detekcja grani}
\label{sec:dzialanieRidge}

\begin{figure}[h]
\begin{center}

\begin{subfigure}[t]{0.3\textwidth}
\includegraphics[width=\textwidth]{Operation/ridge01Intermediate.png}
\caption{Obraz pośredni używany przy detekcji grani, wynik warunku pierwszego \eqref{eq:ridgeDetection}}
\label{fig:ridgeIntermediate1}
\end{subfigure}
~
\begin{subfigure}[t]{0.3\textwidth}
\includegraphics[width=\textwidth]{Operation/ridge02Intermediate.png}
\caption{Obraz pośredni używany przy detekcji grani, wynik warunku drugiego \eqref{eq:ridgeDetection}}
\label{fig:ridgeIntermediate2}
\end{subfigure}

\begin{subfigure}[t]{0.3\textwidth}
\includegraphics[width=\textwidth]{Operation/ridgeResult.png}
\caption{Wynik detekcji grani}
\label{fig:ridgeResult}
\end{subfigure}

\end{center}
\label{fig:showEdge}
\caption{Obazy pośrednie oraz wynik detekcji grani}
\end{figure}

Podczas detekcji krawędzi pierwszym krokiem jest wyznaczenie wartości wyrażeń przedstawionych na równaniu \eqref{eq:ridgeDetection}. Wartości wyznaczone za pomocą przedstawionych wyżej warunków są przedstawione na rys. \ref{fig:ridgeIntermediate1}~i~\ref{fig:ridgeIntermediate2}. Piksele szare oznaczają wartości bliskie zera. Ciemniejsze punkty to wartości ujemne, a jaśniejsze przedstawiają wartości dodatnie. Wynik detekcji grani są przedstawione na rys. \ref{fig:ridgeResult}. Ciemne piksele znajdują się w~miejscach, gdzie wykryto granie.

\chapter{Testy}
\label{cha:testy}

W niniejszym rozdziale przedstawiono proces testowania oraz ich wyniki. Obszarem zainteresowań jest poprawność zaimplementowanego algorytmu oraz porównanie szybkości działania z~innymi implementacjami.

Do porównań opracowano dwie dodatkowe implementacje algorytmu zrealizowane z użyciem biblioteki OpenCV: jedna przeprowadza obliczenia na procesorze (CPU), druga przeprowadza obliczenia na karcie graficznej (GPU) z~wykorzystaniem pakietu CUDA.

\section{Testowanie poprawności}
\label{sec:testPoprawnosc}

Celem tej części testowania było sprawdzenie czy zrealizowane implementacje działają w~ten sam sposób. Proces testowania był następujący:
\begin{enumerate}
\item Przetworzenie tego samego obrazu za pomocą wszystkich implementacji. Wybrany obraz był reprezentowany w~kilku skalach. Przeprowadzono obliczenia dla każdej rozpoznawanej cechy.
\item Zestawienie obrazów wynikowych z~podziałem na cechy oraz skalę. Dla każdej cechy i~skali otrzymano trzy obrazy, po jednym dla każdej implementacji.
\item Porównanie wyników. Idealnym wynikiem byłoby otrzymanie trzech identycznych obrazów wynikowych. W praktyce jest to mało prawdopodobne, z~uwagi na różnice podczas zaokrąglenia obliczeń w~implementacjach.
\end{enumerate}



% porównanie poprawności, obrazy różnicowe itp.

\section{Testowanie szybkości działania}
\label{sec:testSzybkosc1}

Testowanie szybkości działania oparto na porównaniu szybkości działania wszystkich implementacji algorytmu Scale Space. Liczony był czas działania algorytmu. Im krótszy czas tym lepsza implementacja algorytmu Scale Space.

Mierzeniu podlagał czas trwania następujących czynności:
\begin{itemize}
\item czas wyzaczania reprezentacji Scale Space,
\item czas wkrywania cech na obrazach (wraz z~wcześniejszym wyznaczeniem reprezentacji Scale Space) - osobno dla każdej cechy.
\end{itemize}

%opis maszyn, na których były wykonane testy

\section{Podsumowanie testów}
\label{sec:testPodsumowanie}

%podsumowanie testów

\section{Testowanie poprawności}
\label{sec:testPoprawnosc}

Celem tej części testowania było sprawdzenie czy zrealizowane implementacje działają w~ten sam sposób. Proces testowania był następujący:

\begin{enumerate}
\item Przetworzenie tego samego obrazu za pomocą wszystkich implementacji. Wybrany obraz był reprezentowany w~kilku skalach. Przeprowadzono obliczenia dla każdej rozpoznawanej cechy.
\item Zestawienie obrazów wynikowych z~podziałem na cechy oraz skalę. Dla każdej cechy i~skali otrzymano trzy obrazy, po jednym dla każdej implementacji.
\item Porównanie wyników. Idealnym wynikiem byłoby otrzymanie trzech identycznych obrazów wynikowych. W praktyce jest to mało prawdopodobne, z~uwagi na różnice podczas zaokrąglenia obliczeń w~implementacjach.
\end{enumerate}

Dla każdej rozpoznawanej cechy oraz dla samego wyznaczania reprezentacji skali wybrano pięć obrazów. Obliczenia przeprowadzono dla dziesięciu skal przy kroku cztery. Spośród otrzymanych obrazów wybrano dwa najbardziej interesujące (dla każdej cechy i~dla reprezentacji skali) i~zaprezentowano w~części \ref{subsec:prezentacjaObrazowRoznicowych}.

Wszystkie obrazy wyjściowe zostały wykorzystane do analizy poprawności implementacji. W celu porównania wszystkich implementacji opracowano dwa współczynniki przedstawione we wzorach: \eqref{eq:procentZlychPikseli} i~\eqref{eq:sredniaOdchylenia}. Wyniki oraz ich omówienie zaprezentowano w~części \ref{subsec:porownanieNumerycznePoprawnosc}.

\begin{equation}
\label{eq:procentZlychPikseli}
p_{XY} = \frac{\sum_{i}^{N}|sgn(X_{ij}-Y_{ij})	|}{N}
\end{equation}
gdzie:

$ p_{XY} $ - współczynnik ilości różnych pikseli,

$ X, Y $ - obrazy, dla których wyznaczany jest współczynnik,

$ N $ - liczba pikseli w~obrazie,

$ sgn $ - funkcja signum \cite{Signum}.

\begin{equation}
\label{eq:sredniaOdchylenia}
v_{XY} = \frac{\sum_{i}^{N}|X_{ij}-Y_{ij}|}{\sum_{i}^{N}|sgn(X_{ij}-Y_{ij})|}
\end{equation}
gdzie:

$ v_{XY} $ - współczynnik wielkości średniego odchylenia,

$ X, Y $ - obrazy, dla których wyznaczany jest współczynnik,

$ N $ - liczba pikseli w~obrazie,

$ sgn $ - funkcja signum \cite{Signum}.

Współczynnik ilości różnych pikseli pokazuje ile różnych pikseli, w~stosunku do całego obrazu, jest obecnych dla dwóch różnych implementacji. Współczynnik wielkości średniego odchylenia jest wykorzystywany tylko dla obrazów reprezentacji skali, ponieważ obrazy wyjściowe dla wyznaczania cech są binarne. Współczynnik są wyliczane dla jednego obrazu w~jednej skali dla wszystkich trzech implementacji (porównywany jest wynik każdej implementacji z~każdą).

\subsection{Prezentacja różnic w~obrazach wyjściowych}
\label{subsec:prezentacjaObrazowRoznicowych}

Poniżej przedstawiono wyniki porównań pomiędzy obrazami otrzymanymi za pomocą różnych implementacji. Przedstawione są tutaj jedynie dwa wybrane obrazy wejściowe wybrane spośród pięciu, dla każdej rozpoznawanej cechy oraz dla reprezentacji skali. Obrazy różnicowe są otoczone ramką w~celu zwiększenia czytelności.

\subsubsection{Reprezentacja skali}
\label{subsubsec:reprezentacjaSkaliRysunki}

Obrazy wejściowe przedstawione są na rys. \ref{fig:valPure02} i \ref{fig:valPure03}.

\begin{figure}[h]
\begin{center}
\includegraphics[width=0.6\textwidth]{TestyPoprawnosci/in_pure_02.png}
\end{center}
\caption{Pierwszy obraz wejściowy dla reprezentacji skali}
\label{fig:valPure02}
\end{figure}

\begin{figure}[h]
\begin{center}
\includegraphics[width=0.6\textwidth]{TestyPoprawnosci/in_pure_03.png}
\end{center}
\caption{Drugi obraz wejściowy dla reprezentacji skali}
\label{fig:valPure03}
\end{figure}

Obrazy różnicowe są przedstawione na rys. \ref{fig:valPure2} i~\ref{fig:valPure3}. 

\begin{figure}[h]
\begin{subfigure}[t]{0.3\textwidth}
	\centering
	\setlength\fboxsep{0pt}
	\setlength\fboxrule{0.5pt}
	\fbox{\includegraphics[width=\textwidth]{TestyPoprawnosci/diff_pure_CL-CVCPU_02_09.png}}
	\caption{Porównanie implementacji CL i CVCPU dla reprezentacji skali}
	\label{fig:valPure2CLCVCPU}
\end{subfigure}
~
\begin{subfigure}[t]{0.3\textwidth}
	\centering
	\setlength\fboxsep{0pt}
	\setlength\fboxrule{0.5pt}
	\fbox{\includegraphics[width=\textwidth]{TestyPoprawnosci/diff_pure_CL-CVGPU_02_09.png}}
	\caption{Porównanie implementacji CL i CVGPU dla reprezentacji skali}
	\label{fig:valPure2CLCVGPU}
\end{subfigure}
~
\begin{subfigure}[t]{0.3\textwidth}
	\centering
	\setlength\fboxsep{0pt}
	\setlength\fboxrule{0.5pt}
	\fbox{\includegraphics[width=\textwidth]{TestyPoprawnosci/diff_pure_CVCPU-CVCPU_02_09.png}}
	\caption{Porównanie implementacji CVCPU i CVGPU dla reprezentacji skali}
	\label{fig:valPure2CVCPUCVGPU}                 
\end{subfigure}
\caption{Obrazy różnicowe reprezentacji skali dla pierwszego pliku wejściowego}

\label{fig:valPure2}
\end{figure}

\begin{figure}[h]
\begin{subfigure}[t]{0.3\textwidth}
	\centering
	\setlength\fboxsep{0pt}
	\setlength\fboxrule{0.5pt}
	\fbox{\includegraphics[width=\textwidth]{TestyPoprawnosci/diff_pure_CL-CVCPU_03_09.png}}
	\caption{Porównanie implementacji CL i CVCPU dla reprezentacji skali}
	\label{fig:valPure3CLCVCPU}
\end{subfigure}
~
\begin{subfigure}[t]{0.3\textwidth}
	\centering
	\setlength\fboxsep{0pt}
	\setlength\fboxrule{0.5pt}
	\fbox{\includegraphics[width=\textwidth]{TestyPoprawnosci/diff_pure_CL-CVGPU_03_09.png}}
	\caption{Porównanie implementacji CL i CVGPU dla reprezentacji skali}
	\label{fig:valPure3CLCVGPU}
\end{subfigure}
~
\begin{subfigure}[t]{0.3\textwidth}
	\centering
	\setlength\fboxsep{0pt}
	\setlength\fboxrule{0.5pt}
	\fbox{\includegraphics[width=\textwidth]{TestyPoprawnosci/diff_pure_CVCPU-CVCPU_03_09.png}}
	\caption{Porównanie implementacji CVCPU i CVGPU dla reprezentacji skali}
	\label{fig:valPure3CVCPUCVGPU}                 
\end{subfigure}
\caption{Obrazy różnicowe reprezentacji skali dla drugiego pliku wejściowego}

\label{fig:valPure3}
\end{figure}

Na podstawie obrazów różnicowych można stwierdzić, że główne rozbieżności pomiędzy implementacjami występują głównie na krawędziach. Poza tym różnice są niewielkie. Maksymalna wartość odchylenia nie przekracza $ 0,3\% $ maksymalnej wartości piksela. Rozbieżności pomiędzy implementacjami są spowodowane błędami związanymi z~ograniczaniemi systemów obliczeniowych (m. in. błąd zaokrąglenia). Istotne obiekty, które mają być rozpoznane znajdują się w~środku obrazu, nie w~bezpośrednim sąsiedztwie krawędzi, dlatego różnice na brzegu obrazu nie są istotne dla dalszego przetwarzania. W centrum obrazu widać także niewielkie różne, lecz są one niewielkie (podobnie, nie przekraczają $ 0,3\% $ maksymalnej wartości piksela). Widać je zwłaszcza na obrazach \ref{fig:valPure3CLCVCPU} i~\ref{fig:valPure3CLCVCPU}. Można też zauważyć, że różnice występują pojedynczo, nie są zgrupowane.

Implementacje wykonane w~OpenCV dla liczenia reprezentacji skali dają identyczne wyniki, co można zauważyć na obrazach \ref{fig:valPure2CVCPUCVGPU} i~\ref{fig:valPure3CVCPUCVGPU}.

\subsubsection{Plamy}
\label{subsubsec:plamyRysunki}

Obrazy wejściowe przedstawione są na rys. \ref{fig:valBlob01} i \ref{fig:valBlob02}.

\begin{figure}[h]
\begin{center}
\includegraphics[width=0.6\textwidth]{TestyPoprawnosci/in_blob_01.png}
\end{center}
\caption{Pierwszy obraz wejściowy dla rozpoznawania plam}
\label{fig:valBlob01}
\end{figure}

\begin{figure}[h]
\begin{center}
\includegraphics[width=0.6\textwidth]{TestyPoprawnosci/in_blob_02.png}
\end{center}
\caption{Drugi obraz wejściowy dla rozpoznawania plam}
\label{fig:valBlob02}
\end{figure}

Obrazy różnicowe są przedstawione na rys. \ref{fig:valBlob1} i~\ref{fig:valBlob2}. 

\begin{figure}[h]
\begin{subfigure}[t]{0.3\textwidth}
	\centering
	\setlength\fboxsep{0pt}
	\setlength\fboxrule{0.5pt}
	\fbox{\includegraphics[width=\textwidth]{TestyPoprawnosci/diff_blob_CL-CVCPU_01_09.png}}
	\caption{Porównanie implementacji CL i CVCPU dla rozpoznawania plam}
	\label{fig:valBlob2CLCVCPU}
\end{subfigure}
~
\begin{subfigure}[t]{0.3\textwidth}
	\centering
	\setlength\fboxsep{0pt}
	\setlength\fboxrule{0.5pt}
	\fbox{\includegraphics[width=\textwidth]{TestyPoprawnosci/diff_blob_CL-CVGPU_01_09.png}}
	\caption{Porównanie implementacji CL i CVGPU dla rozpoznawania plam}
	\label{fig:valBlob2CLCVGPU}
\end{subfigure}
~
\begin{subfigure}[t]{0.3\textwidth}
	\centering
	\setlength\fboxsep{0pt}
	\setlength\fboxrule{0.5pt}
	\fbox{\includegraphics[width=\textwidth]{TestyPoprawnosci/diff_blob_CVCPU-CVCPU_01_09.png}}
	\caption{Porównanie implementacji CVCPU i CVGPU dla rozpoznawania plam}
	\label{fig:valblob2CVCPUCVGPU}                 
\end{subfigure}
\caption{Obrazy różnicowe rozpoznawania plam dla pierwszego pliku wejściowego}

\label{fig:valBlob1}
\end{figure}

\begin{figure}[h]
\begin{subfigure}[t]{0.3\textwidth}
	\centering
	\setlength\fboxsep{0pt}
	\setlength\fboxrule{0.5pt}
	\fbox{\includegraphics[width=\textwidth]{TestyPoprawnosci/diff_blob_CL-CVCPU_02_09.png}}
	\caption{Porównanie implementacji CL i CVCPU dla rozpoznawania plam}
	\label{fig:valBlob3CLCVCPU}
\end{subfigure}
~
\begin{subfigure}[t]{0.3\textwidth}
	\centering
	\setlength\fboxsep{0pt}
	\setlength\fboxrule{0.5pt}
	\fbox{\includegraphics[width=\textwidth]{TestyPoprawnosci/diff_blob_CL-CVGPU_02_09.png}}
	\caption{Porównanie implementacji CL i CVGPU dla rozpoznawania plam}
	\label{fig:valBlob3CLCVGPU}
\end{subfigure}
~
\begin{subfigure}[t]{0.3\textwidth}
	\centering
	\setlength\fboxsep{0pt}
	\setlength\fboxrule{0.5pt}
	\fbox{\includegraphics[width=\textwidth]{TestyPoprawnosci/diff_blob_CVCPU-CVCPU_02_09.png}}
	\caption{Porównanie implementacji CVCPU i CVGPU dla rozpoznawania plam}
	\label{fig:valBlob3CVCPUCVGPU}                 
\end{subfigure}
\caption{Obrazy różnicowe rozpoznawania plam dla drugiego pliku wejściowego}

\label{fig:valBlob2}
\end{figure}

Z~przedstawionych obrazów wynika, że podczas rozpoznawania plam, najwięcej różnic powstaje w~okolicach krawędzi. Jest to dobrze widoczne na obrazach \ref{fig:valBlob2CLCVCPU} i~\ref{fig:valBlob2CLCVGPU}. Porównując do wyników testów dla reprezentacji skali można wywnioskować że te różnice są spowodowane rozbieżnościami powstałymi podczas filtracji Gaussa.

Na rys. \ref{fig:valBlob2} widać również różnice w~środku obrazu. Różnice te są spowodowane błędami związanymi z~ograniczaniemi systemów obliczeniowych (m. in. błąd zaokrąglenia). Liczba różnych pikseli nie jest duża, lecz większa niż w~przypadku innych rozpoznawanych cech.


\subsubsection{Krawędzie}
\label{subsubsec:krawedzieRysunki}

Obrazy wejściowe przedstawione są na rys. \ref{fig:valEdge00} i \ref{fig:valEdge02}.

\begin{figure}[h]
\begin{center}
\includegraphics[width=0.6\textwidth]{TestyPoprawnosci/in_edge_00.png}
\end{center}
\caption{Pierwszy obraz wejściowy dla rozpoznawania krawędzi}
\label{fig:valEdge00}
\end{figure}

\begin{figure}[h]
\begin{center}
\includegraphics[width=0.6\textwidth]{TestyPoprawnosci/in_edge_02.png}
\end{center}
\caption{Drugi obraz wejściowy dla rozpoznawania krawędzi}
\label{fig:valEdge02}
\end{figure}

Obrazy różnicowe są przedstawione na rys. \ref{fig:valEdge0} i~\ref{fig:valEdge2}. 

\begin{figure}[h]
\begin{subfigure}[t]{0.3\textwidth}
	\centering
	\setlength\fboxsep{0pt}
	\setlength\fboxrule{0.5pt}
	\fbox{\includegraphics[width=\textwidth]{TestyPoprawnosci/diff_edge_CL-CVCPU_00_00.png}}
	\caption{Porównanie implementacji CL i CVCPU dla rozpoznawania krawędzi}
	\label{fig:valEdge0CLCVCPU}
\end{subfigure}
~
\begin{subfigure}[t]{0.3\textwidth}
	\centering
	\setlength\fboxsep{0pt}
	\setlength\fboxrule{0.5pt}
	\fbox{\includegraphics[width=\textwidth]{TestyPoprawnosci/diff_edge_CL-CVGPU_00_00.png}}
	\caption{Porównanie implementacji CL i CVGPU dla rozpoznawania krawędzi}
	\label{fig:valEdge0CLCVGPU}
\end{subfigure}
~
\begin{subfigure}[t]{0.3\textwidth}
	\centering
	\setlength\fboxsep{0pt}
	\setlength\fboxrule{0.5pt}
	\fbox{\includegraphics[width=\textwidth]{TestyPoprawnosci/diff_edge_CVCPU-CVCPU_00_00.png}}
	\caption{Porównanie implementacji CVCPU i CVGPU dla rozpoznawania krawędzi}
	\label{fig:valEdge0CVCPUCVGPU}                 
\end{subfigure}
\caption{Obrazy różnicowe rozpoznawania krawędzi dla pierwszego pliku wejściowego}

\label{fig:valEdge0}
\end{figure}

\begin{figure}[h]
\begin{subfigure}[t]{0.3\textwidth}
	\centering
	\setlength\fboxsep{0pt}
	\setlength\fboxrule{0.5pt}
	\fbox{\includegraphics[width=\textwidth]{TestyPoprawnosci/diff_edge_CL-CVCPU_02_03.png}}
	\caption{Porównanie implementacji CL i CVCPU dla rozpoznawania krawędzi}
	\label{fig:valEdge2CLCVCPU}
\end{subfigure}
~
\begin{subfigure}[t]{0.3\textwidth}
	\centering
	\setlength\fboxsep{0pt}
	\setlength\fboxrule{0.5pt}
	\fbox{\includegraphics[width=\textwidth]{TestyPoprawnosci/diff_edge_CL-CVGPU_02_03.png}}
	\caption{Porównanie implementacji CL i CVGPU dla rozpoznawania krawędzi}
	\label{fig:valEdge2CLCVGPU}
\end{subfigure}
~
\begin{subfigure}[t]{0.3\textwidth}
	\centering
	\setlength\fboxsep{0pt}
	\setlength\fboxrule{0.5pt}
	\fbox{\includegraphics[width=\textwidth]{TestyPoprawnosci/diff_edge_CVCPU-CVCPU_02_03.png}}
	\caption{Porównanie implementacji CVCPU i CVGPU dla rozpoznawania krawędzi}
	\label{fig:valEdge2CVCPUCVGPU}                 
\end{subfigure}
\caption{Obrazy różnicowe rozpoznawania krawędzi dla drugiego pliku wejściowego}

\label{fig:valEdge2}
\end{figure}

Na obrazach różnicowych powstałych poczas rozpoznawania krawędzi można zauważyć, że licza różnych pikseli jest niewielka. Zdecydowana większość tych pikseli znajduje się na krawędziach. Jest to spowodowane różnicami powstałymi na etapie tworzenia reprezentacji skali. W~środku obrazu również można zauważyć różnice, lecz jest ich niewielka liczba.

\subsubsection{Narożniki}
\label{subsubsec:naroznikiRysunki}

Obrazy wejściowe przedstawione są na rys. \ref{fig:valCorner01} i \ref{fig:valCorner02}.

\begin{figure}[h]
\begin{center}
\includegraphics[width=0.6\textwidth]{TestyPoprawnosci/in_corner_01.png}
\end{center}
\caption{Pierwszy obraz wejściowy dla rozpoznawania narożników}
\label{fig:valCorner01}
\end{figure}

\begin{figure}[h]
\begin{center}
\includegraphics[width=0.6\textwidth]{TestyPoprawnosci/in_corner_02.png}
\end{center}
\caption{Drugi obraz wejściowy dla rozpoznawania narożników}
\label{fig:valCorner02}
\end{figure}

Obrazy różnicowe są przedstawione na rys. \ref{fig:valCorner1} i~\ref{fig:valCorner2}. 

\begin{figure}[h]
\begin{subfigure}[t]{0.3\textwidth}
	\centering
	\setlength\fboxsep{0pt}
	\setlength\fboxrule{0.5pt}
	\fbox{\includegraphics[width=\textwidth]{TestyPoprawnosci/diff_corner_CL-CVCPU_01_07.png}}
	\caption{Porównanie implementacji CL i CVCPU dla rozpoznawania narożników}
	\label{fig:valCorner1CLCVCPU}
\end{subfigure}
~
\begin{subfigure}[t]{0.3\textwidth}
	\centering
	\setlength\fboxsep{0pt}
	\setlength\fboxrule{0.5pt}
	\fbox{\includegraphics[width=\textwidth]{TestyPoprawnosci/diff_corner_CL-CVGPU_01_07.png}}
	\caption{Porównanie implementacji CL i CVGPU dla rozpoznawania narożników}
	\label{fig:valCorner1CLCVGPU}
\end{subfigure}
~
\begin{subfigure}[t]{0.3\textwidth}
	\centering
	\setlength\fboxsep{0pt}
	\setlength\fboxrule{0.5pt}
	\fbox{\includegraphics[width=\textwidth]{TestyPoprawnosci/diff_corner_CVCPU-CVCPU_01_07.png}}
	\caption{Porównanie implementacji CVCPU i CVGPU dla rozpoznawania narożników}
	\label{fig:valCorner1CVCPUCVGPU}                 
\end{subfigure}
\caption{Obrazy różnicowe rozpoznawania narożników dla pierwszego pliku wejściowego}

\label{fig:valCorner1}
\end{figure}

\begin{figure}[h]
\begin{subfigure}[t]{0.3\textwidth}
	\centering
	\setlength\fboxsep{0pt}
	\setlength\fboxrule{0.5pt}
	\fbox{\includegraphics[width=\textwidth]{TestyPoprawnosci/diff_corner_CL-CVGPU_02_06.png}}
	\caption{Porównanie implementacji CL i CVGPU dla rozpoznawania narożników}
	\label{fig:valCorner2CLCVGPU}
\end{subfigure}
~
\begin{subfigure}[t]{0.3\textwidth}
	\centering
	\setlength\fboxsep{0pt}
	\setlength\fboxrule{0.5pt}
	\fbox{\includegraphics[width=\textwidth]{TestyPoprawnosci/diff_corner_CL-CVGPU_02_06.png}}
	\caption{Porównanie implementacji CL i CVGPU dla rozpoznawania narożników}
	\label{fig:valCorner2CLCVGPU}
\end{subfigure}

\begin{subfigure}[t]{0.3\textwidth}
	\centering
	\setlength\fboxsep{0pt}
	\setlength\fboxrule{0.5pt}
	\fbox{\includegraphics[width=\textwidth]{TestyPoprawnosci/diff_corner_CL-CVCPU_02_06.png}}
	\caption{Porównanie implementacji CL i CVCPU dla rozpoznawania narożników}
	\label{fig:valCorner2CLCVCPU}
\end{subfigure}
~
\begin{subfigure}[t]{0.3\textwidth}
	\centering
	\setlength\fboxsep{0pt}
	\setlength\fboxrule{0.5pt}
	\fbox{\includegraphics[width=\textwidth]{TestyPoprawnosci/diff_corner_CL-CVGPU_02_06.png}}
	\caption{Porównanie implementacji CL i CVGPU dla rozpoznawania narożników}
	\label{fig:valCorner2CLCVGPU}
\end{subfigure}
~
\begin{subfigure}[t]{0.3\textwidth}
	\centering
	\setlength\fboxsep{0pt}
	\setlength\fboxrule{0.5pt}
	\fbox{\includegraphics[width=\textwidth]{TestyPoprawnosci/diff_corner_CVCPU-CVCPU_02_06.png}}
	\caption{Porównanie implementacji CVCPU i CVGPU dla rozpoznawania narożników}
	\label{fig:valCorner2CVCPUCVGPU}                 
\end{subfigure}
\caption{Obrazy różnicowe rozpoznawania narożników dla drugiego pliku wejściowego}

\label{fig:valCorner2}
\end{figure}

Na obrazach różnicowych można zauważyć pojedyncze różnice. Jest to też efektem tego, że liczba wykrywanych narożników na przedstawionych obrazach w~wybranej skali nie jest duża. Mniejsza liczba wykrytych obiektów powoduje, że liczba błędów jest mniejsza.

Na rys. \ref{fig:valCorner2} nie widać żadnych różnic.

\subsubsection{Granie}
\label{subsubsec:granieRysunki}

Obrazy wejściowe przedstawione są na rys. \ref{fig:valRidge01} i~\ref{fig:valRidge04}.

\begin{figure}[h]
\begin{center}
\includegraphics[width=0.6\textwidth]{TestyPoprawnosci/in_ridge_01.png}
\end{center}
\caption{Pierwszy obraz wejściowy dla rozpoznawania grani}
\label{fig:valRidge01}
\end{figure}

\begin{figure}[h]
\begin{center}
\includegraphics[width=0.6\textwidth]{TestyPoprawnosci/in_ridge_04.png}
\end{center}
\caption{Drugi obraz wejściowy dla rozpoznawania grani}
\label{fig:valRidge04}
\end{figure}

Obrazy różnicowe są przedstawione na rys. \ref{fig:valRidge1} i~\ref{fig:valRidge4}. 

\begin{figure}[h]
\begin{subfigure}[t]{0.3\textwidth}
	\centering
	\setlength\fboxsep{0pt}
	\setlength\fboxrule{0.5pt}
	\fbox{\includegraphics[width=\textwidth]{TestyPoprawnosci/diff_ridge_CL-CVCPU_01_01.png}}
	\caption{Porównanie implementacji CL i CVCPU dla rozpoznawania grani}
	\label{fig:valRidge1CLCVCPU}
\end{subfigure}
~
\begin{subfigure}[t]{0.3\textwidth}
	\centering
	\setlength\fboxsep{0pt}
	\setlength\fboxrule{0.5pt}
	\fbox{\includegraphics[width=\textwidth]{TestyPoprawnosci/diff_ridge_CL-CVGPU_01_01.png}}
	\caption{Porównanie implementacji CL i CVGPU dla rozpoznawania grani}
	\label{fig:valRidge1CLCVGPU}
\end{subfigure}
~
\begin{subfigure}[t]{0.3\textwidth}
	\centering
	\setlength\fboxsep{0pt}
	\setlength\fboxrule{0.5pt}
	\fbox{\includegraphics[width=\textwidth]{TestyPoprawnosci/diff_ridge_CVCPU-CVCPU_01_01.png}}
	\caption{Porównanie implementacji CVCPU i CVGPU dla rozpoznawania grani}
	\label{fig:valRidge1CVCPUCVGPU}                 
\end{subfigure}
\caption{Obrazy różnicowe rozpoznawania grani dla pierwszego pliku wejściowego}

\label{fig:valRidge1}
\end{figure}

\begin{figure}[h]
\begin{subfigure}[t]{0.3\textwidth}
	\centering
	\setlength\fboxsep{0pt}
	\setlength\fboxrule{0.5pt}
	\fbox{\includegraphics[width=\textwidth]{TestyPoprawnosci/diff_ridge_CL-CVCPU_04_04.png}}
	\caption{Porównanie implementacji CL i CVCPU dla rozpoznawania grani}
	\label{fig:valRidge4CLCVCPU}
\end{subfigure}
~
\begin{subfigure}[t]{0.3\textwidth}
	\centering
	\setlength\fboxsep{0pt}
	\setlength\fboxrule{0.5pt}
	\fbox{\includegraphics[width=\textwidth]{TestyPoprawnosci/diff_ridge_CL-CVGPU_04_04.png}}
	\caption{Porównanie implementacji CL i CVGPU dla rozpoznawania grani}
	\label{fig:valRidge4CLCVGPU}
\end{subfigure}
~
\begin{subfigure}[t]{0.3\textwidth}
	\centering
	\setlength\fboxsep{0pt}
	\setlength\fboxrule{0.5pt}
	\fbox{\includegraphics[width=\textwidth]{TestyPoprawnosci/diff_ridge_CVCPU-CVCPU_04_04.png}}
	\caption{Porównanie implementacji CVCPU i CVGPU dla rozpoznawania grani}
	\label{fig:valRidge4CVCPUCVGPU}                 
\end{subfigure}
\caption{Obrazy różnicowe rozpoznawania grani dla drugiego pliku wejściowego}

\label{fig:valRidge4}
\end{figure}

Na obrazach różnicowych można zauważyć, że na krawędziach powstało bardzo dużo różnic. Są one pochodną rozbieżności powstałych na etapie tworzenia reprezentacji skali. W~środku obrazu widać tylko pojedyncze różnice. 

\subsection{Numeryczna analiza obrazów wynikowych}
\label{subsec:porownanieNumerycznePoprawnosc}

W~poniższych akapitach przedstawiono numeryczne porównanie obrazów otrzymanych z~użyciem trzech implementacji. Wykorzystano do tego celu współczynniki przedstawione na równaniach \eqref{eq:procentZlychPikseli}~i~\eqref{eq:sredniaOdchylenia}.

Do sporządzenia zestawień wykorzystano wcześniej pokazane obrazy, po dwa dla każdej cechy i~dla reprezentacji skali. Przeprowadzono analizę ilości różniących się pikseli oraz dla reprezentacji skali średnią różnicę dla dziesięciu skal. 

Z przedstawionych danych wynika, że implementacje CVCPU i~CVGPU są prawie identyczne. Z~tego powodu uwaga w~niniejszym podrozdziale jest skupiona na porównaniu implementacji wykonanej z~wykorzystaniem OpenCL z~implementacjami wykonanymi z~użyciem biblioteki OpenCV.

\afterpage{
\begin{landscape}
\begin{table}[h]
\caption{Wartości współczynnika $ e $ dla wcześniej przedstawionych obrazów podzielone na skale}
\label{tab:imageScaleRep2}
\begin{tabular}{|p{0.8cm}|p{2cm}|p{2cm}|p{2cm}|p{2cm}|p{2cm}|p{2cm}|p{2cm}|p{2cm}|p{2cm}|p{2cm}|}
\hline
Nr skali & Obraz I  reprezentacja skali [\%] & Obraz II  reprezentacja skali [\%] & Obraz III  plamy [\%] & Obraz IV  plamy [\%] & Obraz V  krawędzie [\%] & Obraz VI  krawędzie [\%] & Obraz VII  narożniki [\%] & Obraz VIII  narożniki [\%] & Obraz IX  granie [\%] & Obraz X  granie [\%] \\ \hline
1        & 0.05                          & 0.02                           & 0.70              & 0.14             & 0.06                & 0.01                 & 0.03                  & 0.02                   & 0.10              & 0.09             \\ \hline
2        & 0.08                          & 0.04                           & 0.84              & 0.15             & 0.03                & 0.01                 & 0.01                  & 0.00                   & 0.35              & 0.30             \\ \hline
3        & 0.11                          & 0.05                           & 0.68              & 0.15             & 0.02                & 0.01                 & 0.01                  & 0.00                   & 0.63              & 0.58             \\ \hline
4        & 0.14                          & 0.07                           & 0.55              & 0.13             & 0.01                & 0.00                 & 0.00                  & 0.00                   & 0.75              & 0.95             \\ \hline
5        & 0.17                          & 0.10                           & 0.48              & 0.13             & 0.01                & 0.00                 & 0.00                  & 0.00                   & 0.89              & 1.17             \\ \hline
6        & 0.20                          & 0.12                           & 0.61              & 0.13             & 0.00                & 0.00                 & 0.00                  & 0.00                   & 1.12              & 1.52             \\ \hline
7        & 0.23                          & 0.15                           & 0.41              & 0.13             & 0.00                & 0.00                 & 0.00                  & 0.00                   & 1.29              & 1.91             \\ \hline
8        & 0.26                          & 0.18                           & 0.32              & 0.13             & 0.00                & 0.00                 & 0.00                  & 0.00                   & 1.56              & 2.08             \\ \hline
9        & 0.28                          & 0.22                           & 0.45              & 0.13             & 0.00                & 0.00                 & 0.00                  & 0.00                   & 1.75              & 1.76             \\ \hline
10       & 0.31                          & 0.25                           & 0.34              & 0.14             & 0.00                & 0.00                 & 0.00                  & 0.00                   & 1.83              & 1.04             \\ \hline
\end{tabular}
\end{table}
\end{landscape}}

\begin{center}
\begin{table}
\centering
\caption{Wartości współczynnika $ e $ dla reprezentacji skali dla obrazu \ref{fig:valPure2} podzielone na skale}
\label{tab:imageScaleRep2}
\begin{tabular}{|c|r|r|r|}
 \hline
Nr skali & CL-CVCPU (\%) & CL-CVGPU (\%) & CVCPU-CVGPU (\%) \\ \hline
1        & 0.05     & 0.05     & 0.00        \\ \hline
2        & 0.08     & 0.08     & 0.00        \\ \hline
3        & 0.11     & 0.11     & 0.00        \\ \hline
4        & 0.14     & 0.14     & 0.00        \\ \hline
5        & 0.17     & 0.17     & 0.00        \\ \hline
6        & 0.20     & 0.20     & 0.00        \\ \hline
7        & 0.23     & 0.23     & 0.00        \\ \hline
8        & 0.26     & 0.26     & 0.00        \\ \hline
9        & 0.28     & 0.28     & 0.00        \\ \hline
10       & 0.31     & 0.31     & 0.00        \\ \hline
\end{tabular}
\end{table}
\end{center}

\begin{center}
\begin{table}
\centering
\caption{Wartości współczynnika $ e $ dla reprezentacji skali dla obrazu \ref{fig:valPure3} podzielone na skale}
\label{tab:imageScaleRep3}
\begin{tabular}{|c|r|r|r|}
 \hline
Nr skali & CL-CVCPU (\%) & CL-CVGPU (\%) & CVCPU-CVGPU (\%) \\ \hline
1        & 0.02     & 0.02     & 0.00        \\ \hline
2        & 0.04     & 0.04     & 0.00        \\ \hline
3        & 0.05     & 0.05     & 0.00        \\ \hline
4        & 0.07     & 0.07     & 0.00        \\ \hline
5        & 0.10     & 0.10     & 0.00        \\ \hline
6        & 0.12     & 0.12     & 0.00        \\ \hline
7        & 0.15     & 0.15     & 0.00        \\ \hline
8        & 0.18     & 0.18     & 0.00        \\ \hline
9        & 0.22     & 0.22     & 0.00        \\ \hline
10       & 0.25     & 0.25     & 0.00        \\ \hline
\end{tabular}
\end{table}
\end{center}

\subsubsection{Reprezentacja skali}
\label{subsubsec:reprezentacjaSakliTabele}

W~tabelach \ref{tab:imageScaleRep2} i~\ref{tab:imageScaleRep3} przedstawiono liczbę różnych pikseli dla porównań implementacji podczas tworzenia reprezentacji skali podzieloną na skale dla wybranych obrazów. Można zauważyć, że wraz ze wzrostem skali rośnie liczba pikseli, które się różnią. Jest to związane z~nieco odmiennym traktowaniu pikseli, które są poza obrazem. Dla większych skal istnieje więcej takich pikseli, dlatego stosunek różnych pikseli do wszystkich pikseli rośnie wraz ze wzrostem skali.

Porównując wyniki dla obu obrazów można zauważyć, że w~obrazie \ref{fig:valPure3} występuje mniej różnic. Ponieważ najwięcej różnic występuje w~otoczeniu brzegów obrazu to największe wartości współczynnika $ e $ będą otrzymywane dla obrazów, które posiadają złożone struktury na brzegach obrazu. Obraz \ref{fig:valPure3} ma mniej szczegółów w~otoczeniu brzegu obrazu, dlatego wyliczone wartości są mniejsze, niż dla obrazu \ref{fig:valPure2}. Drugą własnością obrazu, która może wpływać znacząco na wartości współczynnika $ e $ jest wielkość obrazu. Dla mniejszych obrazów stosunek liczby pikseli leżących w~pobliżu brzegu obrazu do liczby wszystkich pikseli będzie większy, co spowoduje wzrost liczby różnic w~wynikach otrzymanych za pomocą różnych implementacji. Niska wartość obliczona dla obrazu czwartego wynika z~tego, że na tym obrazie obszary bliskie krawędzi są jednorodne.

\begin{center}
\begin{table}
\centering
\caption{Wartość współczynnika $ v $ dla reprezentacji skali dla obrazu \ref{fig:valPure2} podzielone na skale}
\label{tab:devScaleRep2}
\begin{tabular}{|c|r|r|r|}
\hline
Nr skali & CL-CVCPU (\%) & CL-CVGPU (\%) & CVCPU-CVGPU (\%) \\ \hline
1        & 1.12     & 1.12     & 0.00        \\ \hline
2        & 1.16     & 1.16     & 0.00        \\ \hline
3        & 1.18     & 1.18     & 0.00        \\ \hline
4        & 1.20     & 1.20     & 0.00        \\ \hline
5        & 1.23     & 1.23     & 0.00        \\ \hline
6        & 1.26     & 1.26     & 0.00        \\ \hline
7        & 1.28     & 1.28     & 0.00        \\ \hline
8        & 1.31     & 1.31     & 0.00        \\ \hline
9        & 1.33     & 1.33     & 0.00        \\ \hline
10       & 1.36     & 1.36     & 0.00        \\ \hline
\end{tabular}
\end{table}
\end{center}

\begin{center}
\begin{table}
\centering
\caption{Wartość współczynnika $ v $ dla reprezentacji skali dla obrazu \ref{fig:valPure3} podzielone na skale}
\label{tab:devScaleRep3}
\begin{tabular}{|c|r|r|r|}
\hline
Nr skali & CL-CVCPU (\%) & CL-CVGPU (\%) & CVCPU-CVGPU (\%) \\ \hline
1        & 1.00     & 1.00     & 0.00        \\ \hline
2        & 1.00     & 1.00     & 0.00        \\ \hline
3        & 1.00     & 1.00     & 0.00        \\ \hline
4        & 1.00     & 1.00     & 0.00        \\ \hline
5        & 1.00     & 1.00     & 0.00        \\ \hline
6        & 1.00     & 1.00     & 0.00        \\ \hline
7        & 1.00     & 1.00     & 0.00        \\ \hline
8        & 1.00     & 1.00     & 0.00        \\ \hline
9        & 1.00     & 1.00     & 0.00        \\ \hline
10       & 1.00     & 1.00     & 0.00        \\ \hline
\end{tabular}
\end{table}
\end{center}

W~tabelach \ref{tab:devScaleRep2} i~\ref{tab:devScaleRep3} przedstawiono średnią różnicę pomiędzy pikselami dla porównań implemetacji podczas tworzenia reprezentacji skali podzielona na skale dla wybranych obrazów. Wnioski, które można wyciągną z~przedstawionych danych są analogiczne jak podczas analizy tabel przedstawiających wartości współczynnika $ e $ dla reprezentacji skali. Dla obrazów o~większej liczbie szczegółów poza liczbą różnic rośnie również wielkość tych różnic. Podobnie dla obrazów mniejszych, gdy rośnie stosunek liczby pikseli leżących w~pobliżu pikseli brzegowych, to równocześnie ze wzrostem liczby różnic rośnie ich wielkość.

Można zauważyć, że dla obrazu \ref{fig:valPure3} wartość współczynnika $ v $ jest równa jeden niezależnie od skali.


\begin{center}
\begin{table}
\centering
\caption{Wartości współczynnika $ e $ dla detekcji plam dla obrazu \ref{fig:valBlob1} podzielone na skale}
\label{tab:imageScaleBlob1}
\begin{tabular}{|c|r|r|r|}
 \hline
Nr skali & CL-CVCPU (\%) & CL-CVGPU (\%) & CVCPU-CVGPU (\%) \\ \hline
1        & 0.70     & 0.70     & 0.00        \\ \hline
2        & 0.84     & 0.84     & 0.00       \\ \hline
3        & 0.68     & 0.68     & 0.00       \\ \hline
4        & 0.55     & 0.55     & 0.00        \\ \hline
5        & 0.48     & 0.48     & 0.00        \\ \hline
6        & 0.61     & 0.61     & 0.00        \\ \hline
7        & 0.41     & 0.41     & 0.00        \\ \hline
8        & 0.32     & 0.32     & 0.00        \\ \hline
9        & 0.45     & 0.45     & 0.00        \\ \hline
10       & 0.34     & 0.34     & 0.00        \\ \hline
\end{tabular}
\end{table}
\end{center}

\begin{center}
\begin{table}
\centering
\caption{Wartości współczynnika $ e $ dla detekcji plam dla obrazu \ref{fig:valBlob2} podzielone na skale}
\label{tab:imageScaleBlob2}
\begin{tabular}{|c|r|r|r|}
\hline
Nr skali & CL-CVCPU (\%) & CL-CVGPU (\%) & CVCPU-CVGPU (\%) \\ \hline
1        & 0.14     & 0.14     & 0.00       \\ \hline
2        & 0.15     & 0.15     & 0.00        \\ \hline
3        & 0.15     & 0.15     & 0.00       \\ \hline
4        & 0.13     & 0.14     & 0.00       \\ \hline
5        & 0.13     & 0.13     & 0.00       \\ \hline
6        & 0.13     & 0.13     & 0.00       \\ \hline
7        & 0.13     & 0.13     & 0.00       \\ \hline
8        & 0.13     & 0.13     & 0.00       \\ \hline
9        & 0.13     & 0.13     & 0.00       \\ \hline
10       & 0.14     & 0.14     & 0.00       \\ \hline
\end{tabular}
\end{table}
\end{center}

\subsubsection{Plamy}
\label{subsubsec:plamyTabele}

liczba różnych pikseli maleje wraz ze wzrostem skali dla skal większych od dwóch. Wartość współczynnika dla skali pierwszej jest mniejsza niż dla skali drugiej. 

W~tabelach \ref{tab:imageScaleBlob1} i~\ref{tab:imageScaleBlob2} przedstawiono liczbę różnych pikseli dla porównań implementacji rozpoznawania plam podzieloną na skale dla wybranych obrazów. Można zauważyć, że największa wartość współczynnika $ e $ jest osiągana dla skali drugiej. Ogólny trend zmniejszania się liczby różnic jest spowodowany dużą liczbą wykrywanych plam w~niższych skalach. Większa liczba wykrywanych obiektów generuje więcej różnic. Wpływ zwiększania się liczby pikseli mających w~swoim otoczeniu piksele brzegowe na liczbę różnic jest ograniczony. Jest to spowodowane tym, że liczba wykrywanych plam maleje szybciej niż wzrasta liczba rozbieżności spowodowanych różnicami powstałymi na etapie tworzenia reprezentacji skali.

Wartości współczynnika $ e $ dla obrazu \ref{fig:valBlob1} są największymi wyznaczonymi wartościami dla rozpoznawania plam. Ten obraz, jest najmniejszym z~analizowanych obrazów. Można zauważyć, że zdecydowana większość różnic powstała przy brzegach obrazu, co wraz z~małym rozmiarem powoduje, wysoką wartość współczynnika $ e $. Dla tego obrazu widać również, że wartości współczynnika $ e $, pomimo trendu malejącego wraz ze wzrostem skali nie maleją przy każdej kolejnej skali. Wynika to z~rosnącej liczby różnic pojawiąjących się podczas tworzenia reprezentacji skali.

\begin{center}
\begin{table}
\centering
\caption{Wartości współczynnika $ e $ dla detekcji krawędzi dla obrazu \ref{fig:valEdge0} podzielone na skale}
\label{tab:imageScaleEdge0}
\begin{tabular}{|c|r|r|r|}
 \hline
Nr skali & CL-CVCPU (\%) & CL-CVGPU (\%) & CVCPU-CVGPU (\%) \\ \hline
1        & 0.06     & 0.06     & 0.000       \\ \hline
2        & 0.03     & 0.03     & 0.00        \\ \hline
3        & 0.02     & 0.02     & 0.00        \\ \hline
4        & 0.01     & 0.01     & 0.00        \\ \hline
5        & 0.01     & 0.01     & 0.00        \\ \hline
6        & 0.00     & 0.00     & 0.00        \\ \hline
7        & 0.00     & 0.00     & 0.00        \\ \hline
8        & 0.00     & 0.00     & 0.00        \\ \hline
9        & 0.00     & 0.00     & 0.00        \\ \hline
10       & 0.00     & 0.00     & 0.00        \\ \hline
\end{tabular}
\end{table}
\end{center}

\begin{center}
\begin{table}
\centering
\caption{Wartości współczynnika $ e $ dla detekcji krawędzi dla obrazu \ref{fig:valEdge2} podzielone na skale}
\label{tab:imageScaleEdge2}
\begin{tabular}{|c|r|r|r|}
 \hline
Nr skali & CL-CVCPU (\%) & CL-CVGPU (\%) & CVCPU-CVGPU (\%) \\ \hline
1        & 0.01     & 0.01     & 0.00        \\ \hline
2        & 0.01     & 0.01     & 0.00        \\ \hline
3        & 0.01     & 0.01     & 0.00        \\ \hline
4        & 0.00     & 0.00     & 0.00        \\ \hline
5        & 0.00     & 0.00     & 0.00        \\ \hline
6        & 0.00     & 0.00     & 0.00        \\ \hline
7        & 0.00     & 0.00     & 0.00        \\ \hline
8        & 0.00     & 0.00     & 0.00        \\ \hline
9        & 0.00     & 0.00     & 0.00        \\ \hline
10       & 0.00     & 0.00     & 0.00        \\ \hline
\end{tabular}
\end{table}
\end{center}

\subsubsection{Krawędzie}
\label{subsubsec:krawedzieTabele}

W~tabelach \ref{tab:imageScaleEdge0} i~\ref{tab:imageScaleEdge2} przedstawiono liczbę różnych pikseli dla porównań implementacji podczas rozpoznawania krawędzi podzieloną na skale dla wybranych obrazów. Można zauważyć, że stosunek rozbieżnych pikseli do wszystkich pikseli na obrazie maleje wraz ze wzrostem skali. Zmniejszanie się podanego stosunku wraz ze wzrostem skali jest spowodowany mniejszą liczbą wykrytych krawędzi na obrazach w~kolejnych skalach. Efekt zwiększania się liczby pikseli, które mają w~swoim otoczeniu piksele brzegowe nie jest w~tym przypadku zauważalny. Jest to spowodowane również znacznym zmniejszeniem liczby pikseli, które są uznawane za piksele krawędzi oraz tym, że krawędzie są w~niewielkim stopniu wykrywane na brzegach obrazów.


\begin{center}
\begin{table}
\centering
\centering
\caption{Wartości współczynnika $ e $ dla detekcji narożników dla obrazu \ref{fig:valCorner1} podzielone na skale}
\label{tab:imageScaleCorner1}
\begin{tabular}{|c|r|r|r|}
 \hline
Nr skali & CL-CVCPU (\%) & CL-CVGPU (\%) & CVCPU-CVGPU (\%) \\ \hline
1        & 0.03     & 0.03     & 0.00        \\ \hline
2        & 0.01     & 0.01     & 0.00        \\ \hline
3        & 0.01     & 0.01     & 0.00        \\ \hline
4        & 0.00     & 0.00     & 0.00        \\ \hline
5        & 0.00     & 0.01     & 0.00        \\ \hline
6        & 0.00     & 0.00     & 0.00        \\ \hline
7        & 0.00     & 0.00     & 0.00        \\ \hline
8        & 0.00     & 0.00     & 0.00        \\ \hline
9        & 0.00     & 0.00     & 0.00        \\ \hline
10       & 0.00     & 0.00     & 0.00        \\ \hline
\end{tabular}
\end{table}
\end{center}

\begin{center}
\begin{table}
\centering
\centering
\caption{Wartości współczynnika $ e $ dla detekcji narożników dla obrazu \ref{fig:valCorner2} podzielone na skale}
\label{tab:imageScaleCorner2}
\begin{tabular}{|c|r|r|r|}
 \hline
Nr skali & CL-CVCPU (\%) & CL-CVGPU (\%) & CVCPU-CVGPU (\%) \\ \hline
1        & 0.02     & 0.02     & 0.00        \\ \hline
2        & 0.00     & 0.00     & 0.00        \\ \hline
3        & 0.00     & 0.00     & 0.00        \\ \hline
4        & 0.00     & 0.00     & 0.00        \\ \hline
5        & 0.00     & 0.00     & 0.00        \\ \hline
6        & 0.00     & 0.00     & 0.00        \\ \hline
7        & 0.00     & 0.00     & 0.00        \\ \hline
8        & 0.00     & 0.00     & 0.00        \\ \hline
9        & 0.00     & 0.00     & 0.00        \\ \hline
10       & 0.00     & 0.00     & 0.00        \\ \hline
\end{tabular}
\end{table}
\end{center}

\subsubsection{Narożniki}
\label{subsubsec:naroznikiTabele}

W~tabelach \ref{tab:imageScaleCorner1} i~\ref{tab:imageScaleCorner2} przedstawiono liczbę różnych pikseli dla porównań implementacji podczas rozpoznawania narożników podzieloną na skale dla wybranych obrazów. Można zauważyć, że wartości współczynników są niskie i~maleją wraz ze zwiększaniem skali. W wyniku analizy wszystkich obrazów zauważono, że maksymalne wartości współczynnika $ e $ dla detekcji narożników nie przekracza $ 0,3 \% $. Ogólna niska wartość współczynników wynika z~niewielkiej liczby wykrywanych obiektów oraz dobrej jakości detektora narożników. Zmniejszanie się liczby rozbieżności jest związane z~malejącą liczbą wykrywanych obiektów.

W~tym przypadku również nie jest zauważalny wpływ zwiększającej się liczby pikseli mający w~swoim otoczeniu piksele brzegowe na większą liczbę błędów. Jest to spowodowane analogicznymi powodami jak w~przypadku detekcji krawędzi.


\begin{center}
\begin{table}
\centering
\caption{Wartości współczynnika $ e $ dla detekcji grani dla obrazu \ref{fig:valRidge1} podzielone na skale}
\label{tab:imageScaleRidge1}
\begin{tabular}{|c|r|r|r|}
 \hline
Nr skali & CL-CVCPU & CL-CVGPU & CVCPU-CVGPU \\ \hline
1        & 0.10     & 0.10     & 0.00        \\ \hline
2        & 0.35     & 0.35     & 0.00        \\ \hline
3        & 0.63     & 0.63     & 0.00        \\ \hline
4        & 0.75     & 0.75     & 0.00        \\ \hline
5        & 0.89     & 0.89     & 0.00        \\ \hline
6        & 1.12     & 1.12     & 0.00        \\ \hline
7        & 1.29     & 1.30     & 0.00        \\ \hline
8        & 1.56     & 1.55     & 0.00        \\ \hline
9        & 1.75     & 1.75     & 0.00        \\ \hline
10       & 1.83     & 1.83     & 0.00        \\ \hline
\end{tabular}
\end{table}
\end{center}


\begin{center}
\begin{table}
\centering
\caption{Wartości współczynnika $ e $ dla detekcji grani dla obrazu \ref{fig:valRidge4} podzielone na skale}
\label{tab:imageScaleRidge4}
\begin{tabular}{|c|r|r|r|}
 \hline
Nr skali & CL-CVCPU & CL-CVGPU & CVCPU-CVGPU \\ \hline
1        & 0.09     & 0.09     & 0.00        \\ \hline
2        & 0.30     & 0.30     & 0.00        \\ \hline
3        & 0.58     & 0.58     & 0.00        \\ \hline
4        & 0.95     & 0.95     & 0.00        \\ \hline
5        & 1.17     & 1.17     & 0.00        \\ \hline
6        & 1.52     & 1.52     & 0.00        \\ \hline
7        & 1.91     & 1.91     & 0.00        \\ \hline
8        & 2.08     & 2.08     & 0.00        \\ \hline
9        & 1.76     & 1.76     & 0.00        \\ \hline
10       & 1.04     & 1.04     & 0.00        \\ \hline
\end{tabular}
\end{table}
\end{center}

\subsubsection{Granie}
\label{subsubsec:granieTabele}

W~tabelach \ref{tab:imageScaleRidge1} i~\ref{tab:imageScaleRidge4} przedstawiono liczbę różnych pikseli dla porównań implementacji podczas rozpoznawania grani podzieloną na skale dla wybranych obrazów. Można zauważyć, że wartości współczynnika $ e $ rosną wraz ze wzrostem skali. Jest to spowodowane przez różnice wprowadzone podczas tworzenia reprezentacji skali oraz przez słabą jakość detektora grani. Detektor grani daje gorsze wyniki, niż pozostałe detektory. Dodatkowym czynnikiem wpływającym na większą liczbę różnic dla większych skal jest większa ilość pikseli, które zostały zakwalifikowane jako piksele grani podczas przetwarzania.

\section{Testowanie szybkości działania}
\label{sec:testSzybkosci}

Testowanie szybkości działania oparto na porównaniu szybkości działania wszystkich implementacji algorytmu Scale Space. Liczony był czas działania algorytmu. Im krótszy czas tym lepsza implementacja algorytmu Scale Space.

Mierzeniu podlagał czas trwania następujących czynności:
\begin{itemize}
\item czas wyzaczania reprezentacji Scale Space,
\item czas wkrywania cech na obrazach (wraz z~wcześniejszym wyznaczeniem reprezentacji Scale Space) - osobno dla każdej cechy.
\end{itemize}



%\include{realizacja}
%\include{czesc_trzecia}
%\include{technologie}
%\include{wnioski}
\chapter{Wnioski i podsumowanie}
\label{cha:wnioski}


%\include{rozdzial2}


\bibliographystyle{plain}
\bibliography{praca}{}

\appendix
\chapter{Instrukcja}
\label{cha:instrukcja}

Kr�tka instrukcja u�ytkownia stworzonego programu.

Program jest uruchamiany w~konsoli i~nie posiada �adnego graficznego interfejsu u�ytkowania. Do programu do��czona jest pomoc. Parametry podawane s� w~spos�b zgodne ze standardem znanym z~projektu GNU.

Program przyjmuje nast�puj�ce parametry, kt�re zosta�y opisane w~poni�szym spisie. Ka�dy parametr mo�e by� podany w formie kr�tkiej lub d�ugiej.

\begin{itemize}
\item -i,--in - input file instead of use camera. Default input file type is gray.
\item -I,--in-prefix -input files wille be in format: prefix_XX.bmp. It ends when files ends (i. e. no next file). Default input files type is bayer.
\item -o,--out - output files prefix. Output file will be PREFIX_PROCESSOR_IMAGENUMBER_SCALENUMBER.bmp.
\item -m,--mode - Scale Space mode. One of: blob, corner, edge, ridge.
\item -t,--type - Input file type. Can be bayer or gray. It works only with file.
\item -p,--processor - Select implementation. It can be:
	\begin{itemize}
		\item cl, opencl - for OpenCL implementation.
        \item cv, opencv, cpu - for OpenCV implementation.
        \item cv_gpu, opencv_gpu, gpu - for OpenCV GPU implementation.
	\end{itemize}
\item -s,--scale - Set scales in format a b, where a is scale step and b is number of scales.
\item --no-show - Don't save images on drive.
\item -d,--debug - Show intermediate images. It can be very slow.
\item -q,--quiet - Don't show any output.
\item --no-first-image - Don't measure time for first image.
\end{itemize}
\chapter{Lista modułów systemu}
\label{cha:dokumentacja}

Program komputerowy ma budowę modułową i~składa się z~przedstawionych poniżej modułów:
\begin{itemize}
\item \texttt{OpenCLInterface} - w~tym module znajdują się funkcje obsługujące procedury OpenCL. Jest on przedstawiony w~rozdziale \ref{cha:opencl}.
\item \texttt{ScaleSpace} - moduł, w~którym są zrealizowane wszystkie trzy implementacje algorytmu ScaleSpace. Główną częścią tego modułu są klas implementujące algorytm. Do tworzenia tych klasy została zrealizowana fabryka. W~skład modułu wchodzi również klasa, która przechowuje wszystkie obrazy używane podczas realizacji algorytmu, wraz z~danymi wejściowymi i~wyjściowymi.
\item \texttt{Main} - jest to moduł, który jest punktem startowym aplikacji oraz kontroluje jej wykonanie. Przetwarza również parametry wejściowe zgodnie z~ich formatem przedstawiony w~dodatku \ref{cha:instrukcja}.
\item \texttt{TTime} - ten moduł jest odpowiedzialny za pomiar czasu wykonania. Składa się z~jednej klasy, której zadaniem jest implementacja funkcji stopera.
\item \texttt{JAICameraInterface} - moduł odpowiedzialny za obsługę kamery. Obsługa jest zgodna z~opisem przedstawionym w~rodziale \ref{cha:obslugakamery}.
\end{itemize}

\chapter{Zaimplementowane kernele}
\label{cha:kernele}

W~trakcie realizacji praca stworzono poniżej przedstawione kernele. Wszystkie kernele używają zmiennoprzecikowego typu danych chyba, że napisano inaczej.

\begin{itemize}
\item 
\texttt{\_\_kernel void findLocalMax(\_\_read\_only image2d\_t input, \_\_write\_only image2d\_t output)} - kernel znajdujący maksima lokalne na obrazie wejściowym. Używany jest przy detekcji plam oraz narożników. Do obrazu wyjściowego są zapisywane wartości 255 dla współrzędnych punktów, które zostały uznane za maksima lokalne obrazu wejściowego.

\item 
\texttt{\_\_kernel void  edge\_max(\_\_read\_only image2d\_t Lvv\_image, \_\_write\_only image2d\_t output, \_\_read\_only image2d\_t Lvvv\_image)} - kernel wyznaczający punkty, w~których znaleziono krawędzie. Obrazami wejściowymi są wartości wyznaczone za pomocą warunków opisanych w~równaniu \eqref{eq:edgeDetection}. Do obrazu wyjściowego są zapisywane wartości 255 w~punktach, które zostały uznane za krawędzie zgodnie z~warunkami.

\item 
\texttt{\_\_kernel void  ridge\_max(\_\_read\_only image2d\_t L1\_image, \_\_write\_only image2d\_t output, \_\_read\_only image2d\_t L2\_image)} - kernel wyznaczający punkty, w~których znaleziono granie. Obrazami wejściowymi są wartości wyznaczone za pomocą warunków opisanych w~równaniu \eqref{eq:ridgeDetection}. Do obrazu wyjściowego są zapisywane wartości 255 w~punktach, które zostały uznane za granie zgodnie z~warunkami.

\item 
\texttt{\_\_kernel void  intToFloat8bit(\_\_read\_only image2d\_t input, \_\_write\_only image2d\_t output)} - kernel zamieniający obraz wejściowy typu całkowitoliczbowego o~rozmiarze jednego bajta na obraz wyjściowy typu zmiennoprzecikowego pojedynczej precyzji.

\item 
\texttt{\_\_kernel void rgb2gray(\_\_read\_only image2d\_t input, \_\_write\_only image2d\_t output)} - kernel zamieniający obraz wejściowy zapisanego w~przestrzeni barw RGB na obraz w~skali szarości.

\item 
\texttt{\_\_kernel void  floatToUInt8ThreeChannels(\_\_read\_only image2d\_t input, \_\_write\_only image2d\_t output)} - kernel zamieniający obraz wejściowy typu zmiennoprzecikowego pojedynczej precyzji na obraz wyjściowy typu całkowitoliczbowego o~rozmiarze jednego bajta .

\item 
\texttt{\_\_kernel void  edge\_detector(\_\_read\_only image2d\_t input, \_\_write\_only image2d\_t out\_Lvv, \_\_write\_only image2d\_t out\_Lvvv)} - kernel wyznaczający wartości współczynników, które są używane do wyznaczania krawędzi i~są przedstawione na~równaniu \eqref{eq:edgeDetection}. Obrazem wejściowym jest obraz reprezentacji skali, a~w~obrazach wyjściwoych są zapisane wartości pikseli wyznaczone z~równania.

\item 
\texttt{\_\_kernel void  corner\_detector(\_\_read\_only image2d\_t input, \_\_write\_only image2d\_t output)} - kernel wyznaczający wartości współczynnika, który jest używany do wyznaczania narożików i~jest przedstawiony na równaniu \eqref{eq:cornerDetection}. Obrazem wejściowym jest obraz reprezentacji skali, a~w~obrazie wyjściwoym jest zapisana wartość pikseli wyznaczona z~równania.

\item 
\texttt{\_\_kernel void  blob\_detector(\_\_read\_only image2d\_t input, \_\_write\_only image2d\_t output)} - kernel wyznaczający wartości współczynnika, który jest używany do wyznaczania plam i~jest przedstawiony na rys. \ref{fig:laplacian_kernel}. Obrazem wejściowym jest obraz reprezentacji skali, a~w~obrazie wyjściwoym jest zapisana wartość pikseli wyznaczona zgodnie z~algorytmem.

\item 
\texttt{\_\_kernel void  ridge\_detector(\_\_read\_only image2d\_t input, \_\_write\_only image2d\_t out\_L1, \_\_write\_only image2d\_t out\_L2)} - kernel wyznaczający wartości współczynników, które są używane do wyznaczania grani i~są przedstawione na~równaniu \eqref{eq:ridgeDetection}. Obrazem wejściowym jest obraz reprezentacji skali, a~w~obrazach wyjściwoych są zapisane wartości pikseli wyznaczone z~równania.

\item 
\texttt{\_\_kernel void  convolution(\_\_read\_only image2d\_t input, \_\_write\_only image2d\_t output, \_\_global float * gaussian, \_\_private \_\_read\_only uint size)} - kernel liczący reprezentację skali. Obrazem wejściowym jest obraz w~skali szarości. Filtr gaussa jest zapisany jako tablica jednowymiarowa w~celu przyśpieszczenia obliczeń.

\end{itemize}

\include{zawartoscCD}

\end{document}
c