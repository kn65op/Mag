\documentclass[pdflatex,11pt]{aghdpl}
% \documentclass{aghdpl}               % przy kompilacji programem latex
% \documentclass[pdflatex,en]{aghdpl}  % praca w języku angielskim
\usepackage[polish]{babel}
\usepackage[utf8]{inputenc}

% dodatkowe pakiety
\usepackage{enumerate}
\usepackage{listings}
\usepackage{hyperref}
\usepackage{amsmath}
\usepackage{listings}
%\usepackage{graphicx}
\lstloadlanguages{TeX}

%---------------------------------------------------------------------------

\author{Tomasz Drzewiecki}
\shortauthor{T. Drzewiecki}

\titlePL{Równoległa realizacja algorytmu Scale Space przy pomocy procesorów graficznych}
\titleEN{Parallel implementation of ScaleSpace Algorithm on GPU processors}

\shorttitlePL{Równoległa realizacja algorytmu Scale Space} % skrócona wersja tytułu jeśli jest bardzo długi
\shorttitleEN{Parallel implementation of ScaleSpace Algorithm}

\thesistypePL{Praca dyplomowa magisterska}
\thesistypeEN{Master's Thesis}

\supervisorPL{dr inż. Mirosław Jabłoński}
\supervisorEN{Mirosław Jabłoński, Ph.D}

\date{2013}

\departmentPL{Katedra Automatyki i~Inżynierii Biomedycznej}
\departmentEN{Department of Automatics and Biomedical Engineering}

\facultyPL{Wydział Elektrotechniki, Automatyki, Informatyki i~Inżynierii Biomedycznej}
\facultyEN{Faculty of Electrical Engineering, Automatics, Computer Science and Biomedical Engineering}

\acknowledgements{Serdecznie dziękuję }



\setlength{\cftsecnumwidth}{10mm}

%---------------------------------------------------------------------------

\begin{document}

\titlepages

\tableofcontents
\clearpage

\chapter{Wprowadzenie}
\label{cha:wprowadzenie}

W~tym rozdziale przedstawiono cel pracy oraz podstawę teoretyczną realizacji. Opisano szczegółowo implementowany algorytm Scale Space: podstawy matematyczne i~sposób działania. Przybliżono również standard OpenCL, który został wykorzystany do realizacji algorytmu.


\section{Cel pracy}
\label{sec:cel}
Celem niniejszej pracy była realizacja algortmu Scale Space, który służy do reprezentacji syngłów (w~szczególności obrazów) w~wielu skalach w celu późniejszej ich analizy.
Wiele algorytmów operujących na obrazach jest złożona obliczeniowo i~ich realizacja zajmuje dużo czasu, zwłaszcza dla dużych porcji danych. Dlatego zdecydowano się na realizację tego algorytmu z użyciem akceleracji w~postaci zrównoleglenia obliczeń z~użyciem karty lub kart graficznych (GPU).

Implementacja algorytmu została zrealizowana w~OpenCL \cite{OpenCL}, wieloplatformowym standardzie do zrównoleglania obliczeń. Wybrano ten standard, ponieważ jest dość rozpowszechnony oraz wspierany przez wszystkich największych producentów sprzętu komputerowego (m. in. Intel\textsuperscript{\textregistered}, NVIDIA\textsuperscript{\texttrademark}, AMD).

\section{Algorytm Scale Space}
\label{sec:algorytm}
Scale Space jest algorytmem służącym do przkształcenia sygnałów do reprezentacji skali. Reprezentacja skali jest to rodzina sygnałów reprezentująca oryginalny syngał w~różnych stopniach skali.
Ta reprezentacja pozwala na analizę oryginalnego sygnału w~różnych stopniach szczegółowości.
Pomysł na stworzenie takiego algorytmu wynika z~wieloskalowej struktury świata. Taka struktura powoduje, że obiekty mogą być różnie postrzegane w~zależności od skali obserwacji \cite{Enc09}.
W~tej pracy przetwarzanymi sygnałami będą obrazy.

Tworząc system automatycznego rozpoznawania obrazów nieznanych scen nie ma możliwości określenie z~góry w~jakiej skali będą przedstawiane obiekty, które będą interesujące dla użytkownika systemu. Dlatego w~takim systemie można użyć algorytmu Scale Space wraz z~automatycznym rozpoznawaniem najbajdziej interesujące skali. Po stworzeniu reprezentacji skali można zrealizować rozpoznawanie brzegów, obiektów. Te opracje mogą być przeprowadzane tylko w~jednej skali lub dla większej ilości skal, w~szcególności we wszystkich skalach.


\subsection{Filtracja Gaussa}
\label{subsec:filtracjaGaussa}
Filtry Gaussa są jednymi z podstawowych operacji wykorzystywanych w przetwarzaniu obrazów cyfrowych. Są to filtry dolnoprzepustowe, rozmywające obraz, po zastosowaniu których ze sceny można odczytać ogólne kształty przedstawionych obiektów. Po tej operacji szczegóły zostają usunięte, bądź zostaje znacznie zmniejszony ich wpływ na całość.

Kolejne filtry Gaussa w przestrzeni ciągłej dwuwymiarowej są określone wzorem \ref{eq:gaussian}:
\begin{equation}
\label{eq:gaussian}
g(x,y,\sigma)=\frac{1}{2 \cdot \pi \cdot ^ {2} }\cdot e^{(-\frac{x^{2} + y^{2}}{2 \cdot \sigma ^{2}})}
\end{equation}
gdzie:\\
$ x,y $ - położenie piksela na obrazie, \\
$ \sigma $ - wariancja.
\newline
Wariancja w~powyższym wzorze określa skalę, w~jakiej obraz wyjściowy jest przedstawiony. 

Ponieważ podczas obilczeń z~użyciem komputera nie jest możliwe używanie przestrzeni ciągłej, dlatego konieczene jest wprowadzenie reprezentacji dyskretnej. 
Dla dwuwymiarowych sygnałów dyskretnych, zostały użyte filtry Gaussa określone wzorem \ref{eq:gaussian_discrete}:
\begin{equation}
\label{eq:gaussian_discrete}
g(x,y,N,\sigma) = \alpha \cdot e^{-((x+y)-N/2)^2/(2 \cdot \sigma)^2}
\end{equation}
gdzie: \newline 
$ x, y$ - współrzędne obrazu, \newline
$ \alpha $ - współczynnik skalujący w celu normalizacji ($ \sum_x \sum_y g(x,y,N,\sigma) = 1 $), \newline
$ N $ - rozmiar filtru, \newline
$ \sigma $ - wariancja obliczona zgodnie ze wzorem $ \sigma = 0.3 \cdot (N \cdot 0,5 - 1) + 0,8$. \newline
W~tym przypadku rozmiar filtru określa skalę.

\subsection{Reprezentacja skali}
\label{subsec:reprezentacjaskali}
Reprezentacja skali dla sygnałów ciągłych dwuwymiarowych (np. obrazów) powstaje w sposób przedstawiony we wzorze \ref{eq:scalespace}:

\begin{equation}
\label{eq:scalespace}
\begin{split}
& L(\cdot,\cdot,0) = f(\cdot,\cdot) \\
& L(\cdot,\cdot,\sigma) = g(\cdot,\cdot,\sigma)\cdot f(\cdot,\cdot)
\end{split}
\end{equation}
gdzie:\\
$ f $ - sygnał oryginalny, \\
$ g $ - filtr Gaussa, \\
$ \sigma $ - wariancja (parametr skali).

Oznacza to, że w celu uzyskania reprezentacji skali obraz poddawany jest filtracji Gaussa, z~różnymi rosnącymi wartościami dyskretnymi $ \sigma $. Obraz bez zastosowania skali to obraz oryginalny.

W~ten sposób można otrzymać wiele wynikowych obrazów, w~których każdy przedstawia początkową w scenę w różnej skali. Dzięki temu można analizować obraz w~różnym stopniu szczegółowości, co jest jedną z głównych zalet algorymtu Scale Space.

Ponieważ filtry Gaussa spełniają aksjomaty Scale Space, to ich użycie gwarantuje nam, że libcza ekstremów lokalnych nie zwiększy się. Również wartości ekstremów nie zostaną zwiększone. Oznacza to, że wartość pikseli w~maksimach lokalnych nie będzie rosła, a~wartość pikseli w minimach lokalnych nie będzie malała. Także w~przestrzeni dyskretnej aksjomany Scale Space są spełnione \cite{SSFDS}.

%\subsection{Rozpoznawnie}
%\label{subsec:rozpoznawanie}

\subsection{Złożność obliczeniowa}
\label{subsec:zlozonosc_obliczeniowa}

\section{OpenCL}
\label{sec:OpenCL}

Ponieważ realizacja algorytmu Scale Space wymaga przeprowadzenia bardzo dużej liczby obliczeń, zdecydowano, że algorytm będzie zrealizowany z użyciem karty graficznej. Do wykorzystania kart graficznych w~dowolnych obliczeniach można zastosować otwarty standard OpenCL lub technologie stworzone przez producentów procesorów graficznych dedykowanych dla urządzeń danego producenta.

OpenCL jest to otwarty, wieloplatformowy standard pozwalający na realizację algorytmów w~sposób równoległy. Umożliwa realizację jednego algorytmu na wielu różnego typu urządzeniach: procesorach wielordzeniowych, kartach graficznych oraz innych, wszystkich które go wspierają[trzeba podać jakie inne obecnie są wspierane]. Standard jest wspierany przez wszystich największych producentów sprzętu elektronicznego, więc jego zastosowanie pozwala stworzyć oprogramowanie, które może być wykorzystane w~praktycznie dowolnej konfiguracji urządzeń przetwarzających dane.

Powyższe zalety spowodowały, że do implentacji został wybrany standard OpenCL, zamiast użycia technologii dedykowanych do urządzeń produkowanych przez konkretne firmy. Przykładem takich rozwiązań są: CUDA lub ATI Stream. Rozwiązania dostarczane przez producentów urządzeń są szybsze niż implementacja standardu OpenCL [!!!!!!Czy zawsze? Trzeba podać jakieś referencje lub wyniki własnych testów.!!!!!!!]. Różnica nie jest jednak duża, więc jest to do zaakceptowania, zwłaszcza w~porówaniu do korzyści osianiętych z wykorzystania OpenCL.



\chapter{OpenCL}
\label{cha:opencl}

W~niniejszym rozdziale przedstawiono sposób implementacji algorytmów z~użyciem standardu OpenCL oraz stworzoną bibliotekę służącą ograniczeniu liczby błędów oraz przyśpieszeniu pracy poprzez uniknięcie powtarzania kodu. Zdecydowano się na realizację danej biblioteki z~uwagi na to, że w~części programu, która kontroluje wykonanie kodu na karcie graficznej (nazywanego kernelami) wiele czynności jest powtarzanych przy implementacji każdego algorytmu. Zatem zadaniem biblioteki jest ułatwienie implementacji kontrolera nie ograniczając możliwości oferowanych przez standard OpenCL oraz umożliwienie łatwego rozszerzania dostępnych funkcji. Biblioteka była tworzona oraz testowana dla wersji 1.1 standardu OpenCL. Przedstawiono i~omówiono również część z~zaimplementowanych kerneli.

\section{Implementacja algorytmów z~użyciem OpenCL}  
\label{sec:szczegolyOpenCL}

W celu implementacji algorytmów z~użyciem standardu OpenCL konieczne jest napisanie dwóch oddzielnych części. Pierwszą jest kod algorytmu zapisany w~języku stworzonym w ramach standardu OpenCL, drugą jest kod wykonywany na procesorze ogólnego przeznaczania wykorzystujący API, które jest zdefiniowane w języku C.

Algorytmy implementowane są w~postaci funkcji, które są nazywane kernelami. Są one napisane w~języku stworzonym przez twórców standardu OpenCL. Bazuje on na języku ISO C99 i~zawiera jego podzbiór wraz z~rozszerzeniami służącymi implementacji algorytmów w sposób równoległy, ułatwiającymi obsługę wektorów, macierzy oraz obrazów. Maksymalna wielkość obrazów, osobno dla obrazów 2D i 3D, jest określona dla konkretnego urządzenia. Kompilacja kodu napisanego w~tym języku odbywa się w~trakcie działania programu kontrolera, ponieważ sposób kompilacji jest zależny od posiadanego urządzenia, na którym będzie wykonywany kod. Jest to spowodowane tym, że kod języka OpenCL jest kompilowany zależnie od platformy, na której jest uruchamiany. Może to powodować problemy z~przenoszalnością pomiędzy różnymi urządzeniami w~przypadku, gdy używane są rozszerzenia OpenCL. W~celu uniknięcia problemów należy stosować konstrukcje języka, które są częścią podstawowej wersji standardu. Ta część jest wystarczająca do większości algorytmów.

Zadania kodu kontrolera można podzielić na dwa typy. Pierwszym, jest inicjalizacja, podczas której jest dokonywany wybór urządzenia docelowego, kompilacja kerneli oraz przygotowanie do wykonania. Proces wyboru urządzenia może być bardziej skomplikowany w przypadku, gdy w~systemie jest dostępnych więcej niż jedna platforma OpenCL. Przykładem takiej sytuacji jest posiadanie dwóch kart graficznych w~jednym komputerze. W~zależności od parametrów urządzeniem może być pożądane użycie jednej, lepszej karty lub obu. OpenCL pozwala na wykorzystanie obu rozwiązań. Dzięki temu, że kod kerneli może być wykonywany asynchronicznie można przeprowadzać obliczenia jednocześnie na wielu urządzeniach. Po wyborze platformy docelowej należy skompilować kernel, który należy następnie przygotować do wykonania. W~tym celu należy utworzyć obiekty, które będą parametrami kernela. Mogą to być wektory danych lub obrazy. Po stworzeniu tych obiektów, należy przypisać je do argumentów kernela, zgodnie z~deklaracją w~kodzie.

Drugim fragmentem kodu kontrolera jest przekazywanie parametrów, wykonywanie kerneli oraz pobieranie wyników. Te czynności są realizowane przy wykonaniu każdego algorytmu. Dane, z~których korzysta kernel są kopiowane pomiędzy pamięcią urządzenia a~pamięcią RAM komputera, na którym jest wykonywany kontroler. Każdą z~tych operacji można wykonać synchroniczne lub asynchronicznie. Przy wykonaniu asynchronicznym jest możliwość, aby zarejestrować funkcję zwrotną lub poczekać na zakończenie wykonania.

\section{Zrealizowana biblioteka}  
\label{sec:biblioteka}

Biblioteka zastała zaimplementowana w języku C++. Wybrano ten język ponieważ jego integracja z~API OpenCL napisanym w języku C jest prosta. Biblioteka została zrealizowana w~sposób obiektowy oraz z~użyciem wyjątków ułatwiających obsługę sytuacji krytycznych. Więcej szczegółów dotyczących wyjątków przedstawiono w~sekcji \ref{subsec:obslugabledow}. Obiektowość pozwala na ograniczenie redundancji napisanego kodu odpowiedzialnego za kompilację, przekazywanie argumentów, wykonywanie kernel oraz pobieranie wyników. Aby to umożliwić należy opracować zbiór reguł, których należy przestrzegać podczas pisania kerneli. Są one opisane w sekcji \ref{subsec:regulykerneli}. Reguły te nie ograniczają możliwości oferowanych przez standard. Biblioteka oferuje również możliwość realizacji algorytmów przetwarzających obrazy w~sposób potokowy. Taka realizacja ma zastosowanie wtedy, kiedy zachodzi potrzeba użycia kilku kerneli, dla których dane wejściowe kolejnego programu są danymi wyjściowymi poprzedniego. Implementacja  ogranicza liczbę operacji kopiowania danych pomiędzy pamięcią RAM komputera a~pamięcią wewnętrzną urządzenia, na którym wykonywany jest program poprzez zachowywanie cząstkowych obliczeń w~pamięci urządzenia.

W bibliotece można wyróżnić dwie części: część odpowiedzialną za kompilację kerneli i~wszelkie powiązane z~tym kwestie oraz część odpowiedzialną za wykonywanie kerneli.
Poniżej są przedstawione możliwości oraz szczegóły techniczne poszczególnych części.

\subsection{Kompilacja kerneli oraz obsługa urządzeń}
\label{subsec:kompilacjakerneli}

\begin{figure}
\begin{center}
%v[width=3cm]
\includegraphics[width=0.2\textwidth]{Diagrams/OpenCLDevice.png}
\end{center}
\caption{Klasa OpenCL Device}
\label{fig:opencldevice}
\end{figure}

Funkcjonalność ta jest realizowana przez jedną klasę przedstawioną na rysunku \ref{fig:opencldevice}. Jest ona odpowiedzialna za pobieranie lisy dostępnych urządzeń, pobieranie podstawowych informacji o urządzeniach, tworzenie i~udostępnianie kontekstu oraz kolejki komend (fragmentów standardu OpenCL) oraz za kompilację kerneli na konkretne urządzenie. Kontekst oraz kolejka komend są obiektami z API OpenCL za pomocą, których przeprowadzane jest wykonanie kodu na urządzeniu OpenCL.

Kod kernela może być podany w~jednej z~dwóch postaci: jako zmienna typu std::string, której zawartością jest kod lub jako zapisany w~osobnym pliku, który jest czytany i~przekazywany do funkcji wykonującej kompilację. Umożliwienie dwóch metod jest spowodowane różnymi ich właściwościami. Podczas tworzenia algorytmu bardziej przydatne jest używanie pliku, ponieważ łatwiej go edytować.  Wadą tego rozwiązania jest umieszczenie danego pliku w~odpowiednim miejscu w~systemie plików. Gdy praca nad kernelem jest zakończona wtedy można umieścić jego kod w~źródłach programu aby uniknąć problemów z~przenoszeniem aplikacji. Jest to operacja jednorazowa w~przypadku ostatecznej wersji kernela, więc nie istnieje problem edycji kodu.

\subsection{Wykonywanie kerneli}
\label{subsec:wykonywaniekerneli}

\begin{figure}
\begin{center}
%v[width=3cm]
\includegraphics[width=\textwidth]{Diagrams/OpenCLAlgorithms.png}
\end{center}
\caption{Klasy algorytmów}
\label{fig:openclalgorithms}
\end{figure}

Funkcjonalność ta ułatwia użytkownikowi proces kompilacji oraz wykonywania kerneli wraz z~przekazywaniem argumentów oraz pobieraniem wyników. Ten fragment biblioteki składa się z~wielu klas, których schemat jest przedstawiony na rys. \ref{fig:openclalgorithms}. Duża liczba klas jest spowodowana potrzebą umożliwienia zastosowania biblioteki do wielu rodzajów kerneli. Prace nad biblioteką były skupione na realizacji algorytmów operujących na obrazach. Konieczność wprowadzenia dużego rozróżnienia wynika z dużej liczby struktur wykorzystywanych w~standardzie do przekazywania parametrów wejściowych i wyjściowych oraz chęcią uściślenia zastosowania danego algorytmu.

Zastosowanie klas przedstawionych na rysunku \ref{fig:openclalgorithms} umożliwia, w~optymalnych przypadkach, na minimalne zaangażowanie użytkownika biblioteki w~tworzenie kodu kontrolera i~skupienie się na konstrukcji kodu kernela. Koniecznymi do podania informacjami są: format obrazu danych wejściowych i~wyjściowych oraz kod kernela (zawarty w~pliku lub w~łańcuchu znakowym). Dla bardziej zaawansowanych algorytmów, do działania których potrzebnych jest więcej argumentów wejściowych lub danych rezultatów konieczne jest więcej pracy włożonej w~napisanie kodu kontrolera. Dla takich przypadków zrealizowano funkcje pomocnicze. Aby stworzyć właną klasę odpowiedzialną za realizację danego kernela należy dziedziczyć po odpowiednio wybranej klasie. Wybór klasy jest zależny od danych jakie kernel przyjmuje i~jakie zwraca.

\subsection{Tworzenie potoków kerneli}
\label{subsec:potokikerneli}

\begin{figure}
\begin{center}
\includegraphics[width=\textwidth]{Diagrams/OpenCLStream.png}
\end{center}
\caption{Klasa algorytmów używane w potokach}
\label{fig:diagrampotok}
\end{figure}

Mechanizmem często wykorzystywanym podczas przetwarzania danych, w~tym obrazów są potoki. Pozwalają na łączenie kilku elementarnych operacji w~jeden ciąg zadań. Wiele z~tych mniejszych operacji jest używana w~większej liczbie algorytmów, dlatego pożądane jest, aby umożliwić wielokrotne użycie raz napisanego kernela.

W~bibliotece zaimplementowano możliwość stworzenia w~prosty sposób potoku używającego jako danych wejściowych i~wyjściowych obrazów dwuwymiarowych. Klasy konieczne do realizacji potoku są przedstawione na rys \ref{fig:diagrampotok}. Zaimplementowane rozwiązanie posiada wady oraz zalety. Przy tradycyjnym podejściu wykonania kerneli konieczne jest każdorazowe kopiowanie danych z~i~do pamięci urządzenia. Użycie potoku pozwala na ograniczenie ilości kopiowanych danych. Zaimplementowano sposób wykonania kolejnych kerneli w~sposób sekwencyjny, pozwalający na bezpośrednie wykorzystanie wyników obliczeń jako argumentów wejściowych kolejnego kernela. Dodatkową zaletą jest możliwości wykorzystania kerneli stworzonych do potoku jako samodzielny algorytm przy małym nakładzie pracy oraz braku konieczności edycji kodu kernela. Istnieje możliwość zwrócenia przez ostatni kernel potoku dodatkowej wartości, poza wynikowym obrazem. Wadą użycia potoku jest utrudniona możliwość zwrócenia dodatkowych danych przez kernele, które nie są ostatnimi w~potoku. Znacznie utrudnione jest też pobranie wyników pośrednich.

\begin{figure}
\begin{center}
\includegraphics[width=0.2\textwidth]{Diagrams/OpenCLAlgorithmsStream.png}
\end{center}
\caption{Klasa odpowiedzialna za obsługę potoku}
\label{fig:openCLAlgorithmsStream}
\end{figure}

Za samą realizację potoku jest odpowiedzialna klasa przedstawiona na rys. \ref{fig:openCLAlgorithmsStream}. Sposób jej wykorzystania jest prosty. Najpierw należy dodać algorytmy do potoku. Podczas dodwania sprawdzane jest czy dane wejściowe do kolejnego algorytmu są zgodne ze zwracanymi wartościami z~algorytmu porpzedniego. Dotyczy to między innymi sprawdzenie typów danych oraz ich formatu. Następnie należy przygotować strumień do działania oraz podać wielkość przetwarzanych danych. Po tych operacjach można uruchomić strumień podając skąd pobierać i~gdzie zapisywać dane. Jeden strumień może być uruchamiany dowolną liczbę razy.

\subsection{Reguły dotyczące kerneli}
\label{subsec:regulykerneli}

Każdy kernel, który ma być użyty z~zastosowaniem biblioteki musi spełniać klika warunków. Najważniejszą jest kolejność argumentów. Zawsze pierwszym argumentem są dane wejściowe a drugim struktura, do której są zapisywane wyniki. Obecnie wspierane są następujące typy argumentów:
\begin{itemize}
\item wektor - \texttt{float*} lub \texttt{int*},
\item obraz 2D - \texttt{image2d_t},
\item obraz 3D (tylko jako argument wejściowy) - \texttt{image2d_t}.
\end{itemize}
Kolejnymi argumentami mogą być inne dane. Mogą to być parametry pomocnicze, jeśli są konieczne do wykonania obliczeń lub wksaźniki do miejsc, gdzie można przechować kolejne wyniki, jeśli rezultatem danego algorytmu jest więcej niż jeden obraz lub wektor.

\subsection{Obsługa błędów}
\label{subsec:obslugabledow}


\begin{figure}
\begin{center}
%v[width=3cm]
\includegraphics[width=0.7\textwidth]{Diagrams/OpenCLExceptions.png}
\end{center}
\caption{Diagram wyjątków}
\label{fig:diagramwyjatkow}
\end{figure}

Biblioteka przekazuje błędy występujące podczas wywoływania API OpenCL do użytkownika za pomocą wyjątków, które są usystematyzowane. Hierarchia wyjątków jest przedstawiona na rysunku \ref{fig:diagramwyjatkow}. Większość zwracanych błędów występuje przy wywołaniach funkcji OpenCL. Z~tego powodu do opisu błędu z~biblioteki, dodawana jest informacja o~kodzie błędu, który wystąpił podczas wywołania funkcji standardu. Zwracany kod jest w~postaci zgodnej z dokumentacją, dlatego pozwala na szybkie rozpoznanie przyczyny błędu.

\subsection{Zaimplementowane kernele}
\label{subsec:kernele}


\begin{figure}
\begin{center}
%v[width=3cm]
\includegraphics[width=\textwidth]{Diagrams/AllAlgorithms.png}
\end{center}
\caption{Diagram wszystkich klas algorytmów}
\label{fig:diagramWszystko}
\end{figure}

W~ramach stworzonej biblioteki zaimplementowano algorytmy wykorzystywane w pracy. Są to: obliczanie konwolucji obrazu z maską oraz algorytmy służące detekcji i~wyznaczaniu cech przedstawione w rozdziale \ref{subsubsec:rozpoznawaniePlam}. Na rys. \ref{fig:diagramWszystko} przedstawiono wszystkie klasy, które kontrolują zaimplementowane kernele.

Na listingu \ref{lis:edgeDetector} przedstawiono kernel liczący wartości warunków \eqref{eq:edgeDetection}, które są wykorzystywane do wykrywania krawędzi. 

\lstset{language=C,caption={Detektor krawędzi},label=lis:edgeDetector,breaklines=true,numbers=left}
\begin{lstlisting}
const sampler_t sampler = CLK_NORMALIZED_COORDS_FALSE | CLK_ADDRESS_CLAMP_TO_EDGE | CLK_FILTER_NEAREST;

__kernel void  edge_detector(__read_only image2d_t input, __write_only image2d_t out_Lvv, __write_only image2d_t out_Lvvv)
{
	int i = get_global_id(0); //column number
	int j = get_global_id(1); //row number

	float ul = read_imagef(input, sampler, (int2)(i - 1, j - 1)).x;
	float u = read_imagef(input, sampler, (int2)(i, j - 1)).x;
	float ur = read_imagef(input, sampler, (int2)(i + 1, j - 1)).x;
	float l = read_imagef(input, sampler, (int2)(i - 1, j)).x;
	float c = read_imagef(input, sampler, (int2)(i, j)).x;
	float r = read_imagef(input, sampler, (int2)(i + 1, j)).x;
	float dl = read_imagef(input, sampler, (int2)(i - 1, j + 1)).x;
	float d = read_imagef(input, sampler, (int2)(i, j + 1)).x;
	float dr = read_imagef(input, sampler, (int2)(i + 1, j + 1)).x;

	float Lx =  0.5 * (ur + 2.0 * r + dr - ul - 2.0 * l - dl);
	float Ly =  0.5 * (dl + 2.0 * d + dr - ul - 2.0 * u - ur);
	float Lxx =  ul / 12.0 - u / 6.0 + ur / 12.0 +
                5.0 * l / 6.0 - 5.0 * c / 3.0 + 5.0 * r / 6.0 +
				dl / 12.0 - d / 6.0 + dr / 12.0;//*/
	float Lxy =  ul / 4.0 - ur / 4.0 +
				-dl / 4.0 + dr / 4.0;
	float Lyy =  ul / 12.0 + 5.0 * u / 6.0 + ur / 12.0 -
                l / 6.0 - 5.0 * c / 3.0 - r / 6.0 +
				dl / 12.0 +  5.0 * d / 6.0 + dr / 12.0;
	float Lxxx =  0.5 * (ul + dl - ur - dr) + l - r;
	float Lxxy =  - ul / 8.0 + u / 4.0 - ur / 8.0
					+ dl / 8.0 - d / 4.0 + dr / 8.0;
	float Lxyy =  ul / 8.0 - ur / 8.0
					- l / 4.0 + r / 4.0
					+ dl / 8.0 - dr / 8.0;
	float Lyyy =  0.5 * (ul + ur - dr - dl) + u - d;
	
	float Lvv = Lx * Lx * Lxx + 2.0 * Lx * Ly * Lxy + Ly * Ly * Lyy;
	if (fabs(Lvv) < 1e-5)
	{
		Lvv = 0;
	}
	float Lvvv = (Lx * Lx * Lx * Lxxx + 3.0 * Lx * Lx * Ly * Lxxy + 3.0 * Lx * Ly * Ly * Lxyy + Ly * Ly * Ly * Lyyy);
	write_imagef(out_Lvv, (int2)(i, j), Lvv);
	write_imagef(out_Lvvv, (int2)(i, j), Lvvv);
}
\end{lstlisting}

W~pierwszej linii pojawia się deklaracja i definicja samplera - obiektu, który jest używany do pobierania wartości. We wszystkich kernelach wykorzystywany jest sampler, który nie korzysta ze znormalizowanych wartości. W~przypadku pobierania wartości pikseli spoza obrazu, które są czytane podaczas analizy pikseli brzegowych używane pobierany jest piksel tła. Ponieważ w~kernelach nie są wykorzystywane wartości pomiędzy pikselami ostatni parametr nie jest istotny.

W linii 3, następuje zaczyna się definicja kernela. Posiada on trzy parametry, które są typu \texttt{image2d_t}. Pierwszym jest obraz wejściowy i~dlatego jest oznaczony jako tylko do odczytu. Pozostałe dwa są obrazami, w~których będą zapisane wartości pikseli wyliczone z~warunków. Są one oznaczone jako zmienne tylko do zapisu.

W~liniach 5~i~6 pobierane są współrzędne przetwarzanego piksela. W~kolejnych liniach (8-16) pobierane są wartości pikseli. Jest istotne, aby zapisać wartości pikseli do zmiennej, ponieważ używane są wielokrotnie. Jeśli podczas każdego odwołania się do nich pobierano wartości bezpośrednio z~obrazu, to zmniejszyło by to szybkość działania kernela.

W~liniach 18-34 obliczane są wartości pochodnych dla danego piksela. Następne te wartości są wykorzystywane do  obliczenia warunków rozpozawania krawędzi.

W liniach 37-40 wartości z~pierwszego warunku są zerowane jeśli ich wartość bezwzględna jest mniejsza niż $  10^{-5} $.

Na końcu w liniach 42~i~43 obliczone wartości są zapisywane do obrazów wyjściowych.

\lstset{language=C,caption={Kernel znajdujący krawędzie},label=lis:edge,breaklines=true,numbers=left}
\begin{lstlisting}
const sampler_t sampler = CLK_NORMALIZED_COORDS_FALSE | CLK_ADDRESS_CLAMP | CLK_FILTER_NEAREST;

__kernel void  edge_max(__read_only image2d_t Lvv_image, __write_only image2d_t output, __read_only image2d_t Lvvv_image)
{
	int i = get_global_id(0); //column number
	int j = get_global_id(1); //row number

	float c = read_imagef(Lvv_image, sampler, (int2)(i, j)).x;
	float r = read_imagef(Lvv_image, sampler, (int2)(i + 1, j)).x;
	float d = read_imagef(Lvv_image, sampler, (int2)(i, j + 1)).x;
	float dr = read_imagef(Lvv_image, sampler, (int2)(i + 1, j + 1)).x;
	
	float Lvvv = read_imagef(Lvvv_image, sampler, (int2)(i, j)).x;
	if ((c * r < 0 || c * dr <0 || c * d <0) && Lvvv < 0)
	{
		write_imageui(output, (int2)(i, j), 255);
	}
}
\end{lstlisting}

Na listingu \ref{lis:edges} przedstawiono kernel sprawdzający wartości wyliczonych wcześniej warunków zgodnie z~opisem przedstawionym w~punkcie \ref{subsubsec:rozpoznawanieKrawedzi}. Wynikiem działania kernela jest obraz z~zaznaczonymi punktami, w~których znajdują się krawędzie.

W~pierwszej linii, analogicznie do poprzedniego kernela, pojawia się definicja samplera, który jest używany do pobierania wartości obrazów. Używany jest sampler o~tych samych własnościach, co poprzednio wspomniany.

W~linii 3 rozpoczyna się definicja kernela. Ma on trzy parametry, które są typu obraz 2D. Pierwsze dwa są wartościami pikseli wyliczonymi za pomocą poprzedniego kernela i~są tylko do odczytu. Ostatni argument jest obrazem, do jest zapisywana informacja o~tym, czy znaleziono w~danym miejscu krawędź.

W~liniach 5~i~6 pobierane są współrzędne analizowanego piksela. W kolejnych liniach (8-11) pobierane są wrości obrazu wejściowego. Do obliczeń potrzebne są jedynie wartości obecnie analizowanego piksela oraz jego trzech sąsiadów zgodnie z rys. \ref{fig:comparePixels}.

W~liniach 14-18 sprawdzane są warunki, jakie musi spełniać piksel, aby być uznanym za piksel krawędzi. Jest to: zmiana znaku wartości warunku pierwszego pomiędzy danym pikselem a~jednym z~jego trzech sąsiadów (oznaczonych na rys. \ref{fig:comparePixels}) oraz wartość drugiego warunku mniejsza niż zero. Jeśli warunki są spełnione to w~obrazie wyjściowym jest przypisywana wartość 255 dla tego piksela. Wszystkie piksele obrazu wyjściowego przed wywołaniem kernela mają wartość zero.

\chapter{Obsługa kamery}
\label{cha:obslugakamery}

W niniejszym rozdziale przedstawiono szczegóły techniczne wykorzystanej kamery oraz implementację obsługi kamery.

\section{Specyfikacja techniczna kamery}
\label{sec:specyfikacjaKamery}

Do testów została wykorzystano przemysłową kamerę wysokiej rozdzielczości: JAI BB-500GE. Jej maksymalna rozdzielczość to 2456 na 2058 pikseli, przy której można pobrać 15 klatek na sekundę. Możliwe jest pobranie pikseli o rodzielczości wielkości 8, 10, 12 lub 16 bitów. Dane są pobierane za pomocą standardu Gigabit Ethernet. 

Obraz pobierany za pomocą kamery jest w formacie filtru Bayera. Jest to sposób reprezentacji obrazu, który jest używane w~kamerach lub aparatach cyfrowych. Z~tego powodu konieczne było zastosowanie interpolacji obrazu barwnego. Do realizacji tego celu użyto implementacji stworzonej w \cite{BFIOCL}.

\section{Implementacja obsługi kamery}
\label{sec:implementacjaKamery}

Implementacja obsługi kamery została zrealizowana z użyciem JAI SDK \cite{JAISDK}. Stworzono klasę odpowiedzialną za obsługę kamery, ponieważ sposób pobierania obrazu nie był odpowiadający. Korzystając z JAI SDK obraz pobierany jest asynchronicznie, to znaczy po wysłaniu obrazu z~kamery wywoływana jest określona wcześniej funkcja. Jest to niezgodne z~oczekiwanym zachowaniem, ponieważ potrzeba jest, aby można było pobrać obraz w~określonej chwili.

W~klasie realizującej obsługę kamery stworzono kolejkę obrazów. Trafiają do niej ramki z~kamery w~momencie ich wysłania. W~dowolnym momencie można pobrać obraz. Jeżeli kolejka jest pusta (pobrano wszystkie obrazy oraz w~międzyczasie nie został dodany nowy) to wyrzucany jest wyjątek. Jeśli natomiast kolejka zostanie zapełniona (obrazy są dodawane szybciej niż są przetwarzanie) to nie zostanie dodana nowa ramka i~nie będzie można już jej odzyskać.

\chapter{Przedstawienie działania programu i~algorytmu}
\label{cha:dzialanie}

\begin{figure}[h]
\begin{center}

\begin{subfigure}[t]{0.4\textwidth}
\includegraphics[width=\textwidth]{Operation/input.png}
\caption{Wejściowy obraz pozyskany z~kamery}
\label{fig:input}
\end{subfigure}
~
\begin{subfigure}[t]{0.4\textwidth}
\includegraphics[width=\textwidth]{Operation/reprezentacja.png}
\caption{Fragment reprezentacji skali obrazu wejściowego}
\label{fig:dzialanieRep}
\end{subfigure}

\end{center}
\label{fig:inputIDzialanie}
\end{figure}

W~niniejszym rodziale przedstawiono przykładowe wyniki działania algorytmu. Do stworzenia przykładu użyto napisaną implementację z~zastosowaniem biblioteki OpenCL. Obrazem wejściowym jest obraz pobrany przy pomocy kamery przedstawionej w~rozdziale \ref{cha:obslugakamery}, który jest przedstawiony na rys \ref{fig:input}.

\section{Reprezentacja skali}
\label{sec:dzialanieRep}

Reprezentacja skali jest przedstawiona na rys \ref{fig:dzialanieRep}. Jest to obraz otrzymany po zastosowaniu filtru Gaussa. Filtracji podlega obraz w~skali szarości, które jest otrzymany z~wyniku barwnej aproksymacji obrazu zapisanego w~postaci filtru Bayera. Aproksymacja jest zrealizowana również na karcie graficznej z~użyciem OpenCL. Jest to również obraz wejściowy dla dalszych kroków algorytmu.

Na przedstawionym obrazie widać, że szczegóły są mało widoczne i~zlewają się w jeden obiekt. W~sposób można otrzymać większą skalę i~rozpoznawać obiekty, które są większe, pomijając szczegóły. Reprezentacja skali składa się z~kilku takich obrazów w~różnych skalach.


\section{Rozpoznawanie plam}
\label{sec:dzialanieBlob}

\begin{figure}[h]
\begin{center}

\begin{subfigure}[t]{0.4\textwidth}
\includegraphics[width=\textwidth]{Operation/blobIntermediate.png}
\caption{Przykładowy obraz pośredni używany przy rozpoznawaniu plam}
\label{fig:blobIntermediate}
\end{subfigure}
~
\begin{subfigure}[t]{0.4\textwidth}
\includegraphics[width=\textwidth]{Operation/blobResult.png}
\caption{Przykładowy wynik rozpoznawania plam}
\label{fig:blobResult}
\end{subfigure}

\end{center}
\label{fig:showBlob}
\end{figure}

Do rozpoznawania plam używany jest operator przedstawiony na rys. \ref{fig:laplacian_kernel}. Rys. \ref{fig:blobIntermediate} przedstawia wynik działania operatora Laplace'a na jednym z~obrazów reprezentacji skali. Jasne punkty oznaczają wartości bliskie zera. Im ciemniejszy piksel tym większa wartość. Za punkty środkowe plam (aproksymowane jako koła) są uznawane maksima lokalne, które są przedstawione na rys. \ref{fig:blobResult}. Jest to wynik działania algorytmu rozpoznawania plam. Ciemne piksele oznaczają środek rozpoznanych plam.

\section{Rozpoznawanie krawędzi}
\label{sec:dzialanieEdge}

\begin{figure}[h]
\begin{center}

\begin{subfigure}[t]{0.4\textwidth}
\includegraphics[width=\textwidth]{Operation/edge01Intermediate.png}
\caption{Przykładowy obraz pośredni otrzymany przy rozpoznawaniu krawędzi, wynik warunku pierwszego}
\label{fig:edgeIntermediate1}
\end{subfigure}
~
\begin{subfigure}[t]{0.4\textwidth}
\includegraphics[width=\textwidth]{Operation/edge02Intermediate.png}
\caption{Przykładowy obraz pośredni używany przy rozpoznawaniu krawędzi, wynik warunku drugiego}
\label{fig:edgeIntermediate2}
\end{subfigure}

\begin{subfigure}[t]{0.4\textwidth}
\includegraphics[width=\textwidth]{Operation/edgeResult.png}
\caption{Przykładowy wynik rozpoznawania krawędzi}
\label{fig:edgeResult}
\end{subfigure}

\end{center}
\label{fig:showEdge}
\end{figure}

Podczas detekcji krawędzi pierwszym krokiem jest wyznaczenie wartości wyrażeń przedstawionych na równaniu \eqref{eq:edgeDetection}. Rys. \ref{fig:edgeIntermediate1}~i~\ref{fig:edgeIntermediate2} przedstawiają wartości wyżej wymienionych wyrażeń. Kolor szary na tych obrazach oznacza wartości zerowe lub bliskie zeru. Punkty ciemniejsze to wartości ujemne a punkty jaśniejsze to wartości dodatnie. Można zauważyć, że w~miejscach, gdzie występują krawędzie jest zmiana znaku wartości pierwszego warunku oraz drugi warunek przyjmuje wartości ujemne. Wynik działania algorytmu rozpoznawania krawędzi jest przedstawiony na rys. \ref{fig:edgeResult}. Ciemne piksele występują w~miejscach, gdzie wykryto krawędź.

Z~przedstawionych obrazów wynika, że konieczna jest modyfikacja pierwszego warunku, aby wyszukiwać miejsca, gdzie pierwszy znak zmienia znak zamiast wartości zerowych.

\section{Rozpoznawanie narożników}
\label{sec:dzialanieCorner}

\begin{figure}[h]
\begin{center}

\begin{subfigure}[t]{0.4\textwidth}
\includegraphics[width=\textwidth]{Operation/cornerIntermediate.png}
\caption{Przykładowy obraz pośredni używany przy rozpoznawaniu narożników}
\label{fig:cornerIntermediate}
\end{subfigure}
~
\begin{subfigure}[t]{0.4\textwidth}
\includegraphics[width=\textwidth]{Operation/cornerResult.png}
\caption{Przykładowy wynik rozpoznawania narożników}
\label{fig:cornerResult}
\end{subfigure}

\end{center}
\label{fig:showEdge}
\end{figure}

Podczas detekcji krawędzi pierwszym krokiem jest wyznaczenie wartości współczynnika przedstawionego na równaniu \eqref{eq:cornerDetection}. Otrzymane wyniki dla podanego obrazu wejściowego są przedstawione na rys. \ref{fig:cornerIntermediate}. Jasne punkty oznaczają wartości bliskie zera. Im ciemniejszy piksel tym większą ma wartość. Wyszukiwane są lokalne maksima, które są przedstawione na rys \ref{fig:cornerResult}. Jest to wynik algorytmu rozpoznawania krawędzi. Ciemne piksele oznaczają miejsca, gdzie wykryto narożniki.

\section{Rozpoznawanie grani}
\label{sec:dzialanieRidge}

\begin{figure}[h]
\begin{center}

\begin{subfigure}[t]{0.4\textwidth}
\includegraphics[width=\textwidth]{Operation/ridge01Intermediate.png}
\caption{Przykładowy obraz pośredni używany przy rozpoznawaniu grani, wynik warunku pierwszego}
\label{fig:ridgeIntermediate1}
\end{subfigure}
~
\begin{subfigure}[t]{0.4\textwidth}
\includegraphics[width=\textwidth]{Operation/ridge02Intermediate.png}
\caption{Przykładowy obraz pośredni używany przy rozpoznawaniu grani, wynik warunku drugiego}
\label{fig:ridgeIntermediate2}
\end{subfigure}

\begin{subfigure}[t]{0.4\textwidth}
\includegraphics[width=\textwidth]{Operation/ridgeResult.png}
\caption{Przykładowy wynik rozpoznawania krawędzi}
\label{fig:ridgeResult}
\end{subfigure}

\end{center}
\label{fig:showEdge}
\end{figure}

Podczas detekcji krawędzi pierwszym krokiem jest wyznaczenie wartości wyrażeń przedstawionych na równaniu \eqref{eq:ridgeDetection}. Wartości wyznaczone za pomocą przedstawionych wyżej warunków są przedstawione na rys. \ref{fig:ridgeIntermediate1}~i~\ref{fig:ridgeIntermediate1}. Piksele szare oznaczają wartości bliskie zera. Ciemniejsze punkty to wartości ujemne, a jaśniejsze przedstawiają wartości dodatnie. Wynik rozpoznawania grani są przedstawione na rys. \ref{fig:ridgeResult}. Ciemne piksele znajdują się w~miejscach, gdzie wykryto granie.

\chapter{Testy}
\label{cha:testy}

W niniejszym rozdziale przedstawiono proces testowania zaimplementowanych algorytmów oraz uzyskane wyniki. Obszarem zainteresowań jest poprawność zaimplementowanego algorytmu oraz porównanie szybkości działania z~innymi implementacjami.

Do porównań opracowano dwie dodatkowe implementacje algorytmu zrealizowane z użyciem biblioteki OpenCV: jedna przeprowadza obliczenia na procesorze (CPU), druga przeprowadza obliczenia na karcie graficznej (GPU) z~wykorzystaniem pakietu CUDA.

\section{Środowisko testowe}
\label{sec:srodowiskoTesty}

Do testów użyto dwóch kart graficznych.

\begin{enumerate}
\item GeForce GT 555M \cite{GT555M} o następujących parametrach:
\begin{itemize}
\item liczba rdzeni: 144,
\item zegar układu graficznego: 675 MHz,
%\item zegar procesora: 1350 MHz,
\item częstotliwość danych pamięci: 1800 MHz,
\item interfejs pamięci: 128-bitowy,
\item szerokość pasma pamięci: 28,80GB/s,
\item dostępna pamięć: 4095 MB.
\end{itemize}

\item GeForce GTX 670 \cite{GTX670} o następujących parametrach:
\begin{itemize}
\item liczba rdzeni: 1344,
\item zegar układu graficznego: 980 MHz,
%\item zegar procesora: 1350 MHz,
\item częstotliwość danych pamięci: 6008 MHz,
\item interfejs pamięci: 256-bitowy,
\item szerokość pasma pamięci: 192.26 GB/s,
\item dostępna pamięć: 4096 MB.
\end{itemize}
\end{enumerate}

Implementacja wykonywana na CPU była uruchamiana na procesorze Intel Core i7-2670QM pracującym z~częstotliwością 2200 MHz.

W~niniejszym rodziale użyto następujących skrótów:
\begin{itemize}
\item CL, OpenCL - implementacja wykonana z~użyciem OpenCL,
\item CVCPU, OpenCVCPU - implementacja wykonana z~użyciem OpenCV wykonywania na procesorze CPU,
\item CVGPU, OpenCVGPU - implementacja wykonana z~użyciem OpenCV wykonywania na procesorze graficznym GPU,
\item CL-CVCPU - oznacza porównanie pomiędzy wynikiem implementacji wykonanej w~OpenCL a implementacją wykonaną za pomocą OpenCV na procesorze,
\item CL-CVCPU - oznacza porównanie pomiędzy wynikiem implementacji wykonanej w~OpenCL a implementacją wykonaną za pomocą OpenCV na karcie graficznej,
\item CL-CVCPU - oznacza porównanie pomiędzy wynikiem implementacji wykonanej za pomocą OpenCV na procesorze a implementacją wykonaną za pomocą OpenCV na karcie graficznej.
\end{itemize}

%\include{realizacja}
%\include{czesc_trzecia}
%\include{technologie}
%\include{wnioski}
%\chapter{Wnioski i podsumowanie}
\label{cha:podsumowanie}

Biorąc pod uwagę uzyskane wyniki testów udało się zrealizować cel pracy, jakim było stworzenie implementacji algorytmu Scale Space z~zastosowaniem biblioteki OpenCL. Testy poprawności pokazały, że zrealizowana implementacja jest zgodna (z~dokładnością do niewielkich różnic spowodowanych błędami w~obliczeniach numerycznych). Testy wydajnościowe wykazały, że implementacja wykorzystująca OpenCL jest szybsza niż realizacja algorytmu na procesorze ogólnego przeznaczenia CPU. Porównano również szybkość wykonywania obliczeń zrealizowanej biblioteki do szybkości działania rozwiązań zrealizowanych przy użyciu biblioteki OpenCV. Stworzona implementacja wykonuje obliczenia szybciej niż rozwiązanie opearte o~bibliotekę OpenCV wykorzystujące kartę graficzną. Uzyskane przyśpieszenie jest zależne od wielkości skali i~wynosi, pomijając najmniejszą skalę, od 2 do 12,5 przy błędzie mniejszym niż $ 0,8\% $. Dla obrazów o~rozdzielczości 512x512 pikseli oraz dla 4 skal osiągnięto działanie programu w~czasie rzeczywistym.

Można zauważyć, że poszczególne detektory różnią się od siebie. Wydaje się, że najlepiej działającym jest detektor narożników. W~przypadku kontynuacji prac warto byłoby przeanalizować i~udoskonalić działanie detektora grani.

Dalszymi krokami usprawniającymi przedstawione rozwiązanie może być umożliwienie wykorzystanie większej liczby kart graficznych, jeśli są dostępne w~danym systemie. Użycie większej ilości kart graficznym mogłoby umożliwić przetwarzanie obrazu wejściowego w~czasie rzeczywistym dla obrazów o~większej rozdzielczości.

Drugą możliwym kierunkiem rozwoju zrealizowanego systemu może być wykorzystanie otrzymanych informacji. Obecnie system realizuje detekcję krawędzi, narożników, plam oraz grani. Otrzymane informacje mogą być wykorzystane w~kolejnych krokach bardziej złożonego algorytmu, pozwalającego na analizę i~rozpoznawanie obiektów. Dzięki wstępnemu przetworzeniu, które jest realizowane w~tej pracy można do analizy obrazu wykorzystać mniejszą ilość danych. Tymi danymi będę miejsca (piksele), w~których znaleziono poszczególne strukury.


%\include{rozdzial2}


\bibliographystyle{plain}
\bibliography{praca}{}

%\appendix
%\include{zawartoscCD}
%\include{opiskodow}

\end{document}
c