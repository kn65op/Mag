\chapter{Krótka dokumentacja kodu}
\label{cha:dokumentacja}

Stworzony program ma budowę modułową i~składa się z~przedstawionych poniżej części.

\begin{itemize}
\item OpenCLInterface - w~tym module znajduje się implementacja wykorzystania obsługi OpenCL. Jest ona przedstawiona w~rozdziale \ref{cha:opencl}.
\item ScaleSpace - moduł, w~którym są zrealizowane wszystkie trzy implementacje algorytmu ScaleSpace. Główną cześcią tego modułu są klasy implementujące algorym. Do tworzenia tych klasy została zrealizowana fabryka. W~skład modułu wchodzi również klasa, która przechowuje wszystkie obrazy używane podczas realizacji algorytmu, wraz z~danymi wejściowymi i~wyjściowymi.
\item Main - jest to moduł, który jest punktem startowym aplikacji oraz kontroluje jej wykonanie. Przetwarza również parametry weściowe zgodnie z~ich formatem przedstawiony w~dodatku \ref{cha:instrukcja}.
\item TTime - ten moduł jest odpowiedzialny za liczenie czasu wykonania. Składa się z~jednej klasy, której zadaniem jest implementacja funkcji stopera.
\item JAICameraInterface - moduł odpowiedzialny za obsługę kamery. Osługa jest zgodna z~opisem przedstawionym w~rodziale \ref{cha:obslugakamery}.
\end{itemize}
