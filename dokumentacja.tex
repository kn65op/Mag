\chapter{Lista modułów systemu}
\label{cha:dokumentacja}

Program komputerowy ma budowę modułową i~składa się z~przedstawionych poniżej modułów:
\begin{itemize}
\item \texttt{OpenCLInterface} - w~tym module znajdują się funkcje obsługujące procedury OpenCL. Jest on przedstawiony w~rozdziale \ref{cha:opencl}.
\item \texttt{ScaleSpace} - moduł, w~którym są zrealizowane wszystkie trzy implementacje algorytmu ScaleSpace. Główną częścią tego modułu są klas implementujące algorytm. Do tworzenia tych klasy została zrealizowana fabryka. W~skład modułu wchodzi również klasa, która przechowuje wszystkie obrazy używane podczas realizacji algorytmu, wraz z~danymi wejściowymi i~wyjściowymi.
\item \texttt{Main} - jest to moduł, który jest punktem startowym aplikacji oraz kontroluje jej wykonanie. Przetwarza również parametry wejściowe zgodnie z~ich formatem przedstawiony w~dodatku \ref{cha:instrukcja}.
\item \texttt{TTime} - ten moduł jest odpowiedzialny za pomiar czasu wykonania. Składa się z~jednej klasy, której zadaniem jest implementacja funkcji stopera.
\item \texttt{JAICameraInterface} - moduł odpowiedzialny za obsługę kamery. Obsługa jest zgodna z~opisem przedstawionym w~rodziale \ref{cha:obslugakamery}.
\end{itemize}
