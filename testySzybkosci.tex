\section{Testowanie szybkości działania}
\label{sec:testSzybkosc1}

Testowanie szybkości działania oparto na porównaniu szybkości działania wszystkich implementacji algorytmu Scale Space. Liczony był czas działania algorytmu. Im krótszy czas tym lepsza implementacja algorytmu Scale Space.

Mierzeniu podlagał czas trwania następujących czynności:
\begin{itemize}
\item czas wyzaczania reprezentacji Scale Space,
\item czas wkrywania cech na obrazach (wraz z~wcześniejszym wyznaczeniem reprezentacji Scale Space) - osobno dla każdej cechy.
\end{itemize}

Do testów użyto różnych kart graficznych.
\begin{enumerate}
\item GeForce GT 555M \cite{GT555M} o następujących parametrach:
\begin{itemize}
\item libcza rdzeni: 144,
\item zegar układu graficznego: 675 MHz,
\item zegar procesora: 1350 MHz,
\item częstotliwość danych pamięci: 1800 MHz,
\item interfejs pamięci: 128-bitowy,
\item szerokość pasma pamięci: 28,80GB/s,
\item dostępna pamięć: 4095 MB.
\end{itemize}
\item druga karta z laboratorium.
\end{enumerate}

%opis maszyn, na których były wykonane testy


