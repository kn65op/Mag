\chapter{OpenCL}
\label{cha:opencl}

W~tym rodziale przedstawiono sposób implementacji algorytmów z~użyciem standardu OpenCL oraz stworzoną bibliotekę służącą jej ułatwieniu. Zdecydowano się na jej stworzenie z~uwagi na to, że w cześci programu, która kontroluje wykonanie kodu na karcie graficznej (kerneli) wiele czynności się powtarza przy każdym kolejnym algorytmie. Zatem zadaniem biblioteki jest ułatwienie implementacji kontrolera nie ograniczając możliwości oferowanych przez standard OpenCL.

\section{Szczegóły implementacji z użyciem OpenCL}  
\label{sec:szczegolyOpenCL}

