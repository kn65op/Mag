\chapter{OpenCL}
\label{cha:opencl}

W~tym rodziale przedstawiono sposób implementacji algorytmów z~użyciem standardu OpenCL oraz stworzoną bibliotekę służącą jej ułatwieniu. Zdecydowano się na jej stworzenie z~uwagi na to, że w cześci programu, która kontroluje wykonanie kodu na karcie graficznej (kerneli) wiele czynności się powtarza przy każdym kolejnym algorytmie. Zatem zadaniem biblioteki jest ułatwienie implementacji kontrolera nie ograniczając możliwości oferowanych przez standard OpenCL oraz umożliwienie łatwiego rozszerzania dostępnych funkji.

\section{Szczegóły implementacji z użyciem OpenCL}  
\label{sec:szczegolyOpenCL}

Realizacja algorytmów z~użyciem standardu OpenCL składa się z~dwóch cześci. Pierwszą jest kod algorytmu zapisany w~języku stworzonym w ramach standardu OpenCL, drugą jest kod wykonywany na procesorze ogólnego przeznaczania wykorzystujący API, które jest zdefiniowane w języku C.

Algorytmy zapisywane są w~postaci funkcji, które są nazywane kernelami. Są one zapisywane w~języku stworzonym przez twórców standardu. Bazuje on na języku ISO C99 i~zawiera jego podzbiór wraz z~rozszerzeniami służacymi implementacji algorytmów w sposób równoległy, ułatwiającymi obsługę wektorów oraz macierzy. 

Kod kontrolera można podzielić na dwa etapy. W~pierwszym, jest to inicjalizacja, jest dokonywana kompilacja kerneli. W~drugim, podczas realizacji algorytmu jest to przekazywanie parametrów, wykonywanie kerneli oraz pobieranie wyników. Te czynności są realizwane przy każdym nowy algorytmie.

\section{Zrealizowana biblioteka}  
\label{sec:biblioteka}

Biblioteka zastała zaimplementowana w języku C++. Wybrano ten język ponieważ jego integracja z~API OpenCL napisanym w języku C jest ?prosta?. Biblioteka została zrealizowana w~sposób obiektowy oraz z~użyciem wyjątków ułatwiających pracę. Więcej szczegółów dotyczącyh wyjątków jest przedstawiona w~sekcji \ref{subsec:obslugabledow}. Obietkowość zapewnia maksymalne ograniczenie redundancji napisanego kodu odpowiedzialnego za kompilację, przekazywanie argumentów, wykonywanie kernel oraz pobieranie wyników. Aby to umożliwić należy opracować zbiór reguł, których należy przestrzegać podczas pisania kerneli. Są one opisane w sekcji \ref{subsec:regulykerneli}. Reguły te nie ograniczają możliwości oferowanych przez standard. Biblioteka oferuje również możliwość realizacji algorytów w sposób potokowy. Taka realizacja ma zastosowanie wtedy, kiedy zachodzi potrzeba użycia kilku kerneli, dla których dane wejściowe kolejnego programu są danymi wyjściowymi poprzedniego. Implementacja  ogranicza liczbę operacji kopiowania danych pomiędzy pamięcią RAM komputera a~pamięcią wewnętrzną urządzenia, na którym wykonywany jest program poprzez zachowywanie cząstkowych obliczeń w~pamięci urządzenia?.

Bibliotekę można podzielić na dwie części: część odpowiedzialną za kompilację kerneli oraz wszelkie powiązane z~tym kwestie oraz część odpowiedzialną za wykonywanie kerneli.
Ponieżej są przedstawione możliwości oraz szczegóły techniczne poszczególnych cześci.

\subsection{Kompilacja kerneli oraz obsługa urządzeń}
\label{subsec:kompilacjakerneli}
W~skład tej części wchodzi jedna klasa. Jest to klasa \lstinline{OpenCLDevice}. Jest ona odpowiedzialna za pobieranie lisy dostępnych urządzeń, pobieranie podstawowych informacji o urządzeniu, tworzenie i~udostępnianie kontekstu oraz kolejki komend (fragmentu standardu OpenCL) oraz za kompilację kernela na dane urządzenie. 

\subsection{Wykonywanie kerneli}
\label{subsec:wykonywaniekerneli}
Zadaniem tej części jest ułatwienie użytkownikowi procesu kompilacji oraz wykonywania kerneli wraz z~przekazywaniem argumentów oraz pobieraniem wyników. Ten fragment biblioteki składa się z~wielu klas. Jest to spowodowane chęcią umożliwienia zastosowania biblioteki do wielu rodzajów kerneli. Konieczność wprowadzenia dużego rozróżnienia wynika z~istnienia dużej liczby struktur wykorzystywanych w~standardzie do przekazywania parametrów wejściowych i wyjściowych.



\subsection{Tworzenie potoków kerneli}
\label{subsec:potokikerneli}



\subsection{Reguły dotyczące kerneli}
\label{subsec:regulykerneli}



\subsection{Obsługa błędów}
\label{subsec:obslugabledow}

Biblioteka przekazuje błędy występujące podczas wyoływania API OpenCL do użytkownika za pomocą wyjątków, które są usystematyzowane. Hierarchia wyjątków jest przedstawiona na rysunku \ref{fig:diagramwyjatkow}.

\subsection{Zaimplementowane algorytmy}
\label{subsec:algorytmy}
