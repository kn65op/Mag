%---------------------------------------------------------------------------

\begin{frame}
\frametitle{Plan prezentacji}

\begin{enumerate}
\item Algorytm Scale Space
\item Standard OpenCL
\item Wyniki
\end{enumerate}

\end{frame}

%---------------------------------------------------------------------------

%Jednym z istotniejszych problemów podczas automatycznego przetwarzania i analizy obrazów jest wybór skali.

%Obrazy mogą być rejestrowane w~różnych skalach.

\begin{frame}
\frametitle{Algorytm Scale Space}
\begin{block}{Scale Space}
Algorytm służący do przedstawiania sygnałów w reprezentacji skali.
\end{block}
Można go zastosować do dowolnych sygnałów. Praca skupia się na wykorzystaniu algorytmu do przetwarzania obrazów.

\end{frame}

%---------------------------------------------------------------------------
\begin{frame}
\frametitle{Podstawy matematyczne}
Do wyznaczenia przestrzeni skali stosuje się filtry Gaussa. Poniżej przedstawiono sposób wyznaczania współczynników filtra w przestrzeni dwuwymiarowej.

\begin{center}
\includegraphics[width=3cm]{gaussian2d.png}

$ g(x,y,\sigma)= \frac{1}{2 \cdot \pi \cdot \sigma ^ {2} }\cdot e^{(-\frac{x^{2} + y^{2}}{2 \cdot \sigma ^{2}})} $
\end{center}
\end{frame}
%---------------------------------------------------------------------------
\begin{frame}
\frametitle{Algorytm Scale Space dla obrazów}


\begin{figure}[h]
\begin{center}
\begin{subfigure}[b]{2cm}
                 \centering
                 \includegraphics[width=2cm]{Lena_scales1.jpg}
         \end{subfigure}%
		 ~
\begin{subfigure}[b]{2cm}
                 \centering
                 \includegraphics[width=2cm]{Lena_scales2.jpg}
         \end{subfigure}%
		 ~
\begin{subfigure}[b]{2cm}
                 \centering
                 \includegraphics[width=2cm]{Lena_scales3.jpg}
         \end{subfigure}%
		 
		 
\begin{subfigure}[b]{2cm}
                 \centering
                 \includegraphics[width=2cm]{Lena_scales4.jpg}
         \end{subfigure}%
		 ~
		 \begin{subfigure}[b]{2cm}
                 \centering
                 \includegraphics[width=2cm]{Lena_scales5.jpg}
         \end{subfigure}%
		 ~
\begin{subfigure}[b]{2cm}
                 \centering
                 \includegraphics[width=2cm]{Lena_scales6.jpg}
         \end{subfigure}%
\caption{Obraz w przestrzeni skali}
\label{lena_scales}
\end{center}
\end{figure}
\end{frame}


%---------------------------------------------------------------------------
\begin{frame}
\frametitle{Przetwarzanie obrazów w przestrzeni skali}

Po wyznaczeniu reprezentacji skali dla obrazu można przeprowadzić dalsze operacje. W pracy zrealizowano:
\begin{itemize}
\item detekcję plam,
\item detekcję krawędzi,
\item detekcję narożników,
\item detekcję grani.
\end{itemize}
Dzięki temu, że obraz jest przetwarzany w~różnych skalach, można rozpoznawać dane struktury dla różnych stopni szczegółowości.
\end{frame}

%---------------------------------------------------------------------------
\begin{frame}
\frametitle{OpenCL}
\begin{block}{OpenCL}
Otwarty, wieloplatformowy standard pozwalający na wykorzystanie procesorów kart graficznych oraz innych urządzeń w celu wykonywania obliczeń ogólnego przeznaczenia. Wykorzystanie karty graficznej pozwala na zrównoleglenie obliczeń.
\end{block}

\end{frame}


%---------------------------------------------------------------------------
\begin{frame}
\frametitle{Wykorzystanie OpenCL do zrównoleglenia obliczeń}

Do implementacji algorytmu z~wykorzystaniem OpenCL konieczne jest:
\begin{itemize}
\item Napisanie kodu kerneli - fragmentów kodu wykonywanych na karcie graficznej.
\item Napisanie kodu wykonywanego na procesorze, który będzie kontrolował część wykonywaną na karcie graficznej. W tym zawiera się przekazanie parametrów do kerneli oraz pobranie wyników.
\end{itemize}

\end{frame}


%---------------------------------------------------------------------------
\begin{frame}
\frametitle{Detekcja struktur}

Po wyznaczeniu reprezentacji skali obrazu wykonywane są określone operacje matematyczne w~celu detekcji określonych struktur.


\end{frame}
%---------------------------------------------------------------------------
\begin{frame}
\frametitle{Przykład detekcji}

\begin{figure}[h]
\begin{center}
\begin{subfigure}[b]{2cm}
\centering
\includegraphics[width=2cm]{../Operation/blobResult.png}
\end{subfigure}~
\begin{subfigure}[b]{2cm}
\centering
\includegraphics[width=2cm]{../Operation/edgeResult.png}
\end{subfigure}

\begin{subfigure}[b]{2cm}
\centering
\includegraphics[width=2cm]{../Operation/cornerResult.png}
\end{subfigure}~
\begin{subfigure}[b]{2cm}
\centering
\includegraphics[width=2cm]{../Operation/ridgeResult.png}
\end{subfigure}
\caption{Wynik przetworzenia obrazów w reprezentacji skali operatorem Laplace'a}
\label{fig:wynik}
\end{center}
\end{figure}

\end{frame}
%---------------------------------------------------------------------------
\begin{frame}
\frametitle{Poprawność rozwiązania}

Na podstawie porównania z implementacją realizowaną na procesorze CPU stwierdzono, że wykonanie algorytmu na karcie graficznej daje poprawne wyniki.

Największe rozbieżności są na poziomie 2\%, w~większości przypadków nie przekraczają 1\%.


\end{frame}
%---------------------------------------------------------------------------
\begin{frame}
\frametitle{Szybkość rozwiązania}

Z pomiarów czasu obliczeń można zauważyć, że obliczenia na karcie graficznej Nvidia GeForce GTX 670.
Wykonaniu algorymtu na karcie graficznej okazało się znacznie szybsze.
\begin{center}
\includegraphics[width=7cm]{../TestySzybkosci/PureAGH.png}

\end{center}



\end{frame}
%---------------------------------------------------------------------------

\begin{frame}
\frametitle{}
\begin{center}
Dziękuję za uwagę.
\end{center}
\end{frame}
