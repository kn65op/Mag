\chapter{Przdstawienie działania programu i~algorytmu}
\label{cha:dzialanie}

W~niniejszym rodziale przedsawiono przykład działania algorytmu. Do stworzenia przykładu użyto napisaną implementację z~zastosowaniem biblioteki OpenCL. Obrazem wejściowym jest obraz pobrany przy pomocy kamery przedstawionej w~rozdziale \ref{cha:kamera}, który jest przedstawiony na rys \ref{fig:input}.

%\begin{figure}
%\begin{center}
%\includegraphics[width=3cm]{Operation/input.png}
%\end{center}
%\caption{Wejściowy obraz użyty do przedstawienia działania algorytmu}
%\label{fig:input}
%\end{figure}

\section{Reprezentacja skali}
\label{sec:dzialanieRep}

Reprezentacja skali jest przedstawiona na rys \ref{fig:dzialanieRep}. Jest to obraz otrzymany po zastosowaniu filtru Gaussa. Filtracji podlega obraz w~skali szarości, które jest otrzymany z~wyniku barwnej aproksymacji obrazu zapisanego w~postaci filtru Bayera. Aproksymacja jest zrealizowana również na karcie graficznej z~użyciem OpenCL. Jest to również obraz wejściowy dla dalszych kroków algorytmu.

%\begin{figure}
%\begin{center}
%\includegraphics[width=3cm]{Operation/reprezentacja.png}
%\end{center}
%\caption{Fragment reprezentacji skali obrazu wejściowego}
%\label{fig:input}
%\end{figure}

\section{Rozpoznawanie plam}
\label{sec:dzialanieBlob}

Do rozpoznawnia plam używany jest operator przedstawiony na rys. \ref{fig:laplacian_kernel}. Wynikiem działania podanego operatora jest obraz przedstawiony na rys. \ref{blobLaplacian}. Na obrazie są przedstawione otzymane wartości

\section{Rozpoznawanie krawędzi}
\label{sec:dzialanieEdge}

Podczas detekcji krawędzi pierwszym krokiem jest wyznaczenie wartości wyrażeń przedstawionych na równaniu \ref{eq:edgeDetection}.

\section{Rozpoznawanie narożników}
\label{sec:dzialanieCorner}

Podczas detekcji krawędzi pierwszym krokiem jest wyznaczenie wartości współczynnika przedstawionego na równaniu \ref{eq:cornerDetection}. Otrzymane wyniki dla podanego obrazu wejściowego są przedstawione na rys. \ref{cornerK}.

\section{Rozpoznawanie grani}
\label{sec:dzialanieRidge}

Podczas detekcji krawędzi pierwszym krokiem jest wyznaczenie wartości wyrażeń przedstawionych na równaniu \ref{eq:ridgeDetection}. 
