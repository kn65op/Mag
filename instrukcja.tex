\chapter{Instrukcja}
\label{cha:instrukcja}

Krótka instrukcja użytkownia stworzonego programu.

Program jest uruchamiany w~konsoli i~nie posiada żadnego graficznego interfejsu użytkowania. Do programu dołączona jest pomoc (można ją wyświelić używają parametru -h lub --help). Parametry podawane są w~sposób zgodne ze standardem znanym z~projektu GNU.

Program przyjmuje następujące parametry, które zostały opisane w~poniższym spisie. Każdy parametr może być podany w formie krótkiej lub długiej.

\begin{itemize}
\item -i,--in - obraz jest odczytywany z pliku podanego jako argument.
\item -I,--in-prefix - obrazy wejściowe są odczytywane z~podanej lokalizacji. Format nazw plików wejściowy musi być w~formacie prefix\_XX.YYY. W~prefiksie może być podany katalog. YYY jest to rozszerzenie pliku. Wspierane są pliki w~formatach: bmp, jpg lub png.
\item -o,--out - określa prefiks nazwy plików wujściowych. Pliki wyjściowe są zapisywane w formacie: A\_B\_XX\_YY.bmp, gdzie:
	\begin{itemize}
		\item A - podany prefiks,
		\item B - nazwa implementacji, która była wykorzystana,
		\item XX - numer przetwarzanego obrazu (zawsze dwucyforwy),
		\item YY - numer przetwarzanej skali (zawsze dwucyfrowy).
	\end{itemize}
\item -m,--mode - wybór detekcji cech. Dozwolone wartości:
	\begin{itemize}
		\item blob - plamy,
		\item edge - krawędzie,
		\item ridge - granie,
		\item corner - narożniki.
	\end{itemize}
	Jeśli parametr nie jest ustalony, to wyznaczana jest tylko reprezentacja skali.
\item -t,--type - format pliku wejściowego. Dostępne wartości to: bayer - oznaczający odczytywanie pliku zapisanego za pomocą filtru Bayera lub gray - oznaczający odczytywanie pliku nie zapisanego za pomocą filtru Bayera.
\item -p,--processor - wybór implementacji. Dozwolone wartości:
	\begin{itemize}
		\item cl, opencl - implementacja z~użyciem OpenCL,
        \item cv, opencv, cpu - implementacja z~użyciem OpenCV wykonywana na procesorze,
        \item cv\_gpu, opencv\_gpu, gpu - implementacja z~użyciem OpenCV wykonywana na karcie graficznej.
	\end{itemize}
\item -s,--scale - określenie ilości skal oraz kroku. Konieczne jest podanie dwóch parametrów w~podanej kolejności: kroku i~liczby skal.
\item --no-show - ustalenie, że żadne obrazy wyjściowe nie będą zapisane na dysku. Ta opcja jest przydatna w~przypadku przeprowadzania testów szybkościowych.
\item -d,--debug - dodatkowe zapisywanie obrazów pośrednich na dysku.
\item -q,--quiet - nie jest wyświetlane nic w~konsoli, jeśli nie ma błędów.
\item --no-first-image - pozwala na rozpoczęcie mierzenia czasu po przetworzeniu pierwszego obrazu. Ta opcja jest przydatna w~przypadku przeprowadzania testów szybkościowych.
\end{itemize}