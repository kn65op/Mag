\chapter{Instrukcja}
\label{cha:instrukcja}

Kr�tka instrukcja u�ytkownia stworzonego programu.

Program jest uruchamiany w~konsoli i~nie posiada �adnego graficznego interfejsu u�ytkowania. Do programu do��czona jest pomoc. Parametry podawane s� w~spos�b zgodne ze standardem znanym z~projektu GNU.

Program przyjmuje nast�puj�ce parametry, kt�re zosta�y opisane w~poni�szym spisie. Ka�dy parametr mo�e by� podany w formie kr�tkiej lub d�ugiej.

\begin{itemize}
\item -i,--in - input file instead of use camera. Default input file type is gray.
\item -I,--in-prefix -input files wille be in format: prefix_XX.bmp. It ends when files ends (i. e. no next file). Default input files type is bayer.
\item -o,--out - output files prefix. Output file will be PREFIX_PROCESSOR_IMAGENUMBER_SCALENUMBER.bmp.
\item -m,--mode - Scale Space mode. One of: blob, corner, edge, ridge.
\item -t,--type - Input file type. Can be bayer or gray. It works only with file.
\item -p,--processor - Select implementation. It can be:
	\begin{itemize}
		\item cl, opencl - for OpenCL implementation.
        \item cv, opencv, cpu - for OpenCV implementation.
        \item cv_gpu, opencv_gpu, gpu - for OpenCV GPU implementation.
	\end{itemize}
\item -s,--scale - Set scales in format a b, where a is scale step and b is number of scales.
\item --no-show - Don't save images on drive.
\item -d,--debug - Show intermediate images. It can be very slow.
\item -q,--quiet - Don't show any output.
\item --no-first-image - Don't measure time for first image.
\end{itemize}