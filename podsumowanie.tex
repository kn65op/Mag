\chapter{Wnioski i podsumowanie}
\label{cha:podsumowanie}

Biorąc pod uwagę wyniki testów udało się zrealizować cel pracy, jakim było stworzenie implementacji algorytmu Scale Space z~zastosowaniem biblioteki OpenCL. Testy poprawności pokazały, że zrealizowana implementacja jest zgodna (z~dokładnością do niewielkich różnic spowodowanych błędami w~obliczeniach numerycznych). Testy wydajnościowe wykazały, że implementacja wykorzystująca OpenCL jest szybsza niż realizacja algorytmu na procesorze ogólnego przeznaczenia CPU. Porównano również szybkość wykonywania obliczeń zrealizowanej implementacji z~rozwiązaniem stworzonym z~wykorzystaniem obliczeń na karcie graficznej, które są dostępne w~bibliotece OpenCV. Stworzona implementacja również przeprowadza obliczenia szybciej niż rozwiązanie wykorzystujące bibliotekę OpenCV do obliczeń na karcie graficznej.

Można zauważyć, że jakość poszczególnych detektorów jest zmienna. Wydaje się, że najlepiej działającym jest detektor narożników, a~najgorzej działa detektor grani. Z~tego powodu w~przypadku kontynuacji prac warto byłoby przeanalizować działanie detektora grani, a następnie albo jego poprawa, albo stworzenie nowego.

Dalszymi krokami usprawniającymi przedstawione rozwiązanie może być umożliwienie wykorzystanie większej liczby kart graficznych, jeśli są dostępne w~danym systemie. Obecnie może być wykorzystywana tylko jedna karta graficzna. Użycie większej ilości kart graficznym mogłoby umożliwić przetwarzanie obrazu wejściowego w~czasie rzeczywistym dla obrazów o~większej rozdzielczości.

Drugą możliwym kierunkiem rozwoju zrealizowanego systemu może być wykorzystanie otrzymanych informacji. Obecnie system realizuje detekcję krawędzi, narożników, plam oraz grani. Otrzymane informacje mogą być wykorzystane w~kolejnych krokach bardziej złożonego algorytmu, pozwalającego na analizę całego obrazu. Dzięki wstępnemu przetworzeniu, które jest realizowane w~tej pracy można do analizy całego obrazu wykorzystać mniejszą ilość danych.
