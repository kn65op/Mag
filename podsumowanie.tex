\chapter{Wnioski i podsumowanie}
\label{cha:podsumowanie}

Biorąc pod uwagę uzyskane wyniki testów udało się zrealizować cel pracy, jakim było stworzenie implementacji algorytmu Scale Space z~zastosowaniem biblioteki OpenCL. Testy poprawności pokazały, że zrealizowana implementacja jest zgodna (z~dokładnością do niewielkich różnic spowodowanych błędami w~obliczeniach numerycznych). Testy wydajnościowe wykazały, że implementacja wykorzystująca OpenCL jest szybsza niż realizacja algorytmu na procesorze ogólnego przeznaczenia CPU. Porównano również szybkość wykonywania obliczeń zrealizowanej biblioteki do szybkości działania rozwiązań zrealizowanych przy użyciu biblioteki OpenCV. Stworzona implementacja wykonuje obliczenia szybciej niż rozwiązanie opearte o~bibliotekę OpenCV wykorzystujące kartę graficzną. Uzyskane przyśpieszenie jest zależne od wielkości skali i~wynosi, pomijając najmniejszą skalę, od 2 do 12,5 przy błędzie mniejszym niż $ 0,8\% $. Dla obrazów o~rozdzielczości 512x512 pikseli oraz dla 4 skal osiągnięto działanie programu w~czasie rzeczywistym.

Można zauważyć, że poszczególne detektory różnią się od siebie. Wydaje się, że najlepiej działającym jest detektor narożników. W~przypadku kontynuacji prac warto byłoby przeanalizować i~udoskonalić działanie detektora grani.

Dalszymi krokami usprawniającymi przedstawione rozwiązanie może być umożliwienie wykorzystanie większej liczby kart graficznych, jeśli są dostępne w~danym systemie. Użycie większej ilości kart graficznym mogłoby umożliwić przetwarzanie obrazu wejściowego w~czasie rzeczywistym dla obrazów o~większej rozdzielczości.

Drugą możliwym kierunkiem rozwoju zrealizowanego systemu może być wykorzystanie otrzymanych informacji. Obecnie system realizuje detekcję krawędzi, narożników, plam oraz grani. Otrzymane informacje mogą być wykorzystane w~kolejnych krokach bardziej złożonego algorytmu, pozwalającego na analizę i~rozpoznawanie obiektów. Dzięki wstępnemu przetworzeniu, które jest realizowane w~tej pracy można do analizy obrazu wykorzystać mniejszą ilość danych. Tymi danymi będę miejsca (piksele), w~których znaleziono poszczególne strukury.
